\section{Introduction}\label{sec:intro}

We now give a proper presentation of our UTP theory
of concurrent programming, looking at both the condition-free command language
of the Views paper\cite{conf/popl/Dinsdale-YoungBGPY13},
as well as the more complete language covered in the UTPP paper\cite{DBLP:conf/icfem/WoodcockH02}.
in UTP.

All references in bold to things
like \textbf{Definition 9} or \textbf{Parameter E} refer to corresponding material
in the Views paper.
The generic definitions from\cite{conf/popl/Dinsdale-YoungBGPY13}
are in Appendix \ref{sec:viewdefs}.

\subsection{The Language Menagerie}

Both languages are concurrent with shared state.
The most abstract is the Command language from \cite{conf/popl/Dinsdale-YoungBGPY13}:
\[
 C ::= a \mid \cskip \mid C \cseq C \mid C \parallel C \mid C+C \mid C^*
\]
with non-deterministic choice and iteration.
Note that we use $\cseq$ and $+$ for sequential composition
and non-deterministic choice in the Command language,
rather than the $;$ and $\sqcap$ of UTP,
because their semantics do not coincide.
Unlike in the standard theory of Designs as applied to sequential programs,
concurrent versions of these constructs do not match the relational
calculus forms.


The language from \cite{DBLP:conf/icfem/WoodcockH02} replaces both
non-deterministic constructs by deterministic ones depending
on a condition over shared-state:
\[
 D ::= a \mid \cskip \mid D \cseq D \mid D \parallel D \mid D \dcond c D \mid c \ddo D
\]
We will also consider Process Modelling Language
(PML)\cite{DBLP:journals/infsof/AtkinsonWN07},
which is like the Views command language except that atomic actions
have a state-based guard:
\[
 M ::= c \pgrd a \mid M \cseq M \mid M \parallel M \mid M+M \mid M^*
\]
A key point to note is that the semantics of common constructs is the same
in every language.
We also note that for $M$ we can merge the guard with the state-change,
as a conjunction, so $c \pgrd a$ in $M$ corresponds to an atomic
action of the form $c \land a$ in $D$.
Given this observation we shall define a single semantics over
a universal concurrent programming language:
\[
 P ::= a \mid \cskip \mid P \cseq P  \mid P \parallel P
   \mid P+P\mid P^*
   \mid P \dcond c P \mid c \ddo P
\]
where $c \pgrd a$ is simply sugar for $c \land a$.

PML maps into $M$ the obvious way,
with
\begin{verbatim}
action A {
  requires { rr }
  provides { pr }
}
\end{verbatim}
turning into atomic action
\[
rr \subseteq rs ~~\pgrd~~ rs' = rs \cup pr
\]
Here the program state (formerly denoted by $s$)
has turned into $rs:\power Res$, where $Res$ is the type of resources.

\subsubsection{On Notation}

We make liberal use of a number of shorthand notations,
largely to reduce expression size and clutter.
Typically these reduce the use of infix operators,
and in one notable case, collapse large conjunctions
of implications into a simple form with strong visual cues.
Each shorthand is introduced by a subsection similar to this one.

\subsubsection{Structure of this paper}

Right now this paper is poorly structured,
organised roughly around the themes of
Related Work;
Commands;
Calculations;
Laws (\S\ref{sec:related}--\S\ref{sec:algebra}),
and and an appendix with Proofs (\S\ref{sec:proofs}).
It needs to be re-organised to provide a much better narrative,
because there really is a story to tell here.

A key principle has to be a focus on the higher level results and how
they knot together,
with detailed justifications kept ``offline'' in appendices.

We propose the following structure:
\begin{itemize}
  \item
     We introduce the language under study and our motivation for doing this.
     (This section)
     We also survey past work in this area (\S\ref{sec:related})
  \item
    We formally present the syntax of our language (\S\ref{sec:syntax}).
  \item
     We then explain the reasoning underlying our denotational
     UTP semantics and present it (\S\ref{sec:denote}).
  \item
     We then discuss how best to perform calculations with the theory,
     and show how this leads to the notional of a normal form
     (\S\ref{sec:calc}).
  \item
     Next we present the desired algebraic laws,
     and show how they can be proven given the denotational semantics
     (\S\ref{sec:algebra}).
  \item
     Finally we show how to derive an operational semantics from
     the denotational theory
     (\S\ref{sec:operational}).
\end{itemize}
In all cases detailed argument and proofs are left to appendices.
