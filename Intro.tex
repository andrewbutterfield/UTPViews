\section{Introduction}\label{sec:intro}

We now give a proper presentation of our UTP theory
of concurrent programming, looking at both the condition-free command language
of the Views paper\cite{conf/popl/Dinsdale-YoungBGPY13},
as well as the more complete language covered in the UTPP paper\cite{DBLP:conf/icfem/WoodcockH02}.
in UTP.

All references in bold to things
like \textbf{Definition 9} or \textbf{Parameter E} refer to corresponding material
in the Views paper.
The generic definitions from\cite{conf/popl/Dinsdale-YoungBGPY13}
are in Appendix \ref{sec:viewdefs}.

\subsection{The Language Menagerie}

Both languages are concurrent with shared state.
The most abstract is the Command language from \cite{conf/popl/Dinsdale-YoungBGPY13}:
\[
 C ::= a \mid \cskip \mid C \cseq C \mid C \parallel C \mid C+C \mid C^*
\]
with non-deterministic choice and iteration.
Note that we have replaced 
The language from \cite{DBLP:conf/icfem/WoodcockH02} replaces both
non-deterministic constructs by deterministic ones depending
on a condition over shared-state:
\[
 D ::= a \mid \cskip \mid D \cseq D \mid D \parallel D \mid D \dcond c D \mid c \ddo D
\]
We will also consider Process Modelling Language (PML)\cite{DBLP:journals/infsof/AtkinsonWN07}
which is like the Views command language except that atomic actions
have a state-based guard:
\[
 M ::= c \pgrd a \mid \cskip \mid M \cseq M \mid M \parallel M \mid M+M \mid M^*
\]
A key point to note is that the semantics of common constructs is the same
in every language.
Given this observation we shall define a single semantics over
a universal concurrent programming language:
\[
 P ::= c \pgrd a \mid \cskip \mid P \cseq P  \mid P \parallel P
   \mid P+P\mid P^*
   \mid P \dcond c P \mid c \ddo P
\]
where $a$ is simply $\true\pgrd a$.
