\section{TO DO}\label{ha:TO-DO}

\DRAFT{From original UTCP working document}

\subsection{Designs}

From the above we can see that a possible interpretation of our semantics as a
Design is
\RLEQNS{
   ok &\defs&  ls(in)
\\ ok' &\defs&  ls'(out)
}

\subsection{Healthiness Conditions}

We expect that the following three predicates
are mutually exclusive, but not exhaustive:
\RLEQNS{
   in &\in& ls
\\ out &\in& ls
\\ lab(g) \cap ls &\neq& \emptyset
}
where $in$, $out$, $g$ and $ls$ are in the alphabet.

What other healthiness conditions are there?
How does it all connect to a theory of designs?
Should atomic state change actions be considered as designs?

We can define some program predicates as follows:
\RLEQNS{
   justStarted(C)
   &\defs&
   in \in ls
\\ running(C)
   &\defs&
   \setof{in,out} \cap ls = \emptyset
   \land
   \exists \ell \in ls \bullet \ell \in labs(g)
\\ justStopped(C) &\defs& out \in ls
}
Note that $running(a)$ is always false.

How useful are these for reasoning?


\subsection{Laws}

We posit the following laws:
\RLEQNS{
   P \lseq ( Q \lseq R) &=& ( P \lseq Q ) \lseq R
   & \elab{$\lseq$-assoc}{;;-assoc}
\\ Idle \lseq P &=& P & \elab{$\lseq$-l-unit}{;;-l-unit}
\\ P \lseq Idle &=& P & \elab{$\lseq$-Q-unit}{;;-r-unit}
\\ P \parallel Q &=& Q \parallel P & \elab{$\parallel$-comm}{||-comm}
\\ P \parallel ( Q \parallel R) &=& ( P \parallel Q ) \parallel R
   & \elab{$\parallel$-assoc}{||-assoc}
\\ P \lcond{true} Q &=& P & \elab{$\lcond{true}$}{lcond-true}
\\ P \lcond{false} Q &=& Q & \elab{$\lcond{false}$}{lcond-false}
\\ c \wdo P &=& (P \lseq c \wdo P) \lcond c Idle
   & \elab{$\wdo$-unroll}{wdo-unroll}
}
The important question here is what do we mean by ``$=$''?
It looks like it asserts the existence of a label bijection,
that respects generators in some sense.

Perhaps we need to introduce the idea of a label relation, induced
as follows:
\RLEQNS{
   \ell_1 &\mapsto_A& \ell_2& \textbf{given } \A(A)[\ell_1,\ell_2,\ldots/in,out,\ldots]
\\ \ell_1 &\mapsto& \ell_2,\ell_3
   & \textbf{given } Split(\ell_2,\ell_3)[\ell_1,\ldots/in,\ldots]
\\ \ell_1,\ell_2 &\mapsto& \ell_3
   & \textbf{given } Merge(\ell_1,\ell_2)[\ell_3,\ldots/out,\ldots]
\\ \ell_1 &\mapsto_{c}& \ell_2
    & \textbf{given } Cond(\ell_2,c,\_)[\ell_1,\ldots/in,\ldots]
\\ \ell_1 &\mapsto_{\lnot c}& \ell_3
    & \textbf{given } Cond(\_,c,\ell_3)[\ell_1,\ldots/in,\ldots]
}
The bijection would need to respect this relation,
which is beginning to look like an operational semantics!
