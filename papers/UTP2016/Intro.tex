\section{Introduction}\label{sec:intro}

\DRAFT{From TASE submission --- needs revising and re-targeting}


In this paper we present results obtained while developing
an initial formal semantics for the Command language
in the Views paper \cite{conf/popl/Dinsdale-YoungBGPY13}
using the
Unifying Theorems of Programming (UTP) framework\cite{Hoare-He98}.
We define a UTP semantics for  shared global state concurrency,
inspired by the work of Woodcock and Hughes on unifying parallel programming (UTPP,\cite{DBLP:conf/icfem/WoodcockH02}).

\DRAFT{Revise this para. Part of our contribution is extending the UTP methodology
to use non-homogenous relation that mix dynamic state-change
(observations with before- and after-values)
along with static context parameters
(observations whose value is unchanged during program execution.
We also develop a notion of label generators to allow us to describe
flow of control in a compositional manner.}

The structure of this paper is as follows:
