\section{Related Work}\label{sec:related}

\DRAFT{From TASE submission --- also needs re-working}


Key work was done on concurrent semantics in the 80s and 90s,
with a strong focus on fully abstract denotational
semantics.
Notable work form this period includes that by
Stephen Brookes\cite{DBLP:journals/iandc/Brookes96}
and Frank de Boer and colleagues\cite{DBLP:conf/concur/BoerKPR91}.
Both looked at denotations based on the notion of sets of transition traces,
these being sequences of pairs of before-after states.
In order to get compositionality the traces of any program fragment
had to have arbitrary ``stuttering'' and ``mumbling'' state-pairs
added to capture the notion of outside interference.
Full abstraction meant that the semantics had to identify
programs like $skip\lseq skip$ with $skip$,
while distinguishing between $x:=2$ and $x:=1\lseq x:=x+1$.
This latter aspect required the language to be augmented with
an atomic wrapper construct.
A common feature of both was the very close linkage of the denotational
semantics to the operational one.
The work of Brookes\cite{DBLP:journals/iandc/Brookes96}
focussed on imperative languages with fair schedulers,
while that of de Boer et.al\cite{DBLP:conf/concur/BoerKPR91}
looked at a general framework (``failures of failures'')
that covered not just imperative programs
but also constraint solving systems
and \emph{asynchronous} versions of process algebras.

As already state earlier,
the key inspiration and starting point for the work presented here
was the UTPP paper\cite{DBLP:conf/icfem/WoodcockH02}.
This combined guarded commands\cite{1976:book:dijkstra}
with the idea of action systems\cite{PODC::BackK1983},
interpreted in UTP as non-deterministic choice
over guarded atomic actions,
where disabled actions behave the the unit for that choice.
This basic lattice-theoretic architecture for the UTPP semantics
forms the foundation for the UTCP semantics presented here.
