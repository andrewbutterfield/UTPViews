\section{Commands}

We do a quick run-down of the Commands\cite{conf/popl/Dinsdale-YoungBGPY13}.

\subsection{Syntax}

\begin{eqnarray*}
   a &\in& \Atom
\\ C &::=& a \mid \cskip \mid C \cseq C \mid C+C \mid C \parallel C \mid C^*
\\ g &:& Gen
\\ \ell &:& Lbl
\\ G &::=&  g \mid G_{:} \mid G_1 \mid G_2
\\ L &::=& \ell_G
\end{eqnarray*}

\subsection{Domains}
\begin{eqnarray*}
   s &\in& \mathcal S
\\ \sem{-} &:& \Atom \fun \mathcal S \fun \mathcal P(\mathcal S)
\\ new &:& Gen \fun Lbl \times Gen
\\ split &:& Gen\fun Gen \times Gen
\end{eqnarray*}
If $(\ell,g') = new(g)$ and $(g_a,g_b) = split(g)$ then:
\begin{eqnarray*}
   labs(g) &=& \setof\ell \cup labs(g') \cup labs(g_a) \cup labs (g_b)
\\ \setof\ell \cap labs(g') &=& \emptyset
\\ labs(g_a) \cap labs(g_b) &=& \emptyset
\end{eqnarray*}

\subsection{Alphabet}

\begin{eqnarray*}
   s, s' &:& \mathcal S
\\ ls, ls' &:& \mathcal P (Lbl)
\\ g &:& Gen
\\ in, out &:& Lbl
\end{eqnarray*}

\subsection{``Standard'' UTP Constructs}

\begin{eqnarray*}
   P \cond c Q
   &\defs&
   c \land P \lor \lnot c \land Q
\\ P ; Q
   &\defs&
   \exists s_m,ls_m \bullet P[s_m,ls_m/s',ls'] \land Q[s_m,ls_m/s,ls]
\\ c * P
   &=&
   P ; c * P \cond c \Skip
\end{eqnarray*}

\subsection{Shorthands}

\begin{eqnarray*}
   ls(\ell) &\defs& \ell \in ls
\\ ls(L) &\defs& L \subseteq ls
\\ \B{ls(\ell)} &\defs& \ell \notin ls
\\ \B{ls(L)} &\defs& L \cap ls = \emptyset
\\ ls(L) \land ls(M) &=& ls(L \cup M)
\\ \B{ls(L)} \land \B{ls(M)} &=& \B{ls(L \cup M)}
\\
\\ S \ominus (T|V)
   &\defs& (S \setminus T) \cup V
\\ S \ominus (r_1,\dots,r_m|a_1,\dots,a_n)
   &\defs& (S \setminus \setof{r_1,\dots,r_m}) \cup \setof{a_1,\dots,a_n}
\\
\\ \ell_G &\defs& \pi_1 (new (G))
\\ G_: &\defs& \pi_2 (new (G))
\\ G_1 &\defs& \pi_1 (split(G))
\\ G_2 &\defs& \pi_2 (split(G))
\\
\\ ii &\defs& s'=s
\\ ~[a] &\defs& s' \in \sem a s
\end{eqnarray*}

\subsection{Operational Semantics}

\def\arr#1{\stackrel{#1}{\fun}}
\def\arrs{\fun^*}
From \cite{conf/popl/Dinsdale-YoungBGPY13}.
\begin{mathpar}
\inferrule{C_1 \arr\alpha C'_1}{C_1 \cseq C_2 \arr\alpha C'_1 \cseq C_2}
\and
\inferrule{ }{\cskip \cseq C_2 \arr{ii} C_2}
\and
\inferrule{ }{C_1 + C_2 \arr{ii} C_i} i \in \setof{1,2}
\and
\inferrule{ }{C^* \arr{ii} C\cseq C^*}
\and
\inferrule{ }{C^* \arr{ii} C\cseq\cskip}
\and
\inferrule{ }{\cskip \parallel C_2 \arr{ii} C_2}
\and
\inferrule{ }{C_1 \parallel \cskip \arr{ii} C_1}
\and
\inferrule{ }{a \arr{a} \cskip}
\and
\inferrule{C_1 \arr\alpha C'_1}{C_1 \parallel C_2 \arr\alpha C'_1 \parallel C_2}
\and
\inferrule{C_2 \arr\alpha C'_2}{C_1 \parallel C_2 \arr\alpha C_1 \parallel C'_2}
\\\\
\inferrule{ }{C,s \arrs C,s}
\and
\inferrule
 {C_1 \arr\alpha C_2
  \\
  s_2 \in \sem\alpha(s_1)
  \\
  C_2,s_2 \arrs C_3,s_3
 }
 {C_1,s_1 \arrs C_3,s_3}
\end{mathpar}





\subsection{UTP Denotational Semantics}

This follows the TASE2016 Submission style,
with a simpler version of $run$.

\begin{eqnarray*}
   a
   &\defs&
   ls(in) \land [a] \land ls'=ls \ominus (in|out)
\\ \cskip
   &\defs&
   ls(in) \land ii \land ls'=ls \ominus (in|out)
\\ C \cseq D
   &\defs&
   C[g_{:1},\ell_g/g,out]
\\ &\lor&
   D[g_{:2},\ell_g/g,in]
\\ C + D
   &\defs&
   ls(in)
   \land ii
   \land ls' \in \{ ls \ominus (in,\ell_{g2}|\ell_{g1})
                  , ls \ominus (in,\ell_{g1}|\ell_{g2})\}
\\ &\lor&
   C[g_{1:},\ell_{g1}/g,in]
\\ &\lor&
   D[g_{2:},\ell_{g2}/g,in]
\\ C \parallel D
   &\defs&
   ls(in)
   \land s' = s
   \land ls'= ls \ominus (in|\ell_{g1},\ell_{g2})
\\ &\lor&
   C[g_{1::},\ell_{g1},\ell_{g1:}/g,in,out]
\\ &\lor&
   D[g_{2::},\ell_{g2},\ell_{g2:}/g,in,out]
\\ &\lor&
   ls(\ell_{g1:},\ell_{g2:})
   \land ii
   \land ls'= ls \ominus (\ell_{g1:},\ell_{g2:}|out)
\\ C^*
   &\defs&
   ls(in)
   \land ii
   \land ls' \in \{ ls \ominus (in|out), ls \ominus (in|\ell_g)\}
\\ &\lor&
   C[g_{:},\ell_g,in/g,in,out]
\\ run(C)
   &\defs&
   ( \B{ls(out)} * C)[in/ls]
\end{eqnarray*}

\subsection{Calculations}

Let us do some simple calculations.

\subsubsection{$run(a)$}

\begin{eqnarray*}
  && run(a)
\EQ{defn $run$}
\\&& ( \B{ls(out)} * a)[in/ls]
\EQ{will do at least one iteration}
\\&& (\B{ls(out)} \land a \seq run(a))[in/ls]
\EQ{$(P;Q)[e/x] = P[e/x];Q$, for undashed $x$}
\\&& (\B{ls(out)} \land a) [in/ls] \seq \B{ls(out)} * a
\EQ{defn $a$}
\\&& (\B{ls(out)}
      \land ls(in) \land [a] \land ls'=ls \ominus (in|out))
      [in/ls]
      \seq \B{ls(out)} * a
\EQ{substitution}
\\&& \B{in(out)}
      \land in(in) \land [a] \land ls'=in \ominus (in|out)
      \seq \B{ls(out)} * a
\EQ{set theory, propositional logic}
\\&& [a] \land ls'=out \seq \B{ls(out)} * a
\EQ{push after-obs through $\seq$}
\\&& [a] \seq  ls=out \land \B{ls(out)} * a
\EQ{$\lnot c \land c*P = \lnot c \land \Skip$}
\\&& [a] \seq  ls=out \land \Skip
\EQ{Easy UTP calculation}
\\&& [a] \land  ls'=out
\end{eqnarray*}
We have shown that
an execution of an atomic action $a$ is
\begin{eqnarray*}
   a &=& [a] \land  ls'=out
\\   &=& s' \in \sem a s \land ls'=out
\end{eqnarray*}
We end with a predicate with only $s,s',ls'$ as free variables.
This is a pattern that recurs.

\subsubsection{$run(\cskip)$}

\begin{eqnarray*}
  && run(\cskip)
\EQ{the only difference is that $ii$ replaces $[a]$}
\\&& ii \land  ls'=out
\end{eqnarray*}

\subsubsection{$run(\cskip \cseq a)$}

\begin{eqnarray*}
  && run(\cskip \cseq a)
\EQ{First three steps of $run(a)$ calculation apply.}
\\&& (\B{ls(out)} \land (\cskip \cseq a)) [in/ls] \seq \B{ls(out)} * a
\EQ{Defn. $\cseq$}
\\&& ( \B{ls(out)}
       \land ( \cskip[\g{:1},\ell_g/g,out]
               \lor
                a[\g{:2},\ell_g/g,in]
              )
     ) [in/ls]
     \seq \B{ls(out)} * (\cskip \cseq a)
\EQ{Substitution}
\\&& \B{in(out)}
       \land ( \cskip[\g{:1},\ell_g,in/g,out,ls]
               \lor
                a[\g{:2},\ell_g,in/g,in,ls]
              )
     \seq \B{ls(out)} * (\cskip \cseq a)
\EQ{Distributivity of $\land$, and $\seq$ w.r.t $\lor$}
\\&& \B{in(out)}
       \land \cskip[\g{:1},\ell_g,in/g,out,ls]
     \seq \B{ls(out)} * (\cskip \cseq a)
\\&\lor&
     \B{in(out)}
       \land a[\g{:2},\ell_g,in/g,in,ls]
     \seq \B{ls(out)} * (\cskip \cseq a)
\EQ{defns of $\cskip$ and $a$}
\\&& \B{in(out)}
       \land (ls(in) \land ii \land ls'=ls \ominus (in|out))[\g{:1},\ell_g,in/g,out,ls]
     \seq \B{ls(out)} * (\cskip \cseq a)
\\&\lor&
     \B{in(out)}
       \land (ls(in) \land [a] \land ls'=ls \ominus (in|out))[\g{:2},\ell_g,in/g,in,ls]
     \seq \B{ls(out)} * (\cskip \cseq a)
\EQ{substitution}
\\&& \B{in(out)}
       \land in(in) \land ii \land ls'=in \ominus (in|\ell_g)
     \seq \B{ls(out)} * (\cskip \cseq a)
\\&\lor&
     \B{in(out)}
       \land in(\ell_g) \land [a] \land ls'=in \ominus (\ell_g|out)
     \seq \B{ls(out)} * (\cskip \cseq a)
\EQ{simplification, noting $in(\ell_g)=\false$}
\\&& ii \land ls'=\ell_g
     \seq \B{ls(out)} * (\cskip \cseq a)
\EQ{Push $ls'$ through}
\\&& ii \seq ls=\ell_g
     \land \B{ls(out)} * (\cskip \cseq a)
\EQ{$ii \seq P=P$, loop condition still true, unroll}
\\&& ls=\ell_g
     \land (\cskip \cseq a) \seq \B{ls(out)} * (\cskip \cseq a)
\EQ{Defn $\cseq$, distributivity of most things w.r.t. $\lor$}
\\&& ls=\ell_g
     \land \cskip[\g{:1},\ell_g/g,out] \seq \B{ls(out)} * (\cskip \cseq a)
\\&\lor&
     ls=\ell_g
     \land a[\g{:2},\ell_g/g,in] \seq \B{ls(out)} * (\cskip \cseq a)
\EQ{Defns.  $\cskip$ and $a$, with substitutions}
\\&& ls=\ell_g
     \land ls(in) \land ii \land ls'=ls \ominus (in|\ell_g) \seq \B{ls(out)} * (\cskip \cseq a)
\\&\lor&
     ls=\ell_g
     \land ls(\ell_g) \land [a] \land ls'=ls \ominus (\ell_g|out) \seq \B{ls(out)} * (\cskip \cseq a)
\EQ{Simplification, noting $ls=\ell_g$ and so $ls(in) = \false$}
\\&& [a] \land ls'=out \seq \B{ls(out)} * (\cskip \cseq a)
\EQ{$P \land c' \seq \lnot c * Q = P \land c'$}
\\&& [a] \land ls'=out
\end{eqnarray*}
We have shown $\cskip \cseq a = a$!

\subsubsection{$run(\cskip \cseq C)$}

\begin{eqnarray*}
  && run(\cskip \cseq C)
\EQ{First three steps of $run(a)$ calculation apply.}
\\&& (\B{ls(out)} \land (\cskip \cseq C)) [in/ls] \seq \B{ls(out)} * (\cskip \cseq C)
\EQ{Defn. $\cseq$}
\\&& ( \B{ls(out)}
       \land ( \cskip[\g{:1},\ell_g/g,out]
               \lor
                C[\g{:2},\ell_g/g,in]
              )
     ) [in/ls]
     \seq \B{ls(out)} * (\cskip \cseq C)
\EQ{Substitution}
\\&& \B{in(out)}
       \land ( \cskip[\g{:1},\ell_g,in/g,out,ls]
               \lor
                C[\g{:2},\ell_g,in/g,in,ls]
              )
     \seq \B{ls(out)} * (\cskip \cseq C)
\EQ{Distributivity of $\land$, and $\seq$ w.r.t $\lor$}
\\&& \B{in(out)}
       \land \cskip[\g{:1},\ell_g,in/g,out,ls]
     \seq \B{ls(out)} * (\cskip \cseq C)
\\&\lor&
     \B{in(out)}
       \land C[\g{:2},\ell_g,in/g,in,ls]
     \seq \B{ls(out)} * (\cskip \cseq C)
\EQ{defn of $\cskip$ }
\\&& \B{in(out)}
       \land (ls(in) \land ii \land ls'=ls \ominus (in|out))[\g{:1},\ell_g,in/g,out,ls]
     \seq \B{ls(out)} * (\cskip \cseq C)
\\&\lor&
     \B{in(out)}
       \land C[\g{:2},\ell_g,in/g,in,ls]
     \seq \B{ls(out)} * (\cskip \cseq C)
\EQ{substitution}
\\&& \B{in(out)}
       \land in(in) \land ii \land ls'=in \ominus (in|\ell_g)
     \seq \B{ls(out)} * (\cskip \cseq C)
\\&\lor&
     \B{in(out)}
       \land C[\g{:2},\ell_g,in/g,in,ls]
     \seq \B{ls(out)} * (\cskip \cseq C)
\EQ{simplification}
\\&& ii \land ls'=\ell_g
     \seq \B{ls(out)} * (\cskip \cseq C)
\\&\lor&
     \B{in(out)}
       \land C[\g{:2},\ell_g,in/g,in,ls]
     \seq \B{ls(out)} * (\cskip \cseq C)
\end{eqnarray*}
At this stage we pause.
Intuitively, we know that none of the atomic parts
inside of $ C[\g{:2},\ell_g,in/g,in,ls]$ can be enabled:
\begin{itemize}
  \item
    $C$ refers to $ls$, $in$, $out$
    and maybe other labels in $labs(g)$.
  \item
    $C[\g{:2},\ell_g,in/g,in,ls]$
    now has mappings as follows:
    \begin{eqnarray*}
       ls &\mapsto& in
    \\ in &\mapsto& \ell_g
    \\ out &\mapsto& out
    \\ labs(g) &\mapsto& labs(\g{:2})
    \end{eqnarray*}
  \item
    Assuming that $C$ is ``sensible'' then it requires
    $ls(in)$ to get started,
    which in $C[\g{:2},\ell_g,in/g,in,ls]$
    becomes the condition $in(\ell_g)$, that is $\false$.
  \item
    So we should conclude that here
    $C[\g{:2},\ell_g,in/g,in,ls]=\false$,
    as it can never get started.
\end{itemize}
Let us do so, and continue:
\begin{eqnarray*}
  && ii \land ls'=\ell_g
     \seq \B{ls(out)} * (\cskip \cseq C)
\EQ{Push $ls'$ through}
\\&& ii \seq ls=\ell_g
     \land \B{ls(out)} * (\cskip \cseq C)
\EQ{$ii \seq P=P$, loop condition still true, unroll}
\\&& ls=\ell_g
     \land (\cskip \cseq C) \seq \B{ls(out)} * (\cskip \cseq C)
\EQ{Defn $\cseq$, distributivity of most things w.r.t. $\lor$}
\\&& ls=\ell_g
     \land \cskip[\g{:1},\ell_g/g,out] \seq \B{ls(out)} * (\cskip \cseq C)
\\&\lor&
     ls=\ell_g
     \land C[\g{:2},\ell_g/g,in] \seq \B{ls(out)} * (\cskip \cseq C)
\EQ{Defn  $\cskip$ with substitutions}
\\&& ls=\ell_g
     \land ls(in) \land ii \land ls'=ls \ominus (in|\ell_g) \seq \B{ls(out)} * (\cskip \cseq C)
\\&\lor&
     ls=\ell_g
     \land C[\g{:2},\ell_g/g,in] \seq \B{ls(out)} * (\cskip \cseq C)
\EQ{Simplification, noting $ls=\ell_g$ and so $ls(in) = \false$}
\\&&      ls=\ell_g
     \land C[\g{:2},\ell_g/g,in] \seq \B{ls(out)} * (\cskip \cseq C)
\EQ{can do a little substitution `fusion' here}
\\&&  (ls=in \land C)[\g{:2},\ell_g/g,in] \seq \B{ls(out)} * (\cskip \cseq C)
\end{eqnarray*}
We pause once more.
This has been transformed into the `execution' of $C$
in a context where it's start ($in$) label is present,
but as seen from outside $C$, so its $in$ appears to us as $\ell_g$,
and its $g$ has become what we, from the top-level,
could consider to be $\g{:2}$.
As we have handed the case of atomic $C$,
we can assume $C$ is non-atomic.
Now all non-atomic commands have a form that flattens
out into a finite disjunction of atomic actions
of the form
\[
   ls(L) \land [a] \land ls'=ls\ominus(L|M)
\]
where both $L$ and $M$ are either subsets of $lab(g)$,
or are precisely one of the sets $\setof{in}$ or $\setof{out}$.
(as seen by the composite), and here $[a]$ can also be $ii$.
In particular every flattened composite contains
exactly one instance of each of the following two
atomic action forms:
\begin{align*}
   ls(in) \land [a] \land ls'=ls\ominus(in|L)
\\ ls(L) \land [a] \land ls'=ls\ominus(L|out)
\end{align*}
where $\setof{in,out}\cap L = \emptyset$.
So in any history of the execution of commands
we find that no actions of any composite
will occur until $in$ (as see by it) appears in $ls$.
At this point precisly one action will be enabled:
\[
 ls(in) \land [a] \land ls'=ls\ominus(in|L)
\]
Once scheduled, $in$ be removed, $L$ added to $ls$,
enabling other actions in that composite.
Again careful inspection of the semantics shows that there
is a ``flow'' that ensures that every action gets
enabled eventually, changing $ls$
until the point where $N$ (say) is in $ls$,
and the following action is enabled:
\[
 ls(N) \land [a] \land ls'=ls\ominus(N|out)
\]
Once this runs, $out$ will be in $ls$,
and this composite will be completely inactive,
either till program execution terminates altogether,
or some outside agency, (another command) places
this commands $in$-label back into $ls$.

Looking back at our calculation
of $run(\cskip \cseq C)$
we see we have got the position where
and execution of $C$ has been setup and ready to go.
We can now state that in
\[
(ls=in \land C)[\g{:2},\ell_g/g,in]
\seq
\B{ls(out)} * (\cskip \cseq C)
\]
we shall see,
from $C$'s perspective,
that  only the $ls(in)$ enabled action happens,
and then $in$ will be out of $ls$,
and some labels in $labs(g)$ will be in,
but not $out$.
So the loop condition will still hold and we will go
around again, with
\[
\setof{in,out} \cap ls = \emptyset
\land labs(g) \cap ls \neq \emptyset
\]
as seen by $C$.
From our perspective at top-level
this property appears as
\[
\setof{\ell_g,out} \cap ls = \emptyset
\land labs(\g{2:}) \cap ls \neq \emptyset
\]
The loop should continue to run until
the final action of $C$ is enabled,
after which we should find $out$ in $ls$
and so $run$ will be done.


\DRAFT{
THIS IS A LOT OF HAND-WAVING!
HOW DO WE FORMALISE THIS?
}
In some sense, what we have done above,
is try to calculate a single ``step'' of
$\cskip \cseq C$, as per the operational semantics
rule:
\[
  \inferrule
    { }
    {\cskip\cseq C \arr{ii} C}
\]
We saw the execution of $\cskip$ and its state change $ii$,
and we are left, not with $C$ exactly,
but something that we hope will behave just like $C$.

\subsubsection{Taking Stock}

\DRAFT{This bit was written before all the material immediately
above discussing the calculation of $\cskip\cseq C$}

We have a notion of ``sensible'' predicates $C$
as being ones that require $ls(in)$ to hold
in order to start or become alive,
and only then will events unfold that allow $in$
to be removed from $ls$ and elements of $labs(g)$
to flow in and out of $ls$, until we are left
with $out$ in $ls$,
at which point $C$'s execution is complete.
If we are in a state where we know that $ls(in$)
is false,
then any disjunctive branch with $C$ as its prefix
will simply disappear.

For $C[\g{:2},\ell_g,in/g,in,ls]$ as the prefix
of a sequence, $ls(in)$ becomes $in(\ell_g)$
which is always false, since in a well-formed context
$in$ and $\ell_g$ are always distinct labels.
