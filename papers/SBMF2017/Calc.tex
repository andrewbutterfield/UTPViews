\section{Calculations}\label{sec:calc}

This semantic theory was validated by a series of test calculations
done to ensure that it was making the right predications about program behaviour.
This typically involved taking small programs with a few atomic actions
and trying to simplify their semantic predicates
down to a non-deterministic choice of atomic action sequences.
Some of the calculations proved to be very long,
repetitive and tedious,
motivating the UTP-calculator development \cite{DBLP:conf/utp/Butterfield16}.

We shall start by sketching out a test calculation for $\catom a$,
where the objective is to reduce it down to a predicate involving
just basic atoms.
\RLEQNS{
  & & \lhsCA
\\&=& \rhsCA                                    & \elabel{\lblCA}
\\&=& DL \land LE \land \WWW(A(in|a|out))       & \elabel{\lblW}
\\&=& DL \land LE \land \bigvee_i A(in|a|out)^i & \elabel{\lblWWW}
}
At this point what remains is to compute $A(in|a|out)^i$ for $i \in \Nat$.
The cases of $i=0,1$ are straightforward.
Computing $i=2$ is easy:
\RLEQNS{
  &  & A(in|a|out) \seq A(in|a|out)     & \elabel{X-def}
\\&=& X(in|a|in|out) \seq X(in|a|in|out) &\elabel{X-then-X}
\\&=& \setof{in} \cap (\setof{in}\setminus\setof{out}) = \emptyset
      \land
      X(\dots) & \mbox{set theory}
\\&=& \false \land X(\dots)
}
We see that $A(in|a|out)^2 = \false$, and as $\false$ is a zero for
semantic sequential composition,
we can deduce that $A(in|a|out)^i = \false$ for all $i \geq 2$.
So our final result is
\begin{equation}
  \catom a = DL \land LE \land(\Skip \lor A(in|a|out))
\end{equation}
Ignoring the healthiness conditions,
this boils down to two possible observations we can make of $\catom a$:
either we observe suttering---no change in state or label-sets ($\Skip$)
or we see the complete execution of the underlyng basic action $A(in|a|out)$.

Test calculations for simple usage of most of the composites is essentially the
same.
One slight complication is that the contents of $\WWW$ in theses cases
is a disjunction of terms, rather than a single basic action,
so we first simplify these out, applying all substitutions,
to get a term $Q$ of the form ($\Skip \lor \textit{basic actions}$).
We need to compute $Q^i$ for $i \geq 2$,
and sequential composition distributes through disjunction,
so we obtain resulting terms of the same form,
by repeated application of law \elabel{X-then-X}.
A large number of these have results
with the set side-condition that evaluates to \false,
as per the $i=2$ example above---these terms vanish.
There are other terms produced that do not vanish,
but some of these can also be eliminated,
because their enabling set violates the Label Exclusivity invariant.
All remaining terms have the form $X(E|a|R|N)$,
and some of these can be immediately re-written to $A(E|a|N)$,
if $R=E$.
In every test calculation we have done
it turns out that the others, where $R\neq E$ can also be re-written,
because $LE$ says that none of $R \setminus E$ can be present in $ls$
when anything from $E$ is present,
so the removal of those labels is ineffective, as they are never present
when that action is enabled.
So, the outcome is that we get final results where every basic action
can be written in the $A$-form.
All of these aspects of these test calculations are supported by current
versions of the tool described in \cite{DBLP:conf/utp/Butterfield16}.
If there is no use of the iteration construct $(P^*)$,
then all calculations terminate because
there is always some $i$ for which $Q^i$ evaluates to \false.
Any use of the language iteration construct however
results in having terms for all values of $i$.

Some calculation results are shown in Fig. \ref{fig:calc-results}.
\begin{figure}[t]
\RLEQNS{
   \catom a \cseq \catom b
    &=&
    \Skip \lor A(in|a|\ell_g) \lor A(\ell_g|b|out) \lor A(in|ab|out)
\\ \catom a + \catom b
   &=& \Skip \lor A(in|ii|\ell_{g1}) \lor A(in|ii|\ell_{g2}) \lor A(\ell_{g1}|a|out)
\\ & &{} \lor A(\ell_{g2}|b|out) \lor A(in|a|out) \lor A(in|b|out)
\\ \catom a \parallel \catom b
   &=&
   \Skip \lor A(in|ii|\ell_{g1},\ell_{g2})
         \lor A(\ell_{g1:},\ell_{g2:}|ii|out)
         \lor A(\ell_{g1}|a|\ell_{g1:})
\\ & & {}
         \lor A(\ell_{g2}|b|\ell_{g2:})
         \lor A(in|a|\ell_{g1:},\ell_{g2})
         \lor A(in|b|\ell_{g2:},\ell_{g1})
\\ & & {}
         \lor A(\ell_{g1},\ell_{g2}|ba|\ell_{g1:},\ell_{g2:})
         \lor A(\ell_{g1},\ell_{g2}|ab|\ell_{g1:},\ell_{g2:})
\\ & & {}
         \lor A(\ell_{g2:},\ell_{g1}|a|out)
         \lor A(\ell_{g1:},\ell_{g2}|b|out)
         \lor A(in|ba|\ell_{g1:},\ell_{g2:})
\\ & & {}
         \lor A(in|ab|\ell_{g1:},\ell_{g2:})
         \lor A(\ell_{g1},\ell_{g2}|ba|out)
         \lor A(\ell_{g1},\ell_{g2}|ab|out)
\\ & & {}
         \lor A(in|ba|out)
         \lor A(in|ab|out)
}
\caption{
  Some Test Calculation Results.
  Here $ab$ ($ba$) is short for $a;b$ ($b;a$), and we have omitted
  the $DL$ and $LE$ invariants for clarity.
}
\label{fig:calc-results}
\end{figure}
If we look at the result for $\catom a \cseq \catom b$
we have $\Skip$, the stuttering step,
and $A(in|ab|out)$ which is
the complete exection of both actions without interference (mumbling),
and $A(in|a|\ell_g)$ that shows the execution of $a$, to an intermediate
point where $b$ has yet to occur.
These three observations are consistent with the idea that
our predicates are relations between a starting state and some subsequent
or final state.
However we also have action $A(\ell_g|b|out)$,
which is an observation that begins after action $a$ has already occured,
and just observes the behaviour of $b$ alone.
What has happened with this UTP theory of concurrency
is that it is now no longer insists that the ``before'' observation
is pinned to be the start of the program. Now we are able to observe program
behaviour that can both start and end at what are intermediate points
in the lifetime of the program.

If we look at $\catom a + \catom b$, we also explictly
see the control-flow ``decisions'',
such as $A(in|ii|\ell_{g1})$ where the decision to execute $a$ is made.
This will remove $in$ from $ls$ if it runs,
so disabling the other choice, denoted by $A(in|ii|\ell_{g2})$.
By contrast, in $\catom a\parallel \catom b$ the initially enabled
action is $A(in|ii|\ell_{g1},\ell_{g2})$, which activates both $a$ and $b$.
The control flow action $A(\ell_{g1:},\ell_{g2:}|ii|out)$ delays termination
until both atomic actions are done.

Finally,
we stress that the explicit inclusion of labels in the final results
is essential in order to ensure compositionality.
In \cite{conf/tase/BMN16} we had the explicit $run$ form,
and this reduced the semantics of $\catom a$ to $a$,
that of $\catom a + \catom b$ to $a \lor b$
and $\catom a \parallel \catom b$ to $ab\lor ba$.
While this looks cleaner, it has lost too much information,
and we cannot compose these further to get correct answers.
With the explcitly labelled semantics presented here for UTCP,
we can, for example, correctly compute $(\catom a \cseq \catom b) \parallel \catom c$
by replacing the first $\cseq$ term by its expansion
from Fig. \ref{fig:calc-results}.
