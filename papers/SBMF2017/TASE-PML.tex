\section{Process Modelling Language}\label{ha:PML}

Process Modelling Language (PML) \cite{DBLP:journals/infsof/AtkinsonWN07}
is a shared-state concurrent imperative language
describing named basic actions ($N?rr!pr$) as
activities that require resources to start ($rr$)
and provide resources when they complete ($pr$).
Actions are non-destructive in that the required resources are
not consumed but remain in place.
Its concrete syntax is C-like, but we present a simpler abstract syntax here:
\RLEQNSs{
   N \in Name
\\ A \in Action &::=& N?rr!pr
\\ P,Q \in PML
   &::=&
   A
   \mid
   P \lseq Q
   \mid
   P \pcond Q
   \mid
   P \parallel Q
   \mid
   \piter P
}
We use $P \lseq Q$ and $P \parallel Q$ to denote
respectively sequential and parallel composition.
Both the conditional ($P \pcond Q$) and the loop construct $\piter P$
are unusual in that they have no explicit conditions.
Instead the decision of which branch to take or whether or not to end the loop
is left unspecified---this is one aspect of the flexible nature of the language.
In its original intended use, it was left to the people enacting the business
process described by PML to use their judgement to determine which branch to take
or when a process has been repeated enough.

We are developing a formal PML semantics framework which
 can investigate and relate at least three interpretations of
a PML description:
\begin{itemize}
  \item \textbf{Strict:}
    the behaviour follows strictly according to the control-flow structure,
    becoming deadlocked if control requires execution of actions
    whose required resources are not available.
  \item \textbf{Flexible:}
    the behaviour is guided by the control flow,
    but actions can run out of sequence if their required resources become available
    before the control flow has determined that they should start.
  \item \textbf{Weak:}
    control flow is completely ignored,
    and execution simply iterates the non-deterministic choice of actions
    whose resources are available (also known as the ``dataflow interpretation'').
\end{itemize}
In our framework, the three interpretations form a refinement hierarchy:
\begin{equation*}
  Weak \refinedby Flexible \refinedby Strict
\end{equation*}

As an example, consider the following PML description:
\[
 A?!r_1 \lseq (B?r1!r_2 \parallel C?r_2!r_3) \lseq D?r_1!r_4
\]
In the strict interpretation, only one sequence is possible,
namely $A,B,C,D$ because (i) $A$ is required by control-flow
to precede both $B$ and $C$, and (ii) $D$ is forced to be last.
In the weak interpretation, we ignore the flow control,
and are left with resource constraints only
($B$ must follow $A$, $C$ must follow $B$ and $D$ must follow$A$)
This admits three possible enactment sequences:
$A,B,C,D$,
or $A,D,B,C$,
or $A,B,D,C$.

In this paper we focus on the weak semantics,
and on developments we have done within the Unifying Theories
of Programming (UTP) framework \cite{Hoare-He98} to create a semantic baseline
on top of which both the flexible and strict semantics can be constructed.
