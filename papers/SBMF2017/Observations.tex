\section{Observations}\label{sec:observe}

Any UTP theory has to clearly define its \emph{alphabet},
that is, the set of observational variables that define
its domain of discourse.
The theory presented here is inspired by UTPP\cite{DBLP:conf/icfem/WoodcockH02}
and uses some of the observations presented there:
the values associated with all (shared) variables
are not mentioned individually, but instead are lumped together;
and we assume that all actions are labelled and that we can observe
the set of labels that are considered to be ``active''.
\begin{eqnarray}
   s, s' &:& State
\\ ls, ls' &:& \power Lbl
\end{eqnarray}
The theory in \cite{DBLP:conf/icfem/WoodcockH02} also uses $ok$ and $ok'$,
but these are not required here
---they can added in though as derived observations, if wanted.

The role of label-generators is rather different, however.
They will be used to generate labels for statements,
and we do not want these to change during the lifetime of the program.
We will also want to be able to refer in a general way to two key labels
associated with any language construct,
namely the label ($in$) that is used to enable the starting of a construct,
and the label ($out$) that is used to signal that the construct has just terminated.
\begin{eqnarray}
   in, out &:& Lbl
\\ g &:& Gen
\end{eqnarray}
These observations are \emph{static},
in that their values do not change during program execution.
Instead, these variables record context-sensitive information about
how a language construct is situated with respect to its ``neighbours'',
in a way that permits a compositional approach.
For details of how this works, see Section \ref{sec:composing}.



Our semantics is built upon basic atomic state-change
actions that modify global shared state $s$.
The concurrent flow of control is managed using a global dynamic label-set $ls$
and the association of two distinguished labels $in$ and $out$,
and a label generator $g$ with every language construct.

This brings us to an important distinction between the usual approach
taken by UTP regarding the distinction between syntax and semantics.
The usual approach,
inspired by the slogan ``programs are predicates''\cite{predprog,conf/mlpl/Hoare85},
is to treat syntax and semantics as the same thing.
A program's syntax is simply a shorthand notation for its semantics.
So, the program text \texttt{x := x+y} \emph{is} a predicate,
a shorthand for the more verbose $x' = x + y \land y' = y$
\footnote{Assuming \texttt{x} and \texttt{y} are the only variables}
.
in particular, the notation for sequential composition, $P ; Q$,
is a shorthand for
$\exists obs_m @ P[obs_m/obs'] \land Q[obs_m/obs]$,
where $obs$ ($obs'$) refers to all the before- (after-)observations.
This ``punning'' between syntax and semantics largely works for theories
of sequential programs or local-state concurrency,
mainly because sequences of code lead to simple semantic sequencing.
However, in global shared-variable concurrency,
code sequences get broken up by interference from parallel execution threads,
and there is no longer a simple correspondence between syntactical and semantic
sequencing.

Here we shall use the notation $P;Q$
to denote \emph{semantic} sequential composition,
which means that the execution of $P$ is immediately followed by the
execution of $Q$, without any intervening external interfence.
We define it as follows:
\begin{equation}
   P ; Q \defs \exists s_m, ls_m @ P[s_m,ls_m/s',ls'] \lor Q[s_m,ls_m/s,ls]
   \label{eqn:semantic-;}
\end{equation}
The key thing to note is that this definition makes no reference at all
to $g$, $in$ or $out$,
as these are static observations.
A desired consequence of this is that
substitutions $\gamma$ that
replace $g$, $in$ and/or $out$
by expressions that only involve the static variables,
will distribute through semantic sequential composition:
\begin{equation}
  (P ; Q)\gamma = (P\gamma) ; (Q\gamma)
\end{equation}
