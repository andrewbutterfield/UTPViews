\section{Atomic Actions}\label{sec:atomic}

An atomic action ($a$) is simply a global state transformer whose effects,
once started,
occur immediately and completely, without any external interference.
We can consider it be a relational predicate that only mentions $s$ and $s'$.
In the semantics of UTPP, and here for UTCP,
every atomic action is labelled.
Flow of control is managed by keeping a dynamic record
of which labels are considered current, or ``enabled''.
The behaviour of an atomic action is that it exhibits none
until its label is enabled.
Noting that many atomic actions can be enabled at once,
what happens is that one of actions is selected non-deterministically to run.
The action so selected transforms the global state,
and then the control-flow management marks its label as disabled,
and enables labels of atomic action that can immediately follow it
according to the control-flow structure of the program.

In order to define this behaviour computationally,
we depart from the technique used in UTPP based on explicit program labels.
Instead we use the static observables $in$ and $out$ to explicitly indicate
the enabling label --- $in$--- for the action,
and a disabling, or ``done'' label---$out$---, for the same action.
In effect we are exploiting the fact that our language is block-structured
with only one entry and exit point for each construct,
in order to be able to decouple the semantics of an atomic action
from whatever might come after.
Dealing with that is the responsibiity of the semantics of language composites.

As already stated,
we use $a$ to denote the predicate describing the core global state-changes,
and  use $ls$ and $ls'$ to record the set of enabled labels both before
and after the atomic action has run.
We can define a predicate that captures the basic behaviour
of such ``flow-controlled'' atomic action:
\begin{equation}
  in \in ls
  \land
  a
  \land
  ls'=(ls\setminus\setof{in})\cup\setof{out}
  \label{pred:flow-atomic-action}
\end{equation}
In short: the action when its $in$-label is in $ls$,
it that it performs the state-change specified by $a$,
and replaces the $in$-label by the $out$-label,
in the updated set $ls'$ of enabled labels.
If $in$ is not in $ls$,
or predicate $a$ is not satisfied by the current value of $s$,
then the semantic predicate reduces to $\false$.

We shall leave the issue of how specific labels
are associated with $in$ and $out$ of each atomic action until we
deal with language composites in Sec. \ref{sec:composing}.
All we shall say here is that we ensure that all actions have unique labels,
and it is always the case that $in \neq out$.
We also note that there will atomic actions that do not change the global
state, but that do change which labels are enabled, used to implement
some of the more complex control-flows.

The semantics of a running composite program,
as per the action systems approach used in \cite{DBLP:conf/icfem/WoodcockH02},
is to imagine all of the labelled atomic actions collected into one large
non-deterministic choice,
itself in a loop that runs until some distinguished stop-label appears
in the enabled label-set. The whole thing is initialised by enabling
at least one atomic action $in$-label.
We can imagine a predicate transformer $run$
that takes such a non-deterministic choice ($NDC$) and produces
a relational predicate that defines its overall running behaviour:
\begin{eqnarray}
   NDC
   &=&
   \bigvee_{j \in 0\dots n} (in_j \in ls
   \land
   a_j
   \land
   ls'=(ls\setminus\setof{in_j})\cup\setof{out_j})
\\ run(NDC)
   &defs&
   ls' = \setof{in_0} ; \textbf{ while } out_n \notin ls \textbf{ do } NDC
\end{eqnarray}
Here we assume $in_0$ and $out_n$ are the initial and final labels of the whole
program.
In effect, the meaning of a shared-variable concurrent program
is all the interleavings of atomic actions that are consistent
with flow-of-control restrictions, with each interleaving
being a series of atomic actions sequentially composed \emph{semantically},
using $;$ as defined in Eqn. \ref{eqn:semantic-;}.

Given that we will be sequentially composing a lot of predicates
like \ref{pred:flow-atomic-action},
we shall introduce a shorthand notation that we refer to as
a ``basic action'', in which generalie from $in$ and $out$
to sets of labels called $E$ (enablers) and $N$ (new):
\RLEQNS{
   \lhsA &\defs& \rhsA & \elabel{\lblA}
}
The plan is to then produce some laws governing
the semantic sequential compositions of basic actions
($A(E_1|a_1|N_1);A(E_2|a_1|N_2) = ?$),
but we rquickly doscover that
in general the outcome cannot be expressed as a single instance
of the form $A(E|a|N)$.
Consider $A(l_1|a|l_2);A(l_2|b|l_3)$,
in a starting state where both $l_1$ and $l_2$ are in $ls$.
The overall result is a combined action that needs $l_1$
to start, and adds in $l_3$ at the end,
but also removes both $l_1$ and $l_2$.
So, in order to effectively calculate with the theory
(see Sec. \ref{sec:calc}), we need to generalise the basic action idea
to an eXtended basic action,
where we explicitly identify the labels that we remove ($R$):
\RLEQNS{
   X(E|a|R|A)
   &\defs&
   E \subseteq ls \land a \land ls' = (ls \sminus R) \cup A
   & \elabel{X-def}
}
Clearly $A(E|a|N) = X(E|a|E|N)$.
We can now prove the following composition law:
\RLEQNS{
   \lefteqn{X(E_1|a|R_1|A_1) ; X(E_2|b|R_2|A_2)} &&& \elabel{X-then-X}
\\ &=& E_2 \cap (R_1\sminus A_1) = \emptyset
\\ && {} \land
   X(E_1 \cup (E_2\sminus A_1)
       ~\mid~ a\seq_s b
       ~\mid~ R_1 \cup R_2
       ~\mid~ (A_1 \sminus R_2) \cup  A_2)
}
Here $a \seq_s b$ is semantic sequential composition restricted
to predicates that only refer to $s$ and $s'$, and not $ls$  or $ls'$.
The condition $E_2 \cap (R_1\sminus A_1) = \emptyset$
characterises all those cases were the second $X$ is enabled
immediately after the first $X$ terminates
(i.e., without any outside interference).
This brings us to a very important aspect of how these predicates are to be
interpreted. The semantic sequential composition of two basic actions
captures the occurence of both actions in sequence without any intervening
interference. This means that the first action once enabled,
must be able to enable the second one without relying on some external agent.
The expression $E_2 \cap (R_1\sminus A_1)$ is all of the labels in $E_2$
that are removed ($R_1$) by the first action, but are not added back in ($A_1$).
If this not empty then some of the labels from $E_2$ will not be present,
and so the second action has been disabled by the first.
So the whole predicate reduces to $\false$,
indicating that it is not possible to observe those two actions in sequence,
unless some other execution thread manages to add in the missing $E_2$ labels
inbetween.

The reason for developing the $A(\dots)$ and $X(\dots)$ shorthands
will become clear once we have discussed the healthiness conditions
required to make the theory work (Sec. \ref{sec:health}), 
and the kinds of calculations that will then ensue
when we consider concrete examples (Sec. \ref{sec:calc}).
