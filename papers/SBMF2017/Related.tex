\section{Related Work}\label{sec:related}

The UTP views paper by van Staden\cite{DBLP:conf/utp/Staden14}
starts algebraically, looking at Kleene alebras over languages.
Languages here are sets of strings over an alphabet $A$.
He then takes $A =\Sigma\times\Sigma$,
which in effect encodes the Brookes model\cite{DBLP:journals/iandc/Brookes96}
(see also Park\cite{conf/ac/Park79}).

This paper may provide a way for me to prove all the laws hold for
my formulation of UTCP.


Needed: a description of prior compositional semantics
for shared-variable concurrency,
including Brookes, de Boer, something from actions systems (Mosses?)
and Petri-Nets (?).
Want to explore the role of stuttering and mumbling,
at least in the Brookes and de Boer work.

Glynn Winskel's fibrations as a unification principle?


\subsection{Concurrency Semantics}

Key work was done on concurrent semantics in the 80s and 90s,
with a strong focus on fully abstract denotational
semantics.
Notable work form this period includes that by
Stephen Brookes\cite{DBLP:journals/iandc/Brookes96}
and Frank de Boer and colleagues\cite{DBLP:conf/concur/BoerKPR91}.
Both looked at denotations based on the notion of sets of transition traces,
these being sequences of pairs of before-after states.
In order to get compositionality the traces of any program fragment
had to have arbitrary ``stuttering'' and ``mumbling'' state-pairs
added to capture the notion of outside interference.
Full abstraction meant that the semantics had to identify
programs like $skip\cseq skip$ with $skip$,
while distinguishing between $x:=2$ and $x:=1\cseq x:=x+1$.
This latter aspect required the language to be augmented with
an atomic wrapper construct.
A common feature of both was the very close linkage of the denotational
semantics to the operational one.
The work of Brookes\cite{DBLP:journals/iandc/Brookes96}
focussed on imperative languages with fair schedulers,
while that of de Boer et.al\cite{DBLP:conf/concur/BoerKPR91}
looked at a general framework (``failures of failures'')
that covered not just imperative programs
but also constraint solving systems
and \emph{asynchronous} versions of process algebras.


As already stated earlier,
the key inspiration and starting point for the work presented here
was the UTPP paper\cite{DBLP:conf/icfem/WoodcockH02}.
This combined guarded commands\cite{1976:book:dijkstra}
with the idea of action systems\cite{PODC::BackK1983},
interpreted in UTP as non-deterministic choice
over guarded atomic actions,
where disabled actions behave like the unit for that choice.
This basic lattice-theoretic architecture for the UTPP semantics
forms the foundation for the UTCP semantics presented here.

\subsection{Action Systems}

Here we give a short introduction to semantics of action systems
in UTP, following the material on unifying theories
of parallel programming in \cite{DBLP:conf/icfem/WoodcockH02}.

We assume the UTP theory of designs ($P \design Q$), with definitions
of assignment ($x:=e$),
sequential composition ($P;Q$),
iteration ($c*P$), and assumptions ($c^\top$).

\begin{quote}
``An action system has a state, an initialisation, and a number of actions.
First the initialisation is performed;
then repeatedly, a command whose guard is true is selected and executed.
If several guards are true,
then one of them is selected and its corresponding command is executed''\cite{DBLP:conf/icfem/WoodcockH02}
\end{quote}



We define a guarded action, where $g$ is a condition,
and $P$ a design, as:
\RLEQNS{
     g \rightarrow P &\defs& g^\top ; P & \elabel{def:guard}
}

An action system program
consists of a state declaration ($v$),
initial state ($V_0$)
a continue condition ($cont$),
and a set of guarded actions $\setof{g_i \rightarrow P_i}$.
It initialises the state,
and then, while the continue condition remains true,
repeatedly evaluates the disjunction of all the guarded actions.
\begin{eqnarray*}
(v,V_0,cont,\setof{g_i \rightarrow P_i})
   &\defs&
   \textbf{var }v := V_0 ~;
\\&& cont * \bigvee_i (g_i \rightarrow P_i)
\end{eqnarray*}


\subsection{TEMP Related Work: S. D. Brookes}

Some touchstone papers in this area are the following three by
Stephen Brookes:
\cite{DBLP:conf/concur/Brookes84},
\cite{DBLP:conf/lics/Brooks93}, and
\cite{DBLP:journals/iandc/Brookes96}.
We shall summarise the denotational results here,
using our notational style.

\subsubsection{Syntax}

\def\lseq{;}
\def\lcond#1{\cond{#1}}
\def\wdo{\circledast}
\def\bskip{skip}
\def\bseq{\lseq}
\def\bpar{\parallel}
\def\bcond#1{\lcond{#1}}
\def\bdo{\wdo}
\def\bawait{\mathbin{!}}
His programming language has one extra (command) construct --- the await command
which we shall consider as a form of guard

\RLEQNS{
   e,b &\in&  Exp, BoolExp
\\ c \in Cmd
   &::=& \bskip       & \textbf{skip}
\\ & | & i := e       & I:=E
\\ & | & c \bseq c    & C_1;C_2
\\ & | & c \bpar c    & C_1 \parallel C_2
\\ & | & c \bcond b c & \textbf{if }B\textbf{ then }C_1\textbf{ else }C_2
\\ & | & b \bdo c     & \textbf{while }B\textbf{ do }C
\\ & | & b \bawait c  & \textbf{await }B\textbf{ then }C
}
In $b \bawait c$ we require $c$ to be limited to a finite sequence
of assignment and skip.
The \textbf{await} construct is needed to ensure full abstraction.


\subsubsection{Semantics}

It is worth noting the following two examples
that give an idea of what an appropriate semantics should be able to do:
\begin{enumerate}
  \item
     We should be able to deduce that $\bskip = \bskip \bseq \bskip$.
     Some semantics distinguish these, which is too concrete.
  \item
     In isolation we see  that
     \[
        x:=2 = (x:=1 \bseq x:=x+1)
     \]
     However, in context, these behave differently:
     \[
        x:=2 \bpar x:=2
        \neq
        (x:=1 \bseq x:=x+1) \bpar x:=2
     \]
     The latter can result in $x=3$.
\end{enumerate}


\def\E{\mathcal{E}}
\def\B{\mathcal{B}}
\def\M{\mathcal{M}}
\def\bpre{\sqsubseteq_{\M}}
\def\bequiv{\equiv_{\M}}
\def\spre{\leq_{\M}}
\def\sequiv{=_{\M}}
\def\trans{\rightarrow}
\def\rtrans{\trans^{*}}

Initially semantic functions are defined w.r.t. the operational semantics.
\RLEQNS{
   \E\sem{e} &=& \setof{(s,k) | (e,s) \rtrans k }
\\ \B\sem{b} &=& \setof{(s,t) | (e,s) \rtrans t }
}
The partial correctness behaviour function has signature:
\[
\M : Cmd \fun \power(S \times S)
\]
and clearly maps any command to a relation on before and after states,
with the following pre-order:
\[
 C \bpre C'
 \defs
 \forall s @ \M\sem{C}s \subseteq \M \sem{C'}
\]
and then define equivalence ($\bequiv$) from this.

He observes that $\bequiv$ is not substitutive:
\RLEQNS{
    x:=1;x:=x+1             &  \bequiv  & x:=2
\\ (x:=1;x:=x+1) \bpar x:=2 &\not\bequiv& x:=2 \bpar x:=2
}
and then defines a substitutive preorder (and hence equivalence $\sequiv$):
\[
  C \spre C'
  \defs
  \forall P[\cdot] @ P[C] \bpre P[C']
\]

We show only one operational rule --- that for \textbf{await},
that shows its atomic behaviour:
\[
\inferrule%
  {(b,s) \rtrans \\ (c,s) \rtrans (c',s')term}
  {(b \bawait c,s) \trans (\bskip,s')}
\]

\paragraph{Transition Traces}~

\def\T{\mathcal{T}}
\def\ClsTrc{\power^\dagger(\Sigma^+)}
\def\tclose#1{\setof{{#1}}^{\dagger}}

Imagine that a command transforms $s$ to $s'$,
in a run where it is interrupted $k$ times:
\[
  (s_0,s'_0)(s_1,s'_1)(s_2,s'_2)\dots(s_k,s'_k)
\]
Here $s=s_0$ and $s'=s'_k$,
and the change from $s'_{i-1}$ to $s_i$
denotes the effect of the $i$th interruption.
If $s_i = s'_{i-1}$ then that interruption was non-interfering.

We can define the traces for a command from the operational semantics as
\RLEQNS{
  \T\sem{C} &=& \setof{(s_0,s'_0)(s_1,s'_1)\dots(s_k,s'_k)}
\\ && \textbf{where }C_0 = C
\\ && (C_i,s_i) \rtrans (C_{i+1},s'_{i+1}), \quad i \in 0 \dots k-1
\\ && (C_k,s_k) \rtrans (C',s'_k)term
\\
\\ \M\sem{C} &=& \setof{(s,s')|(s,s') \in \T\sem{C})}
}
Stuttering:
\[
\alpha\beta \in \T\sem C \implies \alpha(s,s)\beta \in \T\sem C
\]
Mumbling:
\[
\alpha(s,s')(s',s'')\beta \in \T\sem C \implies \alpha(s,s'')\beta \in \T\sem C
\]
Given a set $T$ of transition traces, then $T^\dagger$
is the closure under stuttering and mumbling. Note that $\T\sem C$ is closed.
The set of closed sets of non-empty traces under inclusion
($\ClsTrc$)
is a complete
lattice, bottom being the emptyset, and unions giving l.u.b.s.

Concatenation:
\[
T_1 ; T_2 = \tclose{\alpha\beta | \alpha\in T_1, \beta\in T_2}
\]
Interleaving:
\RLEQNS{
   \alpha\bpar\epsilon &=\setof\alpha=& \epsilon\bpar\alpha
\\ x:\alpha \bpar y:\beta
   &=&
   \setof{x:\gamma | \gamma \in \alpha\bpar y:\beta}
   \cup
   \setof{y:\gamma | \gamma \in x:\alpha\bpar \beta}
\\ T_1\bpar T_2
   &=&
   \bigcup\tclose{\alpha\bpar\beta | \alpha\in T_1, \beta\in T_2}
}
We define $\T\sem B$ for boolean expression $b$ as
$\setof{(s,s) | (b,s) \rtrans true }$

\newpage
\subsubsection{Denotational Semantics}

\def\TT#1{\T\sem{{#1}}}
\def\override{\oplus}
\def\tpre{\sqsubseteq_{\T}}
\def\tequiv{\equiv_{\T}}
\def\wtrans{\trans^\omega}

Denotations (partial correctness):
\RLEQNS{
   \T &:& Cmp \fun \ClsTrc
\\ \TT{skip} &=& \tclose{(s,s)|s \in S}
\\ \TT{i:=e} &=& \tclose{(s,s\override\maplet i k)|(s,k)\in\E\sem{e}}
\\ \TT{c_1\bseq c_2} &=& \TT{c_1} ; \TT{c_2}
\\ \TT{c_1\bpar c_2} &=& \TT{c_1} \parallel \TT{c_2}
\\ \TT{c_1 \bcond b c_2 } &=& \TT{b} ; \TT{c_1} \cup \TT{\lnot b} ; \TT{c_2}
\\ \TT{b \bawait c} &=& \tclose{(s,s') \in \TT{c} | (s,s) \in \TT{b}}
\\ \TT{b \bdo c} &=& (\TT{b} ; \TT{c})^* ; \TT{\lnot b}
\\ &=& \mu W @ (\TT{b};\TT{c};W \cup \TT{\lnot b})
}

Denotational Pre-order (with induced equivalence $\tequiv$):
\RLEQNS{
   \TT{c}s &=& \setof{s'\alpha | (s,s') \in \TT{c}}
\\ c \tpre c' &=& \forall s @ \TT{c}s \subseteq \TT{c'}s
}

If we allow infinite \emph{fair} traces, then we extend from $\Sigma^+$
to $\Sigma^\infty = \Sigma+ \cup \Sigma^\omega$, adjust closure
accordingly, and only the definition for $\bdo$ changes:
\RLEQNS{
   \TT{b \bdo c}
   &=&
   (\TT{b} ; \TT{c})^* ; \TT{\lnot b}
   \cup
   (\TT{b} ; \TT{c})^\omega
}
This is not the greatest fixed point, whose use results in unfair traces being
included,
and identifies all non-terminating commands,
as gives a theory of total correctness:
\RLEQNS{
\\ \TT{b \bdo c} &=& \nu W @ (\TT{b};\TT{c};W \cup \TT{\lnot b})
\\ \M\sem{c}     &=& \setof{(s,s') | (c,s) \rtrans (c',s')term} \cup {}
\\                && \setof{(s,s') | (c,s) \wtrans \land s' \in S_\bot}
}
