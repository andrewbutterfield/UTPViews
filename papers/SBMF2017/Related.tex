\subsection{Related Work}\label{ssec:related}

Key work was done on concurrent semantics in the 80s and 90s,
with a strong focus on fully abstract denotational
semantics.
Notable work form this period includes that by
Stephen Brookes\cite{DBLP:journals/iandc/Brookes96}
and Frank de Boer and colleagues\cite{DBLP:conf/concur/BoerKPR91}.
Both looked at denotations based on the notion of sets of transition traces,
these being sequences of pairs of before-after states.
In order to get compositionality the traces of any program fragment
had to have arbitrary ``stuttering'' and ``mumbling'' state-pairs
added to capture the notion of outside interference.
Full abstraction meant that the semantics had to identify
programs like $skip\cseq skip$ with $skip$,
while distinguishing between $x:=2$ and $x:=1\cseq x:=x+1$.

The first UTP theory in this area was presented in 
the UTPP paper\cite{DBLP:conf/icfem/WoodcockH02}.
This combined guarded commands\cite{1976:book:dijkstra}
with the idea of action systems\cite{PODC::BackK1983},
interpreted in UTP as non-deterministic choice
over guarded atomic actions,
where disabled actions behave like the unit for that choice.
This basic lattice-theoretic architecture for the UTPP semantics
forms the foundation and inspiration for the UTCP semantics presented here.

More recently, also inspired by \cite{conf/popl/Dinsdale-YoungBGPY13},
the  ``UTP Views'' paper by van Staden\cite{DBLP:conf/utp/Staden14},
starts algebraically, looking at Kleene alebras over languages.
Languages here are sets of strings over an alphabet $A$.
He then takes $A =\Sigma\times\Sigma$,
which in effect encodes the Brookes model\cite{DBLP:journals/iandc/Brookes96}.
His semantics fits with the usual UTP approach to concurrency,
in that it is based on traces as sequences of some notion of event.
The semantics presented in this paper is based on direct relations
between before- and after-program states, without any notion of traces.
