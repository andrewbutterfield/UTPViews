\section{Root}\label{ha:Root}

We do a quick run-down of the Commands\cite{conf/popl/Dinsdale-YoungBGPY13}.

\subsection{Syntax}

\def\Atm#1{\langle#1\rangle}
\def\rr#1{r{\scriptstyle{#1}}}
\def\RR#1{R{\scriptstyle{#1}}}

\begin{eqnarray*}
   a &\in& \Atom
\\ C &::=&
 \Atm a \mid \cskip \mid C \cseq C \mid C+C \mid C \parallel C \mid C^*
\\ g &:& Gen
\\ \ell &:& Lbl
\\ G &::=&  g \mid G_{:} \mid G_1 \mid G_2
\\ L &::=& \ell_G
\end{eqnarray*}


\subsection{Domains}


\subsubsection{Rooted Paths}

This Root file is a re-work of the Views semantics
replacing label generators by ``rooted'' label-paths.

We start by defining three basic ways to transform a rooted path:
``done'' ($\bullet$);
``split-one'' ($1$);
and ``split-two'' ($2$).
No further changes can be made to one that is ``done''.
\RLEQNS{
   S &\defs& \setof{1,2}
}

We now define a rooted path as an expression of the form
of the variable $r$ followed by zero or more $S$ transforms,
possibly terminated by the ``done'' marker.
\RLEQNS{
   \sigma,\varsigma &\defs& S^*
\\ R &::=& r\sigma | r\sigma\bullet
}
These have to be expressions as we shall want to substitute for $r$
in them by ``done-free'' rooted paths.

\subsection{Shorthands}

We support a shorthand (that views a set as its own collection
of corresponding $n$-ary characteristic functions)
that denotes $x \in S$ by $S(x)$ and $ X \subseteq S$ by $S(X)$,
and omits $\{$ and $\}$ from around enumerations when context makes
it clear a set is expected

\begin{eqnarray*}
   ls(\ell) &\defs& \ell \in ls
\\ ls(L) &\defs& L \subseteq ls
\\ ls(\B\ell) &\defs& \ell \notin ls
\\ ls(\B L) &\defs& L \cap ls = \emptyset
\end{eqnarray*}


\paragraph{Set Swapping}

We update a set by removing some elements
and replacing them with new ones:
\RLEQNS{
   A \ominus (B,C) &\defs& (A \setminus B) \cup C
}

\subsection{Alphabet}

\begin{eqnarray*}
   s, s' &:& \mathcal S
\\ ls, ls' &:& \mathcal P (R)
\\ r &:& R
\end{eqnarray*}

Predicate variables $a$, $b$ and $c$
will refer to atomic state-changes,
and so only have alphabet $\setof{s,s'}$.
An expression $g$ is `ground' if its free variables are contained in $\setof r$.

\subsection{``Standard'' UTP Constructs}

\begin{eqnarray*}
   P \cond c Q
   &\defs&
   c \land P \lor \lnot c \land Q
\\ P ; Q
   &\defs&
   \exists s_m,ls_m \bullet P[s_m,ls_m/s',ls'] \land Q[s_m,ls_m/s,ls]
\\ c * P
   &=&
   P ; c * P \cond c \Skip
\\ \Skip &=& ls'=ls \land s'=s
\\ \alpha \Skip &=& \setof{ls,ls',s,s'}
\end{eqnarray*}

We note that ground expressions can ``ignore'' sequential composition
in a certain sense:
\begin{eqnarray*}
   (g \land P) ; Q \quad = &     g \land ( P ; Q)      & = \quad P ; (g \land Q)
\\                         &           =
\\                         & (g \land P) ; (g \land Q)
\end{eqnarray*}

\subsection{WwW Basic Shorthands}

Atomic actions have a basic behaviour that is described by
\[
ls(E) \land s' \in \sem a s \land ls' = (ls\setminus E) \cup N
\]
where $E$ is the set of necessary enabling labels,
$a$ is a relation over shared state,
and $N$ is the set of new labels deposited upon completion.
We shall abstract the above as
\[
A(E|a|N)
\]
There is a slightly more general form that removes labels
that may differ from those doing the enabling:
\[
ls(E) \land s' \in \sem a s \land ls' = (ls\setminus R) \cup A
\]
where $R$ and $A$ are sets of labels respectively removed, then added
on the action is complete.
We abstract this as
\[
  X(E|a|R|A)
\]
and note that
\[
  A(E|a|N) = X(E|a|E|N).
\]


\subsubsection{Generic Atomic Behaviour}

\begin{eqnarray*}
   X(E|ss'|R|A)
   &\defs&
   ls(E) \land [ss'] \land ls'=(ls\setminus R)\cup A
\end{eqnarray*}

We do want the following simplification:
\begin{eqnarray*}
  X(E|a|E|N) &=& A(E|a|N)
\end{eqnarray*}

\begin{eqnarray*}
   A(E|ss'|N)
   &\defs&
   X(E|ss'|E|N)
\end{eqnarray*}


\subsubsection{Coding Label-Set Invariants}

We have a key invariant as part of the healthiness
condition associated with every semantic predicate,
namely that the labels $r$ and $\rr\bullet$ never occur in  $ls$ at
the same time:
\[
 ( r \in ls \implies \rr\bullet \notin ls )
 \land
 ( \rr\bullet \in ls \implies r \notin ls )
\]
This is the Label Exclusivity invariant.

Given the way we shall use substitution of $r$ by
other rooted paths, for sub-components,
we shall see that we will get a number of instances of this.
We adopt a shorthand notation,
so that the above invariant is simply
\[
  [r|\rr\bullet]
\]
So we define the following general shorthand:
\RLEQNS{
   ~[L_1|L_2|\dots|L_n]
   &\defs&
   \forall_{i,j \in 1\dots n}
    @
    i \neq j \implies
     ( L_i \cap ls \neq \emptyset \implies L_j \cap ls = \emptyset )
\\ \multicolumn{3}{c}{\elabel{short-lbl-exclusive}}
}
What needs to be kept in mind regarding this shorthand notation
is that $ls$ is mentioned under the hood,
and it is really all about what can be present in the global label-set
at any instant in time.


Now, we need to define invariant satisfaction.
Our invariant applies to $A$ and $X$ atomic actions:
\RLEQNS{
   A(E|a|N) \textbf{ sat } I
   &\defs&
   E \textbf{ lsat } I \land N \textbf{ lsat } I
\\ X(E|a|R|A)  \textbf{ sat } I
   &\defs&
   E \textbf{ lsat } I \land A \textbf{ lsat } I
}

Now we have to define \textbf{lsat}:
\RLEQNS{
   \textbf{lsat}_S [L_1|\dots|L_n]
   &\defs&
   \#(filter ~\textbf{lsat'}_S \seqof{L_1,\ldots,L_n}) < 2
\\ \textbf{lsat'}_S \setof{\ell_1,\dots,\ell_n}
  &\defs&
  \exists i @ \textbf{lsat''}_S \ell_i
\\ \textbf{lsat''}_S \ell &\defs& \ell \in S
}

\subsubsection{Invariant Trimming}

We can use the invariant to trim removal sets,
given an enabling label or label-set.
\RLEQNS{
   I \textbf{ invTrims } X(E|a|E,L|A)
   &\defs&
   \lnot(\setof{E,L} \textbf{ lsat } I)
}

\subsubsection{Wheels within Wheels}\label{hc:WwW}

The wheels-within-wheels healthiness condition
insists that $r$ and $\rr\bullet$ are never simultaneously in
the label-set $ls$,
and that our semantic predicates are closed under mumbling.
In fact we can strengthent the invariant to say for any rooted path $R$
that $R$ and $\RR\bullet$ are never in $ls$ together.
\RLEQNS{
   \W(C)
   &\defs&
   [R|\RR\bullet] \land \left(~\bigvee_{i\in 0\dots} C^i~\right)
\\ ii &\defs& s'=s
}
In fact it makes sense to define $\W(C)$ as $\bigvee C^i$
and have the following invariant preservation theorem, for every command $C$:
\begin{equation*}
   [R|\RR\bullet] \implies \W(C) \implies [R|\RR\bullet]'
\end{equation*}
Or we have rolled in and the proof obligation is
\begin{equation*}
   \W(C) \implies [R|\RR\bullet]'
\end{equation*}


Unrolling $\W(C)$, using $I$ for $[R|\RR\bullet]$,
and noting that
$
\bigvee_{i \in \Nat} C^i
= \Skip \lor (C\seq\bigvee_{i \in \Nat} C^i)
$.
\RLEQNS{
   \W(C) &=& I \land \bigvee C^i
\\ &=& I \land (\Skip \lor C\seq\bigvee C^i)
\\ &=& I \land (\Skip \lor C\seq((\Skip \lor C\seq\bigvee C^i)))
\\ &=& I \land (\Skip \lor C \lor C^2\seq\bigvee C^i)
\\ &=& I \land (\Skip \lor C \lor C^2\seq((\Skip \lor C\seq\bigvee C^i)))
\\ &=& I \land (\Skip \lor C \lor C^2 \lor C^3\seq\bigvee C^i)
\\ &\vdots&
\\ &=& I \land (\Skip \lor C \lor \dots C^{k-1} \lor C^k\seq\bigvee C^i)
\\ &=& I \land (~(\bigvee_{i < k} C^i) \lor (\bigvee_{i \geq k} C^i)~)
}
We assume $\seq$ binds tighter than $\lor$.
We can also move conjoined ground predicates around at will:
\begin{equation}
  g \land \bigvee C^i
= \bigvee (g \land C^i)
= \bigvee (g \land C)^i
= g \land \bigvee (g \land C)^i
\end{equation}
With disjunction of predicates we get all possible interleavings:
\begin{eqnarray*}
      W(P \lor Q)
  &=& \bigvee (P \lor Q)^i
\\&=& \Skip \lor P \lor Q \lor {}
\\& & P^2 \lor PQ \lor QP \lor Q^2 \lor {}
\\& & P^3 \lor P^2Q \lor PQP \lor PQ^2 \lor {}
\\& & QP^2 \lor QPQ \lor Q^2P \lor Q^3 \lor \dots
\end{eqnarray*}
If one of the disjuncts is a use of $\W$:
\begin{eqnarray*}
      W(P \lor \W(Q))
\\&=& \bigvee (P \lor \W(Q))^i
\\&=& \bigvee_i (P \lor \bigvee_j Q^j)^i
\end{eqnarray*}
This is all possible interleavings of $P$ and $Q^j$,
each of which can be matched to a interleaving or $P$ and $Q$,
so we have:
\begin{equation}
   \W(P \lor \W(Q)) = \W(P \lor Q)
\end{equation}
Conjunctions with ground expressions are not a problem:
\begin{eqnarray*}
   \W(P \lor g \land \W(Q)) &=& \W(P \lor \W(g \land Q))
\\ &=& \W(P \lor g \land Q)
\\ g_1 \land \W(P \lor g_2 \land \W(Q))
   &=& g_1 \land \W(P \lor g_2 \land Q)
\\ &=& \W(g_1 \land P \lor g_1 \land g_2 \land Q)
\end{eqnarray*}

\subsection{WwW Semantic Definitions}

The definitions, using the new shorthands:
\RLEQNS{
   \W(C) &\defs& [r|\rr\bullet]
                 \land
                 \left(\bigvee_{i\in 0\dots} C^i\right)
\\ ii &\defs& s'=s
\\
\\ \Atm a &\defs&\W(A(r|a|\rr\bullet))
\\
\\ \cskip
   &\defs&
   \Atm{ii}
}
The following are all under review (they lack sufficient invariants).
\RLEQNS{
   C \cseq D
   &\defs&
   \W(~    A(r|ii|\rr1)
      \lor C[\rr1/r]
      \lor A(\rr{1\bullet}|ii|\rr2)
      \lor D[\rr2/r]
      \lor A(\rr{2\bullet}|ii|\rr\bullet) ~)
\\
\\ C + D
   &\defs&
   \W(\quad {}\phlor A(r|ii|\rr1) \lor A(r|ii|\rr2)
\\ && \qquad {} \lor
   C[\rr1/r] \lor D[\rr2/r]
\\ && \qquad {} \lor A(\rr{1\bullet}|ii|\rr\bullet) \lor A(\rr{2\bullet}|ii|\rr\bullet) ~)
\\
\\ C \parallel D
   &\defs&
   \W(\quad\phlor A(r|ii|\rr1,\rr2)
\\ && \qquad {}\lor
   C[\rr1/r]
   \lor D[\rr2/r]
\\ && \qquad {}\lor
   A(\rr{1\bullet},\rr{2\bullet}|ii|\rr\bullet)~)
\\
\\ C^*
   &\defs&
   \W(\quad  \phlor A(r|ii|\rr1) \lor A(\rr1|ii|\rr\bullet)
\\ && \qquad {}\lor C[\rr1/r]    \lor A(\rr{1\bullet}|ii|\rr1) ~)
}

\subsubsection{Coding Atomic Semantics}

\RLEQNS{
   \Atm a &\defs&\W(A(r|a|\rr\bullet))
}

\newpage
\subsubsection{Coding Skip}

\RLEQNS{
   ii &\defs& s'=s
\\ \cskip
   &\defs&
   \Atm{ii}
}


\subsubsection{Coding Sequential Composition}

\RLEQNS{
   C \cseq D
   &\defs& [r|\rr1|\rr{1\bullet}|\rr2|\rr{2\bullet}|\rr\bullet] \land {}
\\ && \W(\quad {}\phlor A(r|ii|\rr1)
\\ && \qquad {} \lor C[\rr1/r]
\\ && \qquad {} \lor A(\rr{1\bullet}|ii|\rr2)
\\ && \qquad {} \lor D[\rr2/r]
\\ && \qquad {} \lor A(\rr{2\bullet}|ii|\rr\bullet)~)
}

\subsubsection{Coding (Non-Det.) Choice}

\RLEQNS{
   C + D
   &\defs& [r|\rr1|\rr{1\bullet}|\rr2|\rr{2\bullet}|\rr\bullet] \land {}
\\&& \W(\quad {}\phlor A(r|ii|\rr1) \lor A(r|ii|\rr2)
\\ && \qquad {} \lor
   C[\rr1/r] \lor D[\rr2/r]
\\ && \qquad {} \lor A(\rr{1\bullet}|ii|\rr\bullet) \lor A(\rr{2\bullet}|ii|\rr\bullet) ~)
}

\subsubsection{Coding Parallel Composition}

\RLEQNS{
   C \parallel D
   &\defs& ~
   [r|\rr1,\rr2,\rr{1\bullet},\rr{2\bullet}|\rr\bullet] \land
   [\rr1|\rr{1\bullet}] \land
   [\rr2|\rr{2\bullet}] \land {}
\\&& \W(\quad\phlor A(r|ii|\rr1,\rr2)
\\ && \qquad {}\lor
   C[\rr1/r]
   \lor D[\rr2/r]
\\ && \qquad {}\lor
   A(\rr{1\bullet},\rr{2\bullet}|ii|\rr\bullet)~)
}

\subsubsection{Coding Iteration}

\RLEQNS{
   C^*
   &\defs& [r|\rr2|\rr1|\rr{1\bullet}|\rr\bullet] \land {}
\\&& \W(\quad  \phlor A(r|ii|\rr2)
\\ && \qquad {}\lor A(\rr2|ii|\rr1)
               \lor C[\rr1/r]
               \lor A(\rr{1\bullet}|ii|\rr2)
\\ && \qquad {}\lor A(\rr2|ii|\rr\bullet) ~)
}


\subsection{Reductions for WWW}\label{hb:WWW-reduce}

\subsubsection{Recognisers for WWW}\label{hc:v-recog}

\RLEQNS{
   ls'=ls\ominus(S_1,S_2)
   &\rightsquigarrow&
   \seqof{S_1,S_2}
   & \ecite{sswap-$;$-prop.}
}

\paragraph{$A$ then $A$}

\RLEQNS{
  && A(E_1|a|N_1) \seq A(E_2|b|N_2)
\EQ{proof in Views.tex }
\\&& X(E_1 \cup (E_2\setminus N_1)
       |a\seq b
       |E_1 \cup E_2
       |N_1 \setminus E_2 \cup  N_2)
       \land (E_2\setminus N_1) \cap E_1 = \emptyset
}

\paragraph{$X$ then $A$}

\RLEQNS{
  && X(E_1|a|R_1|A_1) \seq A(E_2|b|N_2)
\EQ{proof in Views.tex }
\\&& X(E_1 \cup (E_2\setminus A_1)
       |a\seq b
       |R_1 \cup E_2
       |A_1 \setminus E_2 \cup  N_2)
       \land (E_2\setminus A_1) \cap R_1 = \emptyset
}

\paragraph{$A$ then $X$}

\RLEQNS{
  && A(E_1|a|N_1) \seq X(E_2|b|R_2|A_2)
\EQ{proof in Views.tex}
\\&& X(E_1 \cup (E_2 \setminus N_1)
      | a\seq b
      | E_1 \cup R_2
      | N_1 \setminus R_2 \cup A_2)
     \land
     (E_2 \setminus N_1) \cap E_1 = \emptyset
}

\paragraph{$X$ then $X$}

\RLEQNS{
  && X(E_1|a|R_1|A_1) \seq X(E_2|b|R_2|A_2)
\EQ{proof in Views.tex }
\\&& X(E_1 \cup (E_2\setminus A_1)
       |a\seq b
       |R_1 \cup R_2
       |A_1 \setminus R_2 \cup  A_2)
       \land (E_2\setminus A_1) \cap R_1 = \emptyset
}

\paragraph{$I$ collapses $X$ to $A$}
\RLEQNS{
   \lnot(\setof{E,L} \textbf{ lsat } I)
   &\implies&  X(E|a|E,L|N) = A(E|a|N)
}
Given $I$ and $X(E|a|R|A)$, we proceed as follows:
\begin{enumerate}
  \item Let $D = R \setminus E$
  \item Compute
    $D' =
       \setof{  d | d \in D,
                    \lnot (E\cup\setof d \textbf{ lsat } I)}
    $
  \item
    If $D = D'$ then return $A(E|a|A)$
  \item
    Else, return $X(E|a|E\cup(D \setminus D')|A)$.
\end{enumerate}

\paragraph{General Stuff}~

If $a$ and $b$ have alphabet $\setof{s,s'}$,
then
$(a \seq b)[G,L,M/g,in,out) = a\seq b$.

\subsection{Loop Unrolling for Views}\label{hb:WWW-unroll}

Iteration  satisfies the loop-unrolling law:
\[
  c * P  \quad=\quad (P ; c * P ) \cond c \Skip
\]
But we also support several styles and degrees of unrolling:
\begin{eqnarray*}
   c*P
   &=_0& (P\seq c*P) \cond c \Skip
\\ &=_1& \lnot c \land \Skip
         \lor
         c \land P ; c * P
\\ &=_2& \lnot c \land \Skip
         \lor
         c \land P ; \lnot c \land \Skip
         \lor
         c \land P ; c \land P ; c * P
\\ &=_3& \lnot c \land \Skip
         \lor
         c \land P ; \lnot c \land \Skip
         \lor
         c \land P ; c \land P ; \lnot c \land \Skip
         \lor
         c \land P ; c \land P ; c \land P ; c *P
\\ && \vdots
\\ &=_n& \left(
           \bigvee_{i=0}^{n-1}  (c \land P)^i \seq \lnot c \land \Skip
         \right)
         \lor
         (c \land P)^n \seq c *P
\end{eqnarray*}


\subsection{Calculation Results}

Given the idiom $\true * (\Skip \lor Q)$
we look for pre-computed $Q$-cores,
noting that our final semantics will have the form
\[
  I \land ( \Skip \lor Q \lor Q^2 \lor \dots Q^k)
\]
where $I$ is the label-set invariant
and $Q^i = \false$ for all $i > k$.

When we change from:
\RLEQNS{
   \W(C) &\defs& \true * (\Skip \lor C)
\\ \Atm a &\defs&\W(A(in|a|out)) \land [in|out]
}
\noindent
to:
\RLEQNS{
   \W(C) &\defs& \true * C
\\ \Atm a &\defs&\W(\Skip \lor A(in|a|out)) \land [in|out]
}
\noindent
the calculation outcomes
for \texttt{athenb}, \texttt{aorb}, \texttt{awithb} and \texttt{itera} are unchanged,
and consequently so are all the other calculations.
