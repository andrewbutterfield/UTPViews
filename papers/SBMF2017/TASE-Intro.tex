\HDRa{Introduction}\label{ha:intro}


Programming-like notations have been used
to describe business processes and workflows for many years
\cite{Campbell98,Combi09,Yang12}.
There is considerable interest at present in healthcare systems
in so-called clinical pathways,
that describe processes for managing patient care.
These too can be described using general business process notations
\cite{Ellis93,Eshuis03,vanderAalst06}.
Deploying process models in the medical domain in practise requires
flexible interpretations of those models
\cite{Ellis93,Ellis95,Grigori01,vanderAalst06,Pesic06,Combi09}
.

PML is such a language\cite{DBLP:journals/infsof/AtkinsonWN07},
developed originally for modelling software development processes,
but applicable to a much wider range of activities,
including clinical pathways\cite{LNR:CP-models:FHIES2013}.
It is designed to encourage a flexible approach to its interpretation
and deployment.
This is most obvious when one considers that the condition and iterative
constructs of the language have no condition predicates,
relying on the judgement of those enacting the process to determine
which conditional branch should be taken, or when a loop should terminate.


In this paper we present results obtained while developing
a range of formal semantics for PML using the
Unifying Theorems of Programming (UTP) framework\cite{Hoare-He98}
to support reasoning about flexible deployment.
We define a UTP semantics for  shared global state concurrency,
as a way to get a suitable semantics for strict and flexible PML,
inspired by the work of Woodcock and Hughes on unifying parallel programming (UTPP,\cite{DBLP:conf/icfem/WoodcockH02}).
We  present
both a formal semantics for a ``weak'' interpretation of PML,
as well as a unified theory of concurrent programs (UTCP)
that will provide a baseline theory for modelling more ``flexible''
and ``strong'' interpretations of PML.
Part of our contribution is extending the UTP methodology
to use non-homogenous relations that mix dynamic state-change
(observations with before- and after-values)
along with static context parameters
(observations whose value is unchanged during program execution).
We also develop a notion of label generators to allow us to describe
flow of control in a compositional manner.

The structure of this paper is as follows:
In \S\ref{ha:PML} we give a quick introduction to an abstract form of PML,
while in \S\ref{ha:UTP} we provide a quick overview of UTP.
We then move on to present the weak semantics for PML in \S\ref{ha:WeakPML},
the baseline UTCP semantics in \S\ref{ha:UTCP},
and then to relate the two in \S\ref{ha:unify},
where we also discuss future work.
We then describe related work (\S\ref{ha:related}) and conclude (\S\ref{ha:conc}).
