\section{Weak PML}\label{ha:WeakPML}

We give the semantics of the weakest interpretation of PML,
where we only consider the dataflow-like behaviour induced
by the ``requires''($?$) and ``provides'' ($!$) annotations on basic actions.
We shall assume a simple model of resources,
as uninterpreted entities that are distinguishable,
and where the only property of interest is whether or not any given
resource is present in the system.
What we want to observe is how the resources in the system
evolve over time, as well as keeping track (by name)
of which basic actions were executed.
We introduce three observations,
all of which can be made both before and after a PML run:
starting, termination ($ok,ok' : \Bool$);
resources in the system ($r,r' : \power Res$);
and basic actions executed ($h,h' : Name^*$).

Due to space limitations,
we do not give a detailed account of the healthiness conditions here,
but note that we have the usual Design conditions ($\HH1,\HH2,\HH3$)
as well as modified versions of some of the Reactive systems conditions\cite[Chp 8]{Hoare-He98}
which we present here without comment:
\RLEQNSs{
   \PP1(P) &\defs& P \land h \leq h'
\\ \PP2(P) &\defs& \exists h_0  \bullet  P[h_0,h_0\cat(h'-h)/h,h']
}
In particular, we have a complete lattice with bottom $\true$
and top $\lnot ok$.

Our semantics has two parts, an action collection phase ($Coll$)
and a dynamic execution model ($doStep$, $RunW$).
The collection function works through a PML description
and collects all the basic actions,
recording them in a finite map.
This works because of the PML requirement
that every textual occurrence
of a basic action has a unique name or identifier.
\RLEQNSs{
   Coll &:& PML \fun (Name \pfun Action)
\\ Coll(N?rr!pr) &\defs& \maplet N {(N?rr!pr)}
\\ Coll(P \oplus Q) &\defs& Coll(P)\uplus Coll(Q),
    \quad \oplus \in \setof{\seq,\pcond,\parallel}
\\ Coll(\piter P) &\defs& Coll(P)
}
Here $\uplus$ denotes map disjoint union.
The meaning of a basic action is a design that runs when its
required resources are available,
here defined with the notion of a guarded action ($g\&A$):
\RLEQNSs{
   N?rr!pr &\defs&
   \true
   \design
   rr \subseteq r
     \pgrd  r' = r \cup pr \land h' = h \cat \seqof N
\\ g \pgrd A &\defs& A \cond g \lnot ok
}
Given a basic-actions map $\varpi$,
we can determine, w.r.t to the current resource set $r$,
if there is an enabled action whose execution will extend that
resource set ($canGo$):
\RLEQNSs{
   isEna(N?rr!pr) &\defs& rr \subseteq r
\\ isDone(N?rr!pr) &\defs& pr \subseteq r
\\ canGo(\varpi)
   &\defs&
   \exists N \bullet
      N \in \dom\varpi
\\ && \quad {}\land
      isEna(\varpi N)
      \land
      \lnot isDone(\varpi N)
}
We describe a single step of the system as one that
makes a non-deterministic choice out of all of the currently
enabled actions
and executes it:
\RLEQNSs{
   doStep(\varpi)
   &\defs&
   \exists N \bullet
     N \in \dom \varpi
     \land
     isEna(\varpi N)
     \land
     \varpi N
}
Here we exploit a standard part of the UTP methodology
 (``programs are predicates'')
that a action ($N?rr!pr$ or $\varpi N$)
is considered the same thing as its defining predicate.

We run a PML description ($RunW$) by repeatedly doing a step
so long as it is possible to make progress (provide more resources):
\RLEQNSs{
   RunW(\varpi)
   &\defs&
   canGo(\varpi) * doStep(\varpi)
}
Here $c * P$ is UTP notation for \textbf{while} $c$ \textbf{do} $P$.

The dynamic semantics of a PML description $P$ is therefore
given as
\RLEQNSs{
  P &=& RunW(Coll(P))
}
The following distributive law is an easy result,
for $\oplus$ ranging over $\seq,\pcond,\parallel$:
\RLEQNSs{
  RunW(Coll(P \oplus Q))
  &=&
  RunW(Coll(P))
\\ && {} \lor
  RunW(Coll(Q))
}
We get one step law for Weak PML:
picking one of the available ready actions, running it and
continuing execution
constitutes a refinement of that program.
This refinement is all of the behaviours that are possible
after that first (possibly non-deterministic) choice is made.
\RLEQNSs{
\lefteqn{%
   canGo(\varpi)
    \land
    N \in \dom \varpi
    \land isEna(\varpi N) } \nonumber
\\&\implies&
   RunW(\varpi) \refinedby (\varpi N) {\seq} RunW(\varpi)
}
