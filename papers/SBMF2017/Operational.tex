\section{Operational Semantics}\label{sec:operational}

\subsection{Operational Semantics}

We map an initial program and state pair to a final one,
in a standard small-step semantics.

\begin{mathpar}
\inferrule{s \in \sem a s}{a,s \arr{} \Vskip,s'}
\and
\inferrule{ }{\Vskip \cseq P_2,s \arr{} P_2,s}
\and
\inferrule{P_1,s \arr{} P'_1,s'}{P_1 \cseq P_,s \arr{} P'_1 \cseq P_2,s'}
\and
\inferrule{ }{\Vskip \parallel P_2,s \arr{} P_2,s}
\and
\inferrule{ }{P_1 \parallel \Vskip,s \arr{} P_1,s}
\and
\inferrule{P_1,s \arr{} P'_1,s'}{P_1 \parallel P_2,s \arr{} P'_1 \parallel P_2,s'}
\and
\inferrule{P_2,s \arr{} P'_2,s'}{P_1 \parallel P_2,s \arr{} P_1 \parallel P'_2,s'}
\and
\inferrule{ }{P_1 + P_2,s \arr{} P_i,s} i \in \setof{1,2}
\and
\inferrule{ }{P^*,s \arr{} \Vskip,s}
\and
\inferrule{ }{P^*,s \arr{} P\cseq P^*,s}
\and
\inferrule{\sem c s }{P_1 \dcond c P_2,s \arr{} P_1,s}
\and
\inferrule{\lnot\sem c s }{P_1 \dcond c P_2,s \arr{} P_2,s}
\and
\inferrule{\lnot \sem c s }{c \ddo P,s \arr{} \Vskip,s}
\and
\inferrule{\sem c s }{c \ddo P,s \arr{} P\cseq c \ddo P,s}
\end{mathpar}
The multi-step operational transition relation
is defined by the following rules:
\begin{mathpar}
\inferrule{ }{P,s \arrs P,s}
\and
\inferrule
 {P_1,s_1 \arr{} P_2,s_2
  \\
  P_2,s_2 \arrs P_3,s_3
 }
 {P_1,s_1 \arrs P_3,s_3}
\end{mathpar}

Contrast this with the presentation in \S\ref{sec:viewdefs}, Definition 3.


\subsection{Induced Transitions}

We shall write an atomic behavior
that is enabled only when $E \subseteq ls$,
does a state change described by predicate $a$ over $s$ and $s'$
and then removes all of $E$ from $ls$ and adds in the labels in $N$
with the suggestive notation:
\[
 E \arr a N
\]
Here we shall assume that both $E$ and $N$
contain only expressions over $in$, $out$ and $\ell_G$
where $G$ is a generator expression over $g$ only.
Written according to our semantics,
this corresponds to the predicate
$A(E|a|N)$.

\textbf{THE FOLLOWING NEEDS REVISING}
If we define $ii$ as being  $s'=s$,
then we can re-write our semantics as follows:
\begin{eqnarray*}
   a
   &\defs&
   in \arr a out
\\ \cskip
   &\defs&
   in \arr{ii} out
\\ C \cseq D
   &\defs&
   C[g_{:1},\ell_g/g,out] \lor D[g_{:2},\ell_g/g,in]
\\ C + D
   &\defs&
   in \arr{ii} \ell_{g1} \lor in \arr{ii} \ell_{g2}
\\ &\lor&
   C[g_{1:},\ell_{g1}/g,in] \lor D[g_{2:},\ell_{g2}/g,in]
\\ C \parallel D
   &\defs&
   in \arr{ii} \setof{\ell_{g1},\ell_{g2}}
\\ &\lor&
   C[g_{1::},\ell_{g1},\ell_{g1:}/g,in,out]
   \lor D[g_{2::},\ell_{g2},\ell_{g2:}/g,in,out]
\\ &\lor&
   \setof{\ell_{g1:},\ell_{g2:}}
   \arr{ii}
   out
\\ C^*
   &\defs&
   in \arr{ii} out
   \lor
   in \arr{ii} \ell_g
\\ &\lor&
   C[g_{:},\ell_g,in/g,in,out]
\\ run(C)
   &\defs&
   ( ls(\B{out}) * C)[in/ls]
\end{eqnarray*}
We note the following substitution laws:
\begin{eqnarray*}
  && (E \arr a N)[G,L,M/g,in,out]
\\&=& (ls(E) \land s' \in \sem a \land ls'=ls\ominus(E|N))[G,L,M/g,in,out]
\\&=& ls(E)[G,L,M/g,in,out]
       \land s' \in \sem a
\\&& {}\land ls'=ls\ominus(E[G,L,M/g,in,out]|N[G,L,M/g,in,out]))
\\&=& E[G,L,M/g,in,out] \arr a N[G,L,M/g,in,out]
\\
\\&& a[G,L,M/g,in,out] = L \arr a M
\end{eqnarray*}
