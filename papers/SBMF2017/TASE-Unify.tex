\subsection{Bringing It All Together}\label{ha:unify}

We link the weak semantics to the UTCP version
by simply recognising that Weak PML simply views a PML description
as being the parallel composition of all its basic tasks.
We instantiate the generic state observation $s$
in the UTCP theory with the $r$ and $h$ observations of
the Weak theory, and define the semantics of a basic action $A$
as:
\RLEQNSs{
  N?rr!pr
  &\defs&
  rr \subseteq r \land \lnot pr \subseteq r
\\&& {} \land r' = r\cup rr \land h'= h \cat \seqof N
}
Here we drop $ok$ and $ok'$, as its role is subsumed by $ls$  and $ls'$
in the UTCP theory, and instead have a basic action return false when
its required resources are not available,
or all its provided resources
have been so provided.
%We can simply
%translate Weak PML to an equivalent Strong PML description
%by replacing every composite construct with parallel.
%\RLEQNSs{
%   W2S(A) &\defs& A
%\\ W2S(P \oplus Q)
%   &\defs& W2S(P) \parallel W2S(Q)
%   , \qquad \oplus\in\setof{\seq,\pcond,\parallel}
%\\ W2S(\piter P) &\defs& W2S(P)
%}
We note that when a PML description consists solely of basic tasks
and parallel composition, then the Weak, Flexible and Strict PML semantics
coincide since the control flow allows any task to run anytime,
so it is all just down to resource constraints.


%We can also circumvent the requirement for Weak PML task names to be unique
%by using the generator idea to create unique task labels
%which then form the domain of $\varpi$
%---we don't need any static parameters, as we just pass a generator
%in as an extra argument to $Coll$.

\subsection{Future Work}\label{ha:future}

One key issue we face in developing a strict or flexible semantics
is that the iteration construct has no explicit halting condition
associated with it syntactically.
The expectation is that an iteration is repeated until it brings
about a state of affairs that enables \emph{what comes after it}.
%In order to model this compositional,
%we need some way, in a sequential composition $P \seq Q$,
%for $Q$ to be able to signal back to $P$ that it has become ``enabled''.
%This has to be done without $P$ or $Q$ being aware of each others existence.
%This in turn requires that the establishment of that connection,
%between the enablement of $Q$
%and the stopping criterion of $P$ (if it is an iteration),
%has to be the responsibility of the $\seq$ constructor.
%The same notion of enablement can be used to allow the conditional
%construct to preferentially select for an enabled arm.
%
We believe that the UTP approach described in this paper will allow us
to produce a semantics that is compositional:
the iteration simply repeats until it gets a stop signal
sent by the enclosing sequential composition.

The instantiation of the flexible and strict semantics on top of UTCP
still has to be done,
as do the proofs regarding the laws of concurrent programs.
%A definitive list of these can be obtained by a careful reading
%of \cite{DBLP:journals/iandc/Brookes96}
%and consists of identities:
%\RLEQNSs{
%   P \lseq ( Q \lseq R) &=& ( P \lseq Q ) \lseq R
%\\ Idle \lseq P &=& P
%\\ P \lseq Idle &=& P
%\\ P \parallel Q &=& Q \parallel P
%\\ P \parallel ( Q \parallel R) &=& ( P \parallel Q ) \parallel R
%\\ P \parallel Idle &=& P
%\\ P \lcond{true} Q &=& P
%\\ P \lcond{false} Q &=& Q
%\\ (P \lcond c Q) \lseq R &=& (P \lseq R) \lcond c (Q \lseq R)
%\\ c \wdo P &=& (P \lseq c \wdo P) \lcond c Idle
%}
%and refinements:
%\RLEQNSs{
%   P \lseq Q &\refinedby& P \parallel Q
%\\ P \lseq (Q \parallel R) &\refinedby& (P \lseq Q) \parallel R
%\\ P \lcond{c_1 \land c_2} Q &\refinedby&
%   (P \lcond{c_2} Q) \lcond{c_1} Q
%}
The proof sketches done so far show indicate that
the proofs will require demonstrating the existence
of label bijections that respect control-flow,
and that a benefit of constructing such bijections
will be a corresponding operational semantics for UTCP.

We are also working on ideas for a variant of UTCP
that shows promise for being fully compositional:
\emph{i.e.}, without having a static disjunction
of actions, that then needs something like $run$
to make it all dynamic.
We also point out, in the spirit of unification,
that this work has potential to go beyond just PML
but also assist in the development of other theories
of concurrency in UTP,
such as pioneering work in \cite{journals/fac/ZhuHQB15}
addressing the semantics of System-C.

We prefer the UTP denotational/algebraic approach
to that of using operational semantics.
The former typically requires redoing induction proofs
when any change is made,
whereas adding new features and merging models is
much simpler with the latter\cite[p277]{Hoare-He98}.
