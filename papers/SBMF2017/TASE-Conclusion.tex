\section{Conclusions \& Future Work}\label{ha:conc}


We have briefly described Process Modelling Language (PML)
that is used to model business processes,
with clinical pathways being an interesting specific case,
with a view to defining formal semantics for the notation
that allows rigorous reasoning in a flexible manner.
This need arises because clinical pathways in real medical practise
are often interpreted in a loose manner,
although also intended to be quite prescriptive.
%A common issue that arises is that an experienced medic
%might enact such a pathway in a very flexible manner,
%based on their experience, while a junior doctor would be expected,
%and advised, to stick the described protocol far more closely.

We have presented a UTP theory of a ``weak'' interpretation
of PML,
as well as a UTCP theory of concurrency that
acts as a baseline theory on top of which the ``flexible''
and ``strict'' interpretations may be constructed.
A key contribution here is the use of label generators
and the distinction between dynamic state-change observables,
and static context-sensitive parameters.

One other benefit of using the UTP framework here
is that it will ease the integration of application-specific
resource semantic models,
so that we can envisage complete formal analyses
that address application area concerns,
e.g. drug interactions in clinical pathways.
