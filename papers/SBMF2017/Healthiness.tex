\section{Healthiness}\label{sec:health}


\subsection{Wheels-within-Wheels}\label{ssec:WwW}

We are building a semantics based on predicates
that define before-after relations on program state $s,s'$
and label-sets $ls,ls'$, using the static observations
to put things in their syntactical context.
In order to be able to extract the correct behaviour from
this semantics,
it was necessary to have a healthiness condition
that effectively said that every program component,
atomic or composite, has to be viewed as being
willing to run as many times as necessary
whenever its labels would appear in $ls$.
At its simplest,
the semantics required every construct to be
embedded in its own infinite loop,
to ensure it was always ready to ``go''.
This lead to our use of the phrase ``Wheels within Wheels''
(WwW) to refer to this principle.
This did not mean that everything ran forever,
but that, in some sense, it should always be ready.

Technically we require any healthy UTCP program predicate
to be equivalent to  a non-deterministic choice
of how many times it repeats itself, including zero,
using UTP semantic sequential composition.
\RLEQNS{
   P^0 & \defs& \Skip           & \elabel{seq-0}
\\ P^{i+1} & \defs& P \seq P^i  & \elabel{seq-i-plus-1}
\\ \lhsWWW &\defs& \rhsWWW & \elabel{\lblWWW}
} \noindent
Here we have introduced a \emph{stuttering} step,
denoted by UTP's \emph{semantic} skip ($\Skip$).
We note also, that $\WWW$ is monotonic and idempotent.

It should be noticed that this theory underwent
a large number of iterations before the WwW principle was finally
elucidated properly and shown to give the right results.
The number and complexity of the test calculations needed to debug,
develop and validate the theory presented in this paper
necessitated the development of a bespoke ``UTP Calculator''\cite{DBLP:conf/utp/Butterfield16}.

%
% This bold step turns out to be remarkably effective,
% with some quite counter-intuitive outcomes.
% However it does depend on a specific tweak to the
% definition of an atomic action.
% In effect we define an atomic action
% as placing a basic action inside such a loop,
% but within a non-deterministic choice between it
% and a \emph{stuttering} step, denoted by UTP's \emph{semantic} skip ($\Skip$):
% \[
%   \catom a = \true * (\Skip \lor A(in|a|out))
% \]
% A result of this is that this stuttering step gets
% propagated up to enclosing composites,
% so in effect we see $\true * C = \true * (\Skip\lor C)$
% where $C$ is any predicate denoting the semantics of a command.
% Given that our loop bodies always have such a disjunction,
% it then becomes interesting to ask what this looks like:
% \[
%   \true * (\Skip \lor P) = \bigvee_{i \in \Nat} P^i
%   \qquad \elabel{loop-as-NDC}
% \]
% We find that such a loop is equivalent to a non-deterministic
% choice over the number of times that $P$ is repeated,
% including zero.
% We \emph{choose to define} the healthiness condition as
% this large choice.
% \RLEQNS{
%    P^0 & \defs& \Skip           & \elabel{seq-0}
% \\ P^{i+1} & \defs& P \seq P^i  & \elabel{seq-i-plus-1}
% \\ \lhsWWW &\defs& \rhsWWW & \elabel{\lblWWW}
% } \noindent
% We note that $\WWW$ is monotonic, idempotent,
% and can be ``un-nested'':
% \RLEQNS{
%    \WWW(P \lor \WWW(Q)) &=& \WWW(P \lor Q) & \elabel{WwW-unnest}
% }

\subsection{Label-Set Invariants}


The semantics we propose here depends on the careful management
of when specific labels are, or are not,
present in the global label-set $ls$.
Key to the success of this semantics is a collection of label-set invariants
which characterise proper label-set contents,
which are preserved by all label-set manipulations performed
by our semantic definitions.
We have two kinds of invariants,
both of which are concerned with the mutual disjointness, in some sense,
of a collection of sets of labels.
We introduce some shorthand notations
to avoid excessively long predicates and expressions.
We use `$\mid$' as a separator between things meant to be disjoint,
and commas to list subsets and/or set- elements that should be unioned together.
So the fragment $ A,b | M,N | x,Y $
is shorthand for the mutual disjointness of
$A \cup \setof b$ and $M \cup N$ and $\setof x \cup Y$.
To assert mutual set disjointness,
we use the following shorthand, where the $L_i$ are label-sets,
\RLEQNS{
   \setof{L_1|L_2|\dots|L_n}
   &\defs&
   \forall_{i,j \in 1\dots n}
    @
    i \neq j \implies L_i \cap L_j = \emptyset
\\ \multicolumn{3}{c}{\elabel{short-disj-lbl}}
}
We also want to assert that certain sets, necessarily mutually disjoint,
can never have any of their elements in the global label-set,
if any element from one of the other sets is present.
Again, we have a shorthand:
\RLEQNS{
   ~[L_1|L_2|\dots|L_n]
   &\defs&
   \forall_{i,j \in 1\dots n}
    @
    i \neq j \implies
     ( L_i \cap ls \neq \emptyset \implies L_j \cap ls = \emptyset )
\\ \multicolumn{3}{c}{\elabel{short-lbl-exclusive}}
}

The first invariant we have, Disjoint Labels ($DL$) is simply one that asserts,
for every construct, that $in$, $out$ and the labels of $g$
are all different.
\footnote{The theory can be developed using only $g$ as a static observation,
and letting $\ell_g$ and $\ell_{g:}$ play the role of $in$ and $out$
respectively, in which case Disjoint Labels is automatically satisfied.
However, while this results in an entirely equivalent theory,
it is notationally more obscure
making it harder to interpret and check.
}%
\RLEQNS{
   DL &\defs& \setof{in|labs(g)|out} & \elabel{Disjoint-Labels}
}
We shall simplify further by stating that in the shorthands presented
here that we use just simple $g$ to denote $labs(g)$,
so $DL$ can we written as $\{in|g|out\}$.
We also need stronger Label Exclusivity invariants,
regarding which labels can, or cannot,
occur in the global label set at any one time.
There is not one such invariant,
but rather we have that some language constructs may define their own
variation, in order to ensure that flow of control is correctly managed.

There is a general version of the invariant ($LE$) that holds
for all language constructs that asserts that any point in time,
only elements from of one of $in$, $labs(g)$ or $out$
can be present in $ls$ or $ls'$ at any point in time:
\RLEQNS{
   LE &\defs&
   ~[in|g|out] \land [in|g|out]'
   & \ecite{Exclusive-Labels}
}
Note that $[in|g|out]'$ is simply indicates that it refers to $ls'$
rather than $ls$.

So, in summary, we have that every healthy predicate describing
a shared-variable concurrent program's behaviour is of the form $\WWW(C)$
for some predicate $C$ and also satisfies $DL$ and  $LE$.
\RLEQNS{
   \lhsW &\defs& DL \land LE \land \WWW(P) & \elabel{\lblW}
}
We note that many of the axioms for a given construct
in the semantics of Lamport\cite{Lamport1985}
exist to ensure the same properties regarding construct
activation as the healthiness conditions described here.
