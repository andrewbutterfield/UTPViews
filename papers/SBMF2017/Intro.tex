\section{Introduction}\label{sec:Intro}

In this paper we present a compositional semantics
for a simple abstract shared-variable concurrent language,
called the ``Command'' language
presented in Figure \ref{fig:cmd-syntax}
\begin{figure}
  \RLEQNS{
     a &\in& \Atom         & \mbox{Atomic state-change actions}
  \\ C &::=& \catom a      & \mbox{Atomic Command}
  \\  &\mid& C \cseq C     & \mbox{Sequential Composition}
  \\  &\mid& C+C           &\mbox{Non-deterministic Choice}
  \\  &\mid& C \parallel C &\mbox{Parallel Composition}
  \\  &\mid& C^*           &\mbox{Non-deterministic Iteration}
  }  \caption{Command language syntax}
  \label{fig:cmd-syntax}
\end{figure}
The Command language is very simple,
with sequential composition ($C_1 \cseq C_2$),
and only non-deterministic choices,
for alternative execution paths ($C_1 + C_2$)
or deciding when to terminate a loop ($C^*$).
The parallel composition ($C_1 \parallel C_2$)
allows arbitrary interference by each side on any variables,
all of which are considered here to be global and shared.
The semantics we present does not itself need to deal explicitly
with any shared variables,
but simply assumes a shared state $s$
and the existence of atomic state-change actions $a$.
This Command language corresponds directly
to Concurrent Kleene Algebra (CKA)\cite{Hoare2009}.

Our interest in this language stems from our general work within
the Unifying Theories of Programming (UTP) framework\cite{Hoare-He98},
in which we seek to find ways to unify the semantics of a wide range of programming and specification languages, and language features,
in order to be able to reason formally about systems built using a mix of such languages.
The Command language in this paper is based on that introduced in the
``Views'' paper\cite{conf/popl/Dinsdale-YoungBGPY13},
which describes how a range of approaches to reasoning about shared-variable concurrency can be mapped down onto CKA,
and the Command language.
Approaches covered in \cite{conf/popl/Dinsdale-YoungBGPY13} include
various Separation logics\cite{conf/lics/CalcagnoOY07}, type-theories, Owicki-Gries\cite{DBLP:journals/acta/OwickiG76},
and Rely-Guarantee\cite{DBLP:journals/toplas/Jones83}, among others.
Our intention in developing a UTP semantics of the Command language
is to be able to use it as a foundation on which to build UTP
theories of the above approaches that will be easy to link together.
In effect we hope to use the results of the Views paper as a
conceptual architecture to organise our work.

Another independent motivation for this work
is a research collaboration that led us to give a UTP semantics to a process modelling
language called PML\cite{DBLP:journals/infsof/AtkinsonWN07},
which has the notion of basic actions that require certain resources to run,
and which provide further resources as a result.
Actions can be combined
using sequencing, selection, branching and iteration.
We published initial work on a UTP semantics for PML\cite{conf/tase/BMN16},
noting that it is essentially the same as the Command language.
The semantics we gave in \cite{conf/tase/BMN16} was not compositional, however, and finding a fully compositional semantics was noted
for future work.

Compositionality is important.
By it we mean the property that the semantics of a composite
construct can determined from the semantics of its parts,
so for example, the meaning of the construct $C_1\cseq C_2$
would be determined by the meanings of $C_1$ and $C_2$,
combined with the meaning of $\cseq$.
This property is desirable as without it both the semantics
and any reasoning principle based on it would not scale up
to large programs or systems.

The structure of the rest of this paper is as follows:
we describe some related work (Sec. \ref{sec:related}),
followed by an introduction to the UTP methodology (Sec. \ref{sec:UTP}).
We then explain various aspects of our UTP semantics,
touching on labels (Sec. \ref{sec:labels}),
observations (Sec. \ref{sec:observe}),
atomic actions (Sec. \ref{sec:atomic}),
and healthiness conditions (Sec. \ref{sec:health}).
We can then present the semantics in Section \ref{sec:semantics}.
Finally we discuss some calculations
that contribute to the validation of the semantics (Sec. \ref{sec:calc})
and conclude in Section \ref{sec:conc}.
