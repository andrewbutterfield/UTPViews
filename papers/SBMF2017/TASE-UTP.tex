\section{Unifying Theories of Programming}\label{ha:UTP}

The Unifying Theories of Programming framework \cite{Hoare-He98}
uses predicate calculus to define before-after relationships
over appropriate collections of free observation variables.
The before-variables are undashed,
while after-variables have dashes%
\footnote{%
We follow \cite[\S1,p25]{Hoare-He98} in using `dashed',
rather than `primed'
}.
Some observations correspond to the values of program variables ($v,v'$),
while others deal with  important observations one might make of
a running program,
such as start ($ok$) and termination ($ok'$)
or event traces ($tr,tr'$).
The set of observation variables associated with a theory
or predicate is known as its \emph{alphabet}.
For example,
the meaning of an assignment statement might be given as follows:
\begin{equation*}
  x := e  \qquad\defs\qquad  ok \implies ok' \land x' = e \land \nu'=\nu
\end{equation*}
The definition says that once started, the assignment terminates,
with the final value of variable $x$ set equal
to the value of expression $e$ in the before-state,
while the other variables, denoted collectively by $\nu$, remain unchanged.
In UTP,
a predicate of the form $ok \land P \implies ok' \land Q$
is called a \emph{Design},
denoting a total-correctness assertion,
is typically shortened to $P \design Q$.
UTP supports specification languages as well as programming
ones, and a general notion of refinement as universally-closed
reverse implication:
\begin{equation*}
  S \refinedby P \defs [ P \implies S ]
\end{equation*}

A key part of the UTP framework is the use of healthiness conditions
to ensure that the predicates actually correspond to feasible program
outcomes.
These are typically defined
using idempotent and monotonic predicate-transformers ($\H$)
that make an un-healthy predicate healthy ($\H(Bad) = Better$),
while leaving healthy ones untouched ($\H(Good) = Good$).
For example, the unhealthy predicate $\lnot ok \land ok'$
(I've finished even though I didn't start!)
is ruled out by the following predicate-transformer:
\begin{equation*}
   \HH1(P) =  \lnot ok \lor P
\end{equation*}
In contrast,
applying $\HH1$ to the definition of assignment leaves it unchanged.
Most of the healthiness conditions define sets of healthy predicates
that form complete lattices,
where the bottom element is the most liberal specification (``chaos''),
and the top element is
the infeasible program that satisfies any specification (``miracle'').
The healthiness conditions that do not produce lattices
are typically optional ones that can be invoked to rule out miracles,
and which effectively chop off the upper infeasible predicates,
and leaving fully refined programs as maximal elements.

\DRAFT{}
