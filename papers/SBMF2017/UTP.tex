\section{UTP}\label{sec:UTP}

The Unifying Theories of Programming framework \cite{Hoare-He98}
uses predicate calculus to define before-after relationships
over appropriate collections of free observation variables.
The before-variables are undashed,
while after-variables have dashes.
A simple approach would be to simply observe the values of program variables,
in which case the before- and after-\emph{values}
of program variable \texttt{v}
would be represented by observational variables $v$ and $v'$ respectively.
For example,
the meaning of an assignment statement might be given as follows:
\begin{equation*}
  x := e  \qquad\defs\qquad  x' = e \land \nu'=\nu
\end{equation*}
The definition says that the assignment terminates,
with the final value of variable $x$ set equal
to the value of expression $e$ in the before-state,
while the other variables, denoted collectively here  by $\nu$, remain unchanged.
This leads to a theory of partial correctness for imperative programs.

The theory can be extended to cover total correctness by introducing
Boolean observations of program starting ($ok$) and termination ($ok'$).
In this case, we find that we need a technique that allows us to identify
predicates whose interpretation is nonsense, and eliminate them from any
semantic theory we might construct.
For example, the predicate $\lnot ok \land ok'$ describes a situation
in which a program has not started, but has terminated.

In UTP we use the concept of healthiness conditions to specify which predicates
are meaningful in the context of our theory.
For the total correctness theory to work,
we need to ensure that all predicates have the form
$ok \land P \implies ok' \land Q$, where $P$ and $Q$ do not refer to $ok$ or $ok'$.
This is interpreted as saying,
if the program is started and $P$ holds true at the start,
then the program will terminate with $Q$ being satisfied at the end.

% A healthiness condition can be thought of as a higher-order predicate
% taking a predicate as argument and returning \true\ iff the predicate is healthy.
% Considering a healthiness predicate called ``H'',
% we might define something called $\mathbf{isH}$, for instance.
% However, it considerably simplifies most UTP theories
% if we can find a monotonic, idempotent predicate transformer
% instead ($\mathbf{mkH}$)
% that transforms an unhealthy predicate into a healthy one,
% and leaves healthy predicates unchanged.
% Then we can define $\mathbf{isH}(P)$ to be the assertion that tests
% if $P$ is a fixed-point of $\mathbf{mkH}$.
% On of the advantages of this approach is that the set of predicates
% generated by $\mathbf{mkH}$
% inherit the complete lattice property of predicate calculus,
% which makes finding fixpoints for recursion a simple task.
% In most UTP literature the name $\mathbf{H}$ is overloaded
% to denote both $\mathbf{mkH}$ and $\mathbf{isH}$.
%
A standard UTP approach is to define healthy predicates
as being fixed-points of suitable idempotent, monotonic predicate transformers.
For example, in the total correctness theory,
we can define a predicate transformer
$\mathbf{H}(P) \defs ok \implies P$.
A predicate $D$ that satisfies $D = (ok \implies D)$
is one that only asserts its behaviour once it is started ($ok=\true$).
Our healthiness conditions (Sec. \ref{sec:health})
are expressed in this fashion.

An important characteristic of both the UTP theories referred to above,
is that their predicates are interpreted as a relation between the before-state
and after-state of a \emph{complete} program execution.

The ``standard'' treatment of concurrency in UTP\cite[Chps. 7,8]{Hoare-He98},
is focussed on local-state concurrency, without any mutable state variables.
Here it becomes necessary to observe the program state at intermediate points
in its execution,
typically when the program is waiting
for external events to occur.
This necessitates another pair of Boolean observations, $wait$ and $wait'$
that indicate such waiting.
We do not give any further details regarding these theories,
but instead mention them simply to make the observation
that here the predicates are interpreted as a relation between
the before-state,
and some \emph{subsequent} intermediate or final state of the complete execution.
% In particular this allows giving semantics to programs that never terminate,
% such as most server implementations.
% We have no final-state, but many intermediate waiting/response states.

Our focus in this introduction on how the predicates are \emph{interpreted}
in terms of program state is important,
because the theory presented in this paper involves yet more
adjustments in interpretation,
as explained in Sections \ref{sec:observe} and \ref{sec:calc}.


In order to present our UTP semantics of shared-variable
concurrency,
we have to address an issue
that Lamport's semantics\cite{Lamport1985} faced,
namely how to refer to sub-components and their semantics from
within a composite.
In particular he enunciates a number of principles at the start
of his semantics.
One identifies the need to know ``who'' carries out a specific
action, while another says that we need to be able to transform
a statement about command $C$ into one about command $C$ \emph{within
the context of some enclosing construct}.

In the next section we introduce labels and their generators,
which are our approach to addressing these concerns.
We then follow-up with a description of the observation variables
for our theory,
how we handle atomic actions, healthiness conditions,
and then the semantic definitions.
