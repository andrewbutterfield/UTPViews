\section{Calculation Proofs}\label{sec:calc-proofs}

\subsection{Stuff}
This appendix looks like the best place to introduce these shorthands
\RLEQNS{
   ls(\ell) &\defs& \ell \in ls
\\ ls(L) &\defs& L \subseteq ls
\\ ls(\B\ell) &\defs& \ell \notin ls
\\ ls(\B L) &\defs& L \cap ls = \emptyset
}
We define what it means for an atomic action invocation
to satisfy an invariant parameterised on the label type ($Lbl$).
\RLEQNS{
  ls \textbf{ lsat } I \land A(E|a|N) &\implies& ls' \textbf{ lsat } I
}
We can re-formulate this as a following equivalent test:
\RLEQNS{
   A(E|a|N) \textbf{ sat } I_{Lbl}
   &\defs&
   E \textbf{ lsat } I_{Lbl} \land N \textbf{ lsat } I_{Lbl}
}

We present the calculations of normal form examples here.

\subsection{Normal Forms: Sequential Composition}

\subsubsection{Atomic Components}

\subsubsection{Generic Components}

\RLEQNS{
  && P \cseq Q
\EQ{\eref{sem:seq}, with $\DL$}
\\&& [in|\ell_g|out]
     \land \setof{in|g|out}
     \land
     \W( P[\g{:1},\ell_g/g,out]
         \lor
         Q[\g{:2},\ell_g/g,in]   )
}
Shorthands:
We have the following normal forms and shorthands
\RLEQNS{
   P &=& \bigvee_p S_p
\\ Q &=& \bigvee_q T_q
\\ \gamma_1 &=& [\g{:1},\ell_g/g,out]
\\ \gamma_2 &=& [\g{:2},\ell_g/g,in]
\\ I &=& [in|\ell_g|out]
\\ D &=& \{in|g|out\}
\\ D_1 &=& \{in|\g{:1}|\ell_g\}
\\ D_2 &=& \{\ell_g|\g{:2}|out\}
}
So, we shall park the invariants ($I$) and disjointness ($D$) for now,
and focus on $P$ and $Q$ with $\W$.
\RLEQNS{
  && \W( P\gamma_1 \lor Q\gamma_2 )
\EQ{\eref{W-as-NDC}}
\\&& \bigvee_{i \in \Nat} (P\gamma_1 \lor Q\gamma_2)^i
}
Ok, let's try squaring, keeping in mind that we have $I$, $D$, $D_1$
and $D_2$ holding as well.
\RLEQNS{
  && (P\gamma_1 \lor Q\gamma_2)^2
\EQ{$\lor$-$\seq$-distr}
\\&& (P\gamma_1)^2
     \lor
     P\gamma_1\seq Q\gamma_2
     \lor
     Q\gamma_2\seq P\gamma_1
     \lor
     (Q\gamma_2)^2
\EQ{\eref{mumble-closure}}
\\&& P\gamma_1
     \lor
     P\gamma_1\seq Q\gamma_2
     \lor
     Q\gamma_2\seq P\gamma_1
     \lor
     Q\gamma_2
\EQ{Lemma A}
\\&& P\gamma_1
     \lor
     P\gamma_1\seq Q\gamma_2
     \lor
     Q\gamma_2
\EQ{Lemma B}
\\&& P\gamma_1
     \lor
     P\gamma_1\seq Q\gamma_2
     \lor
     Q\gamma_2
}


\paragraph{Lemma A}
We show that $Q\gamma_2\seq P\gamma_1$ vanishes.

We note that we have the following instances of $\DL$:
\[
  [in|g|out] \qquad [in|\g{:1}|\ell_g] \qquad [\ell_g|\g{:2}|out]
\]
\RLEQNS{
  && Q\gamma_2\seq P\gamma_1
\EQ{normal forms}
\\&& (\bigvee_q T_q\gamma_2) \seq (\bigvee_p S_p\gamma_1)
\EQ{$\seq$-$\lor$-distr}
\\&& \bigvee_{q,p} (T_q\gamma_2 \seq S_p\gamma_1)
}
The $T_q\gamma_1$ enable $\g{:2}$ and $out$,
while the $S_p\gamma_1$ are enabled by $in$ and $\g{:1}$.
So all of these terms reduce to $\false.$

\paragraph{Lemma B}
We show that $P\gamma_1 \seq Q\gamma_2$
is all the mumblings that cross from $P$ to $Q$.
\RLEQNS{
  && P\gamma_1\seq Q\gamma_2
\EQ{normal forms}
\\&& (\bigvee_p S_p\gamma_1) \seq (\bigvee_q T_q\gamma_2)
\EQ{$\seq$-$\lor$-distr}
\\&& \bigvee_{p,q}(S_p\gamma_1  \seq  T_q\gamma_2)
\EQ{$X$-composability}
\\&& \bigvee_{p,q}(S_p\gamma_1  \seq  T_q\gamma_2)
\\&& \textbf{where } A_p = \setof{out} \land E_q = \setof{in}
}



\subsection{Normal Forms: Parallel Composition}

\subsubsection{Atomic Components}

\subsubsection{Generic Components}

\subsection{Normal Forms: Non-deterministic Choice}

\subsubsection{Atomic Components}

\subsubsection{Generic Components}

\subsection{Normal Forms: Non-deterministic Iteration}

\subsubsection{Atomic Components}

\subsubsection{Generic Components}

\subsection{Proof of mumble-closure}

\subsection{Normal Forms: Conditional Choice}

\subsubsection{Atomic Components}

\subsubsection{Generic Components}

\subsection{Normal Forms: Conditional Iteration}

\subsubsection{Atomic Components}

\subsubsection{Generic Components}

\subsection{Proof of mumble-closure}

\RLEQNS{
  && P \seq P
\EQ{\eref{NF-as-NDC}, $n \in N$}
\\&& (\bigvee_n X_n) \seq (\bigvee_n X_n)
\EQ{$\seq$-$\lor$-distr, $m,n \in N$}
\\&& \bigvee_{m,n}  (X_m \seq X_n)
\EQ{let $m',n' \in N\setminus\setof{n_0}$}
\\&&\bigvee_{m'}  (X_{m'} \seq X_{n_0})
    \lor
    \bigvee_{n'}  (X_{n_0} \seq X_{n'})
    \lor
    \bigvee_{m',n'}  (X_{m'} \seq X_{n'})
\EQ{by convention, $X_{n_0}=\Skip$}
\\&& \bigvee_{m'}  (X_{m'} \seq \Skip)
     \lor
     \bigvee_{n'} (\Skip \seq X_{n'})
     \lor
     \bigvee_{m',n'}  (X_{m'} \seq X_{n'})
     \EQ{$X\seq\Skip = X = \Skip\seq X$}
\\&& \bigvee_{m'} X_{m'}
     \lor
     \bigvee_{n'} X_{n'}
     \lor
     \bigvee_{m',n'}  (X_{m'} \seq X_{n'})
\EQ{simplify}
\\&& P \lor \bigvee_{m',n'}  (X_{m'} \seq X_{n'})
}
What we need to show is that every term $X_{m'} \seq X_{n'}$
either:
\begin{itemize}
  \item vanishes (infeasible), or:
  \item coincides with $X_p$ for some $p \in N$.
\end{itemize}
