\section{Atomic Heaps}\label{sec:atomicheaps}

\textbf{\emph{Jim,
how closely does the following tie in with what you have been doing?
}}
\subsection{Atomic Heaps as a View}

\textbf{Definition 4 }(Atomic Heap Commands) : \textbf{Parameter A}.
\begin{eqnarray*}
   \vx,\vy &:& \Var
\\ v &\in& \Val
\\ l \in \Loc &\subseteq& \Val
\\ a \in \Atom{H}
   &::=&
   \vx := \vy
   \mid
   [\vx] := v
   \mid
   [\vx] := \vy
   \mid
   \vx := [\vy]
   \mid
   \vx := \reft\vy
\end{eqnarray*}

\textbf{Definition 5} (Heap States) : \textbf{Parameter B}.
\[
 \SS{H} \defs ((\Var \uplus \Loc) \ffun \Val) \uplus \setof{\bang}
\]

\textbf{Definition 6} (Heap Command Interpretation) : \textbf{Parameter C}.
\begin{eqnarray*}
   b \rhd s &\defs& \textit{if }b\textit{ then }s\textit{ else }\bang
\\ \dom(\bang) &\defs& \emptyset
\\ \sem{\vx:=\vy}(s)
              &\defs& \vy \in \dom(s)\rhd \setof{s[\vx \mapsto s(\vy)]}
\\ \sem{\vx:=v}(s)
              &\defs& \vx,s(\vx) \in \dom(s)\rhd \setof{s[s(\vx) \mapsto v]}
\\ \sem{[\vx]:=\vy}(s)
              &\defs& \vx,\vy,s(\vx) \in \dom(s)\rhd \setof{s[s(\vx) \mapsto s(\vy)]}
\\ \sem{\vx:=[\vy]}(s)
              &\defs& \vy,s(\vy) \in \dom(s)\rhd \setof{s[\vx \mapsto s(s(\vy))]}
\\ \sem{\vx:=\reft\vy}(s)
              &\defs& \vy \in \dom(s)\rhd
              \setof{s[\vx \mapsto l,l \mapsto s(\vy)] \mid l \in \Loc \setminus \dom(s) }
\end{eqnarray*}


\subsection{Atomic Heaps in UTP}

All we need to do is to instantiate our state:
\begin{eqnarray*}
  s,s' &:& \SS{H}
\\ &=& ((\Var \uplus \Loc) \ffun \Val) \uplus \setof{\bang}
\end{eqnarray*}


We then define the semantics of each action,
as a binary relation over $s$ and $s'$.
\begin{eqnarray*}
\\ \vx:=\vy
   &\defs& \vy \in \dom(s) \blacktriangleright
           s' = s[\vx \mapsto s(\vy)]
\\ \vx:=v
   &\defs& \vx,s(\vx) \in \dom(s) \blacktriangleright
           s' = s[s(\vx) \mapsto v]
\\ ~[\vx]:=\vy
   &\defs& \vx,\vy,s(\vx) \in \dom(s) \blacktriangleright
           s' = s[s(\vx) \mapsto s(\vy)]
\\ \vx:=[\vy]
   &\defs& \vy,s(\vy) \in \dom(s) \blacktriangleright
           s' = s[\vx \mapsto s(s(\vy))]
\\ \vx:=\reft\vy
              &\defs& \vy \in \dom(s) \blacktriangleright
              \exists l \in \Loc \setminus \dom(s) \bullet
               s' = s[\vx \mapsto l,l \mapsto s(\vy)]
\\ p \blacktriangleright Q &\defs& Q \cond p ( s'=\bang )
\end{eqnarray*}

Open Qs: what is the View here, and the composition?
It looks like we have $\View_{H} = \SS{H}$.
\begin{eqnarray*}
   \bang * s_2 &\defs& \bang
\\ s_1 * \bang &\defs& \bang
\\  s_1 * s_2
   &\defs?& s_1
   \uplus s_2
   \cond{\dom(s_1)\cap\dom(s_2)=\emptyset}
   \bang
\end{eqnarray*}
Is this too harsh?
Should we except overlaps so long as they agree?
\begin{eqnarray*}
   s_1 * s_2
   &\defs?& s_1
   \cup s_2
   \cond{~(\dom(s_1)\lhd s_2) = (\dom(s_2)\lhd s_1)~}
   \bang
\end{eqnarray*}
We can get triples by the rule:
\[
  (s_1,a,s_2) \textbf{ iff } a[s_1,s_2/s,s']
\]
where the first occurrence of $a$ represents its text,
whilst the second is the corresponding semantic predicate.
Or, using the notation in Chapter 10 of the UTP book:
\[
  (s_1,\texttt{a},s_2) \textbf{ iff } a[s_1,s_2/s,s']
\]
