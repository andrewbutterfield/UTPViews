\section{Commands}


We present the commands here,
with a notational tweak that suits a UTP encoding.

\subsection{Syntax}

\begin{eqnarray*}
   a &\in& \Atom{}
\\ C &::=& a \mid \cskip \mid C \cseq C \mid C+C \mid C \parallel C \mid C^*
\\ g &:& Gen
\\ \ell &:& Lbl
\\ G &::=&  g \mid G_{:} \mid G_1 \mid G_2
\\ L &::=& \ell_G
\end{eqnarray*}

\subsection{Domains}
\begin{eqnarray*}
   s &\in& \mathcal S
\\ \sem{-} &:& \Atom \fun \mathcal S \fun \mathcal P(\mathcal S)
\\ S \ominus (T|V)
   &\defs& (S \setminus T) \cup V
\end{eqnarray*}

\subsection{Alphabet}

\begin{eqnarray*}
   s, s' &:& \mathcal S
\\ ls, ls' &:& \mathcal P (Lbl)
\\ g &:& Gen
\\ in, out &:& Lbl
\end{eqnarray*}

\subsection{``Standard'' UTP Constructs}

\begin{eqnarray*}
   P \cond c Q
   &\defs&
   c \land P \lor \lnot c \land Q
\\ P ; Q
   &\defs&
   \exists s_m,ls_m \bullet P[s_m,ls_m/s',ls'] \land Q[s_m,ls_m/s,ls]
\\ c * P
   &=&
   P ; c * P \cond c \Skip
\end{eqnarray*}

\subsection{Shorthands}

\begin{eqnarray*}
   ls(\ell) &\defs& \ell \in ls
\\ ls(L) &\defs& L \subseteq ls
\\ ls(\B\ell) &\defs& \ell \notin ls
\\ ls(\B L) &\defs& L \cap ls = \emptyset
\end{eqnarray*}

\subsection{Simple Semantics}


\begin{eqnarray*}
   a
   &\defs&
   ls(in) \land s' \in \sem a s \land ls'=ls \ominus (in|out)
\\ \cskip
   &\defs&
   ls(in) \land s' = s \land ls'=ls \ominus (in|out)
\\ C \cseq D
   &\defs&
   C[g_{:1},\ell_g/g,out] \lor D[g_{:2},\ell_g/g,in]
\\ C + D
   &\defs&
   ls(in)
   \land s' = s
   \land ls' \in \{ ls \ominus (in,\ell_{g2}|\ell_{g1})
                  , ls \ominus (in,\ell_{g1}|\ell_{g2})\}
\\ &\lor&
   C[g_{1:},\ell_{g1}/g,in] \lor D[g_{2:},\ell_{g2}/g,in]
\\ C \parallel D
   &\defs&
   ls(in)
   \land s' = s
   \land ls'= ls \ominus (in|\ell_{g1},\ell_{g2})
\\ &\lor&
   C[g_{1::},\ell_{g1},\ell_{g1:}/g,in,out]
   \lor D[g_{2::},\ell_{g2},\ell_{g2:}/g,in,out]
\\ &\lor&
   ls(\ell_{g1:},\ell_{g2:})
   \land s' = s
   \land ls'= ls \ominus (\ell_{g1:},\ell_{g2:}|out)
\\ C^*
   &\defs&
   ls(in)
   \land s' = s
   \land ls' \in \{ ls \ominus (in|out), ls \ominus (in|\ell_g)\}
\\ &\lor&
   C[g_{:},\ell_g,in/g,in,out]
\\ run(C)
   &\defs&
   ( ls(\B{out}) * C)[in/ls]
\end{eqnarray*}

\subsection{Simple Calculations}

Careful calculations of simple cases (atomic components only)
give the following results:
\begin{eqnarray*}
   run(a) &=& a \land ls'=out
\\ run(\cskip) &=& s'=s \land ls'=out
\\ run(a\cseq b) &=& (a\seq b) \land ls'=out
\\ run(a \parallel b) &=& ((a \seq b) \lor (b \seq a)) \land ls'=out
\\ run(a + b) &=& (a \lor b) \land ls'=out
\\ run(a^*) &=& run(\cskip) \lor run(a) \lor run(a\cseq a) \lor \dots
\end{eqnarray*}
What is notable is that the only way in which atomic state changes
get combined in the final result is using either disjunction
or sequential composition
that is effectively restricted to relations on shared state only:
\[
  a \seq b = \exists s_m \bullet a[s_m/s'] \land b[s_m/s]
\]


\subsection{Induced Transitions}

We shall write an atomic behavior
that is enabled only when $E \subseteq ls$,
does a state change described by predicate $a$ over $s$ and $s'$
and then removes all of $E$ from $ls$ and adds in the labels in $N$
with the suggestive notation:
\[
 E \arr a N
\]
Here we shall assume that both $E$ and $N$
contain only expressions over $in$, $out$ and $\ell_G$
where $G$ is a generator expression over $g$ only.
Written according to our semantics,
this corresponds to the predicate
\[
  ls(E) \land s' \in \sem a \land ls'=ls\ominus(E|N)
\]

If we define $ii$ as being  $s'=s$,
then we can re-write our semantics as follows:
\begin{eqnarray*}
   a
   &\defs&
   in \arr a out
\\ \cskip
   &\defs&
   in \arr{ii} out
\\ C \cseq D
   &\defs&
   C[g_{:1},\ell_g/g,out] \lor D[g_{:2},\ell_g/g,in]
\\ C + D
   &\defs&
   in \arr{ii} \ell_{g1} \lor in \arr{ii} \ell_{g2}
\\ &\lor&
   C[g_{1:},\ell_{g1}/g,in] \lor D[g_{2:},\ell_{g2}/g,in]
\\ C \parallel D
   &\defs&
   in \arr{ii} \setof{\ell_{g1},\ell_{g2}}
\\ &\lor&
   C[g_{1::},\ell_{g1},\ell_{g1:}/g,in,out]
   \lor D[g_{2::},\ell_{g2},\ell_{g2:}/g,in,out]
\\ &\lor&
   \setof{\ell_{g1:},\ell_{g2:}}
   \arr{ii}
   out
\\ C^*
   &\defs&
   in \arr{ii} out
   \lor
   in \arr{ii} \ell_g
\\ &\lor&
   C[g_{:},\ell_g,in/g,in,out]
\\ run(C)
   &\defs&
   ( ls(\B{out}) * C)[in/ls]
\end{eqnarray*}
We note the following substitution laws:
\begin{eqnarray*}
  && (E \arr a N)[G,L,M/g,in,out]
\\&=& (ls(E) \land s' \in \sem a \land ls'=ls\ominus(E|N))[G,L,M/g,in,out]
\\&=& ls(E)[G,L,M/g,in,out]
       \land s' \in \sem a
\\&& {}\land ls'=ls\ominus(E[G,L,M/g,in,out]|N[G,L,M/g,in,out]))
\\&=& E[G,L,M/g,in,out] \arr a N[G,L,M/g,in,out]
\\
\\&& a[G,L,M/g,in,out] = L \arr a M
\end{eqnarray*}

\subsection{Finding Bisimulations}


Idea: define the desired bisimulations in such a way that
we can manipulate them with the same kind of substitutions as found. in
the semantics.

\subsubsection{\protect{$a \parallel b = b \parallel a$}}

Consider a simple example: prove that $a \parallel b = b \parallel a$,

\RLEQNS{
  && a \parallel b
\EQ{defn of $\parallel$ }
\\&&      in \arr{ii} \setof{\ell_{g1},\ell_{g2}}
     \lor \setof{\ell_{g1:},\ell_{g2:}} \arr{ii} out
\\ &\lor&
   a[g_{1::},\ell_{g1},\ell_{g1:}/g,in,out]
   \lor b[g_{2::},\ell_{g2},\ell_{g2:}/g,in,out]
\EQ{defn of $a$ and $b$}
\\&&      in \arr{ii} \setof{\ell_{g1},\ell_{g2}}
     \lor \setof{\ell_{g1:},\ell_{g2:}} \arr{ii} out
\\ &\lor&
   (in \arr a out)[g_{1::},\ell_{g1},\ell_{g1:}/g,in,out]
   \lor (in \arr b out)[g_{2::},\ell_{g2},\ell_{g2:}/g,in,out]
\EQ{substitutions, as per above}
\\&& in \arr{ii} \setof{\ell_{g1},\ell_{g2}}
     \lor
     \ell_{g1} \arr a \ell_{g1:}
     \lor
     \ell_{g2} \arr b \ell_{g2:}
     \lor
     \setof{\ell_{g1:},\ell_{g2:}} \arr{ii} out
\\
\\&& b \parallel a
\EQ{Above calculations with $a$ and $b$ swapped}
\\&& in \arr{ii} \setof{\ell_{g1},\ell_{g2}}
     \lor
     \ell_{g1} \arr b \ell_{g1:}
     \lor
     \ell_{g2} \arr a \ell_{g2:}
     \lor
     \setof{\ell_{g1:},\ell_{g2:}} \arr{ii} out
}

If we define%
\footnote{If is a label is not mentioned, it maps to itself}
the following bijection on labels:
\[
  \ell_{g1} \mapsto \ell{g2}
, \ell_{g1:} \mapsto \ell{g2:}
, \ell_{g2} \mapsto \ell{g1}
, \ell_{g2:} \mapsto \ell{g1:}
\]
we can show that the two sides are isomorphic modulo that bijection.
Interestingly, this isomorphism works if we define the following
generator bijection $g1 \rightleftharpoons g2$.

\subsubsection{\protect{$(a \cseq b)\parallel c = c \parallel (a \cseq b)$}}

Now, a bit more ambitious---consider
\[ (a \cseq b)\parallel c
    = c \parallel (a \cseq b)
\]

LHS:
\RLEQNS{
   && (a \cseq b)\parallel c
\EQ{defn $\parallel$}
\\&& in \arr{ii} \setof{\ell_{g1},\ell_{g2}}
      \lor \setof{\ell_{g1:},\ell_{g2:}} \arr{ii} out \lor {}
\\ &&
   (a \cseq b)[g_{1::},\ell_{g1},\ell_{g1:}/g,in,out]
   \lor c[g_{2::},\ell_{g2},\ell_{g2:}/g,in,out]
\EQ{defn $\cseq$}
\\&& in \arr{ii} \setof{\ell_{g1},\ell_{g2}}
      \lor \setof{\ell_{g1:},\ell_{g2:}} \arr{ii} out \lor {}
\\ &&
   (a[g_{:1},\ell_g/g,out] \lor b[g_{:2},\ell_g/g,in])[g_{1::},\ell_{g1},\ell_{g1:}/g,in,out]
\\&& {}\lor c[g_{2::},\ell_{g2},\ell_{g2:}/g,in,out]
\EQ{atomic substitution}
\\&& in \arr{ii} \setof{\ell_{g1},\ell_{g2}}
      \lor \setof{\ell_{g1:},\ell_{g2:}} \arr{ii} out \lor {}
\\ &&
   (in \arr a \ell_g \lor \ell_g \arr b out)[g_{1::},\ell_{g1},\ell_{g1:}/g,in,out]
\\&& {}\lor \ell_{g2} \arr c\ell_{g2:}
\EQ{substitution}
\\&& in \arr{ii} \setof{\ell_{g1},\ell_{g2}}
      \lor \setof{\ell_{g1:},\ell_{g2:}} \arr{ii} out \lor {}
\\ &&
   \ell_{g1} \arr a \ell_{g1::}
   \lor \ell_{g1::} \arr b \ell_{g1:}
   \lor \ell_{g2} \arr c\ell_{g2:}
}

RHS:
\RLEQNS{
  && c \parallel (a \cseq b)
\EQ{defn $\parallel$}
\\&& in \arr{ii} \setof{\ell_{g1},\ell_{g2}}
      \lor \setof{\ell_{g1:},\ell_{g2:}} \arr{ii} out \lor {}
\\ &&
   c[g_{1::},\ell_{g1},\ell_{g1:}/g,in,out]
   \lor (a \cseq b)[g_{2::},\ell_{g2},\ell_{g2:}/g,in,out]
\EQ{defn $\cseq$}
\\&& in \arr{ii} \setof{\ell_{g1},\ell_{g2}}
      \lor \setof{\ell_{g1:},\ell_{g2:}} \arr{ii} out \lor {}
\\ &&
   c[g_{1::},\ell_{g1},\ell_{g1:}/g,in,out] \lor {}
\\&& (a[g_{:1},\ell_g/g,out] \lor b[g_{:2},\ell_g/g,in])[g_{2::},\ell_{g2},\ell_{g2:}/g,in,out]
\EQ{atomic substitution}
\\&& in \arr{ii} \setof{\ell_{g1},\ell_{g2}}
      \lor \setof{\ell_{g1:},\ell_{g2:}} \arr{ii} out \lor {}
\\ &&
   \ell_{g1} \arr c \ell_{g1:} \lor {}
\\&& (in \arr a \ell_g \lor \ell_g \arr b out)[g_{2::},\ell_{g2},\ell_{g2:}/g,in,out]
\EQ{substitution}
\\&& in \arr{ii} \setof{\ell_{g1},\ell_{g2}}
      \lor \setof{\ell_{g1:},\ell_{g2:}} \arr{ii} out \lor {}
\\ &&
   \ell_{g1} \arr c \ell_{g1:}
   \lor \ell_{g2} \arr a \ell_{g2::}
   \lor \ell_{g2::} \arr b \ell_{g2:}
}

Reprise LHS:
\[
in \arr{ii} \setof{\ell_{g1},\ell_{g2}} \lor
\setof{\ell_{g1:},\ell_{g2:}} \arr{ii} out \lor
\ell_{g1} \arr a \ell_{g1::} \lor
\ell_{g1::} \arr b \ell_{g1:} \lor
\ell_{g2} \arr c\ell_{g2:}
\]
Reprise RHS:
\[
in \arr{ii} \setof{\ell_{g1},\ell_{g2}} \lor
\setof{\ell_{g1:},\ell_{g2:}} \arr{ii} out \lor
\ell_{g1} \arr c \ell_{g1:} \lor
\ell_{g2} \arr a \ell_{g2::} \lor
\ell_{g2::} \arr b \ell_{g2:}
\]
Reorder LHS:
\[
in \arr{ii} \setof{\ell_{g2},\ell_{g1}} \lor
\setof{\ell_{g2:},\ell_{g1:}} \arr{ii} out \lor
\ell_{g2} \arr c\ell_{g2:} \lor
\ell_{g1} \arr a \ell_{g1::} \lor
\ell_{g1::} \arr b \ell_{g1:}
\]
We immediately see the following bijection: $g1 \leftrightharpoons g2$.

\subsubsection{\protect{$(a \cseq b) \cseq c   = a \cseq (b \cseq c)$}}

Now we look at the following:
\[
  (a \cseq b) \cseq c   = a \cseq (b \cseq c)
\]

LHS:
\RLEQNS{
  && (a \cseq b) \cseq c
\EQ{defn $\cseq$}
\\&& (a \cseq b)[g_{:1},\ell_g/g,out] \lor c[g_{:2},\ell_g/g,in]
\EQ{defn $\cseq$}
\\&& (a[g_{:1},\ell_g/g,out] \lor b[g_{:2},\ell_g/g,in])[g_{:1},\ell_g/g,out]
     \lor c[g_{:2},\ell_g/g,in]
\EQ{atomic substitution}
\\&& (in \arr a out \lor \ell_g \arr b out)[g_{:1},\ell_g/g,out]
     \lor \ell_g \arr c out
\EQ{substitution}
\\&& in \arr a \ell_{g:1} \lor \ell_{g:1} \arr b \ell_g \lor \ell_g \arr c out
}

% C[g_{:1},\ell_g/g,out] \lor D[g_{:2},\ell_g/g,in]
RHS:
\RLEQNS{
  && a \cseq (b \cseq c)
\EQ{defn $\cseq$}
\\&& a[g_{:1},\ell_g/g,out] \lor (b \cseq c)[g_{:2},\ell_g/g,in]
\EQ{defn $\cseq$}
\\&& a[g_{:1},\ell_g/g,out] \lor
  (b[g_{:1},\ell_g/g,out] \lor c[g_{:2},\ell_g/g,in])[g_{:2},\ell_g/g,in]
\EQ{atomic substitution}
\\&& in \arr a \ell_g \lor
  (in \arr b \ell_g \lor \ell_g \arr c out)[g_{:2},\ell_g/g,in]
\EQ{substitution}
\\&& in \arr a \ell_g \lor \ell_g \arr b \ell_{g:2} \lor \ell_{g:2} \arr c out
}

We see the following bijection:
$\ell_{g:1} \leftrightharpoons \ell_{g},
 \ell_{g}   \leftrightharpoons \ell_{g:2}
$

\subsubsection{Bijective Substitutions}

Idea: aren't all our substitutions actually bijections?
\begin{eqnarray*}
   \cseq
   &&
   [g_{:1},\ell_g/g,out]
   [g_{:2},\ell_g/g,in]
\\ +
   &&
   [g_{1:},\ell_{g1}/g,in]
   [g_{2:},\ell_{g2}/g,in]
\\ \parallel
   &&
   [g_{1::},\ell_{g1},\ell_{g1:}/g,in,out]
   [g_{2::},\ell_{g2},\ell_{g2:}/g,in,out]
\\ {}^*
   &&
   [g_{:},\ell_g,in/g,in,out]
\end{eqnarray*}
Under reasonable assumptions about the structure or and use of everything,
at least.
