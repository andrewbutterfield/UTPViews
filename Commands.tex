\section{Commands}

We do a quick run-down of the Commands\cite{conf/popl/Dinsdale-YoungBGPY13}.

\subsection{Syntax}

\begin{eqnarray*}
   a &\in& \Atom
\\ C &::=& a \mid \cskip \mid C \cseq C \mid C+C \mid C \parallel C \mid C^*
\\ g &:& Gen
\\ \ell &:& Lbl
\\ G &::=&  g \mid G_{:} \mid G_1 \mid G_2
\\ L &::=& \ell_G
\end{eqnarray*}

\subsection{Domains}
\begin{eqnarray*}
   s &\in& \mathcal S
\\ \sem{-} &:& \Atom \fun \mathcal S \fun \mathcal P(\mathcal S)
\\ S \ominus (T|V)
   &\defs& (S \setminus T) \cup V
\end{eqnarray*}

\subsection{Alphabet}

\begin{eqnarray*}
   s, s' &:& \mathcal S
\\ ls, ls' &:& \mathcal P (Lbl)
\\ g &:& Gen
\\ in, out &:& Lbl
\end{eqnarray*}

\subsection{``Standard'' UTP Constructs}

\begin{eqnarray*}
   P \cond c Q
   &\defs&
   c \land P \lor \lnot c \land Q
\\ P ; Q
   &\defs&
   \exists s_m,ls_m \bullet P[s_m,ls_m/s',ls'] \land Q[s_m,ls_m/s,ls]
\\ c * P
   &=&
   P ; c * P \cond c \Skip
\end{eqnarray*}

\subsection{Shorthands}

\begin{eqnarray*}
   ls(\ell) &\defs& \ell \in ls
\\ ls(L) &\defs& L \subseteq ls
\\ ls(\B\ell) &\defs& \ell \notin ls
\\ ls(\B L) &\defs& L \cap ls = \emptyset
\end{eqnarray*}

\subsection{Simple Semantics}

This follows the TASE2016 Submission style,
with a simpler version of $run$.

\begin{eqnarray*}
   a
   &\defs&
   ls(in) \land s' \in \sem a s \land ls'=ls \ominus (in|out)
\\ \cskip
   &\defs&
   ls(in) \land s' = s \land ls'=ls \ominus (in|out)
\\ C \cseq D
   &\defs&
   C[g_{:1},\ell_g/g,out] \lor D[g_{:2},\ell_g/g,in]
\\ C + D
   &\defs&
   ls(in)
   \land s' = s
   \land ls' \in \{ ls \ominus (in,\ell_{g2}|\ell_{g1})
                  , ls \ominus (in,\ell_{g1}|\ell_{g2})\}
\\ &\lor&
   C[g_{1:},\ell_{g1}/g,in] \lor D[g_{2:},\ell_{g2}/g,in]
\\ C \parallel D
   &\defs&
   ls(in)
   \land s' = s
   \land ls'= ls \ominus (in|\ell_{g1},\ell_{g2})
\\ &\lor&
   C[g_{1::},\ell_{g1},\ell_{g1:}/g,in,out]
   \lor D[g_{2::},\ell_{g2},\ell_{g2:}/g,in,out]
\\ &\lor&
   ls(\ell_{g1:},\ell_{g2:})
   \land s' = s
   \land ls'= ls \ominus (\ell_{g1:},\ell_{g2:}|out)
\\ C^*
   &\defs&
   ls(in)
   \land s' = s
   \land ls' \in \{ ls \ominus (in|out), ls \ominus (in|\ell_g)\}
\\ &\lor&
   C[g_{:},\ell_g,in/g,in,out]
\\ run(C)
   &\defs&
   ( ls(\B{out}) * C)[in/ls]
\end{eqnarray*}


\newpage
\subsection{Wheels-within-Wheels}

This is the new all-shiny fully compositional form%
\footnote{We hope !}%
.


\subsubsection{WwW Semantic Definitions}

We introduce the following shorthands:
\begin{eqnarray*}
   \ado a &\defs& \atomdo a
\\ \LUPD S T &\defs& \lupd S T
\\ \LUPD S {T \ndc V}
   &\defs&
   ls' \in \setof{\lexc S T, \lexc S V}
\\ \LNDC{S_1}{T_1}{S_2}{T_2}
   &\defs&
   ls' \in \setof{\lexc{S_1}{T_1}, \lexc{S_2}{T_2}}
\\ \LS S &\defs& ls(S)
\\ \LSd S &\defs& ls'(S)
\\ \LS{\B S} &\defs& ls(\B S)
\\ \LSd{\B S} &\defs& ls'(\B S)
\end{eqnarray*}

These  shorthands satisfy the following laws:
\begin{eqnarray*}
   \LUPD S T &=& \LUPD S T \land \LSd T
\\ \LUPD S T &=& \LUPD S T \land \LSd{\B S} \land \LSd T, \textbf{ if }S \cap T = \emptyset
\\ \LUPD S {T\ndc V} &=& \LUPD S T \land (\LSd T \lor \LSd V)
\\ \LNDC{S_1}{T_1}{S_2}{T_2}
   &=&
   \LNDC{S_1}{T_1}{S_2}{T_2}  \land (\LSd{T_1} \lor \LSd{T_2})
\\ &=& \textbf{ if }S_i \cap T_i = \emptyset \textbf{ then\ldots}
\\&& \LNDC{S_1}{T_1}{S_2}{T_2}
\\&& {} \land (\LSd{\B{S_1}}\land\LSd{T_1}
            \lor
            \LSd{\B{S_2}}\land\LSd{T_2})
\\ P \land \LSd T ; Q
   &=&
   P ; \LS T \land Q
\\ P \land \LUPD S T ; Q
   &=&
   P \land \LUPD S T ; \LS T \land Q
\end{eqnarray*}

The definitions:
\begin{eqnarray*}
   \W(C) &\defs& ls(\B{out}) * C)
\\
\\ a &\defs&\W(~\LS{in} \land \ado a \land \LUPD{in}{out}~)
\\ \cskip
   &\defs&
   \W(~\LS{in} \land \ss \land \LUPD{in}{out}~)
\\
\\ C \cseq D
   &\defs&
   \W(C[g_{:1},\ell_g/g,out] \lor D[g_{:2},\ell_g/g,in])
\\
\\ C + D
   &\defs&
   \W(~\LS{in}
   \land \ss
   \land \LNDC{in,\ell_{g2}}{\ell_{g1}}
                {in,\ell_{g1}}{\ell_{g2}}
\\ && \qquad {} \lor
   C[g_{1:},\ell_{g1}/g,in] \lor D[g_{2:},\ell_{g2}/g,in]~)
\\
\\ C \parallel D
   &\defs&
   \W(~\LS{in}
   \land \ss
   \land \LUPD{in}{\ell_{g1},\ell_{g2}}
\\ && \qquad {}\lor
   C[g_{1::},\ell_{g1},\ell_{g1:}/g,in,out]
   \lor D[g_{2::},\ell_{g2},\ell_{g2:}/g,in,out]
\\ && \qquad {}\lor
   \LS{\ell_{g1:},\ell_{g2:}}
   \land \ss
   \land \LUPD{\ell_{g1:},\ell_{g2:}}{out}~)
\\
\\ C^*
   &\defs&
   \W(~\LS{in}
   \land \ss
   \land \LUPD{in}{out \ndc \ell_g}
\\ && \qquad {}\lor
   C[g_{:},\ell_g,in/g,in,out]~)
\end{eqnarray*}

\subsubsection{WwW Calculations/Results}

We will start by explaining a calculation method
that should help structure our reasoning about loops.
We consider a generic iteration $c*P$,
and note the following identity,
obtained by repeated application of the loop unrolling law
coupled with expansion of the definition of conditionals:
\begin{equation}\label{eqn:unroll-n-times}
   c * P
   \quad=\quad
   \bigvee_{i=0}^{n-1} ( (c \land P)^i ; \lnot c \land \Skip)
   \;\lor\;
   (c \land P)^n ; c * P
\end{equation}
From this we define the following shorthands,
and suggest two important calcualtions:
\begin{eqnarray*}
   W &\defs& c * P
\\ D &\defs& \lnot c \land \Skip \EOLC{Done}
\\ S &\defs& c \land P \EOLC{Step}
\\ L &\defs& S ; D \EOLC{Last}
\\ T &\defs& D ; D \EOLC{Two-Step}
\end{eqnarray*}

We note some laws that apply to composition of $D$ with various
things, given certain substitutions, and noting that for us,
$D$ always has a specific instantiation:
\begin{eqnarray*}
   D &=& \LS{out} \land \Skip
\\ D ; D &= & D
\\ P \land \LSd{\ell} ; D[\ell/out] &=&  P \land \LSd{\ell}
\\ P \land \LSd{\B\ell} ; D[\ell/out] &=&  false
\\ D[\ell_1/out] ; D[\ell_2/out] &=& false,\textbf{ if }\ell_1/\neq\ell_2
\end{eqnarray*}

We now illustrate with the definition of $a$,
using subscripts indicating that we are instantiating the above shorthands
w.r.t. the semantics of $a$:

\begin{eqnarray*}
   a &\defs& \W( \LS{in} \land \ado a \land \LUPD{in}{out} )
\\ &=& W_a
\\ D &\defs& \D \EOLC{Independent of $a$.}
\\ S_a &\defs& \LS{\B{out}}
               \land \LS{in} \land \ado a \land \LUPD{in}{out}
\\ L_a &\defs& S_a ; D
\\ &=& \LS{\B{out}}
               \land \LS{in} \land \ado a \land \LUPD{in}{out}
\\ && {} ; \LS{out} \land \Skip
\EQ{shorthand law, and $\Skip$ is $;$-identity}
\\ && \LS{\B{out}}
               \land \LS{in} \land \ado a \land \LUPD{in}{out}
\\ T_a &\defs& S_a ; S_a
\\ &=& \LS{\B{out}}
               \land \LS{in} \land \ado a \land \LUPD{in}{out}
\\ && {} ; \LS{\B{out}}
               \land \LS{in} \land \ado a \land \LUPD{in}{out}
\EQ{shorthand law}
\\ && \LS{\B{out}}
               \land \LS{in} \land \ado a \land \LUPD{in}{out}
\\ && {} ; \LS{out} \land \LS{\B{out}}
               \land \LS{in} \land \ado a \land \LUPD{in}{out}
\\ &=& false
\end{eqnarray*}
We see that all terms $S_a^i$ for $i \geq 2$ just vanish, so
\begin{eqnarray*}
   a &=& D \lor L_a
\\ &=& \D \;\lor\; \LS{\B{out}}
               \land \LS{in} \land \ado a \land \LUPD{in}{out}
\end{eqnarray*}


We overload sequential composition as follows:
\begin{eqnarray*}
   \ado{a;b} &=& \ado a ; \ado b
\end{eqnarray*}

Now we look at $a \cseq b$:
\begin{eqnarray*}
   a \cseq b
   &=& \W(a[g_{:1},\ell_g/g,out] \lor b[g_{:2},\ell_g/g,in])
\\ &=& W_;
\\ D &\defs& \D \EOLC{Independent of $a\cseq b$.}
\\ S_; &=& a[g_{:1},\ell_g/g,out] \lor b[g_{:2},\ell_g/g,in]
\EQ{calc. of $a$ above}
\\ && (D \lor L_a)[g_{:1},\ell_g/g,out]
      \lor
      (D \lor L_b)[g_{:2},\ell_g/g,in]
\\&=& \LS{\ell_g} \land \Skip
      \lor \LS{\B{\ell_g}}
           \land \LS{in} \land \ado a \land \LUPD{in}{\ell_g} \lor {}
\\&& \LS{out} \land \Skip
      \lor \LS{\B{out}}
           \land \LS{\ell_g} \land \ado b \land \LUPD{\ell_g}{out}
\\ L_; &=& S_; ; D
\\ &=& (~\LS{\ell_g} \land \Skip
      \lor \LS{\B{\ell_g}}
           \land \LS{in} \land \ado a \land \LUPD{in}{\ell_g} \lor {}
\\&& \LS{out} \land \Skip
      \lor \LS{\B{out}}
           \land \LS{\ell_g} \land \ado b \land \LUPD{\ell_g}{out}~)
\\&& {} ; \LS{out} \land \Skip
\\ &=& (\LS{\ell_g} \land \Skip ; \LS{out} \land \Skip) \lor{}
\\ &&  (\LS{\B{\ell_g}}
       \land \LS{in} \land \ado a \land \LUPD{in}{\ell_g}
       ; \LS{out} \land \Skip) \lor {}
\\ &&  (\LS{out} \land \Skip ; \LS{out} \land \Skip )\lor {}
\\ &&  (\LS{\B{out}}
           \land \LS{\ell_g} \land \ado b \land \LUPD{\ell_g}{out}
           ; \LS{out} \land \Skip )
\\ &=& \LS{out} \land \Skip \lor
       \LS{\B{out}}
       \land \LS{\ell_g} \land \ado b \land \LUPD{\ell_g}{out}
\\ S_; ; L_; &=&
     \LS{\ell_g} \land \Skip \lor{}
\\&& \LS{\B{\ell_g}}
           \land \LS{in} \land \ado a \land \LUPD{in}{\ell_g} \lor {}
\\&& \LS{out} \land \Skip \lor {}
\\&&\LS{\B{out}}
           \land \LS{\ell_g} \land \ado b \land \LUPD{\ell_g}{out}
\\&& ;
\\&& \LS{out} \land \Skip \lor {}
\\&& \LS{\B{out}}
       \land \LS{\ell_g} \land \ado b \land \LUPD{\ell_g}{out}
\end{eqnarray*}
We pause here, to do a little methodological innovation.
We introduce the notion of a ``label-type''
which is a  list of labels (in $\ltype\dots$),
which might be dashed,
and might be prefixed with negation.
We interpret these as follows:

\begin{tabular}{|c|c|}
  \hline
  % after \\: \hline or \cline{col1-col2} \cline{col3-col4} ...
  $\ell$ & we assert $ls(\ell)$ \\
  $\B\ell$ & we assert $\lnot ls(\ell)$ \\
  $\ell'$ & we assert $ls'(\ell)$ \\
  $\B\ell'$ & we assert $\lnot ls'(\ell)$ \\
  \hline
\end{tabular}

These lists are interpreted as conjunction.
This helps us to quickly spot predicates that reduce to
contradictions when sequentially composed.
Here is $S_; ; L_;$ with label-types added
\RLEQNS{
   S_; ; L_; &=&
     \LS{\ell_g} \land \Skip \lor{}
      & \ltype{\ell_g,\ell'_g}
\\&& \LS{\B{\ell_g}}
           \land \LS{in} \land \ado a \land \LUPD{in}{\ell_g} \lor {}
      & \ltype{\B{\ell_g},in,\B{in'}, \ell'_g}
\\&& \LS{out} \land \Skip \lor {}
     & \ltype{out,out'}
\\&&\LS{\B{out}}
           \land \LS{\ell_g} \land \ado b \land \LUPD{\ell_g}{out}
           & \ltype{\B{out},\ell_g,\B{\ell'_g},out'}
\\&& ;
\\&& \LS{out} \land \Skip \lor {}  & \ltype{out,out'}
\\&& \LS{\B{out}}
       \land \LS{\ell_g} \land \ado b \land \LUPD{\ell_g}{out}
       & \ltype{\B{out},\ell_g,\B{\ell'_g},out'}
}
We see that the last two disjuncts of $S_;$ contradict the last disjunct
of $L_;$, so we get
\begin{eqnarray*}
  &&
     \LS{\ell_g} \land \Skip ; \LS{out} \land \Skip \lor{}
\\&& \LS{\B{\ell_g}}
           \land \LS{in} \land \ado a \land \LUPD{in}{\ell_g}
            ; \LS{out} \land \Skip\lor {}
\\&& \LS{out} \land \Skip ;\LS{out} \land \Skip\lor {}
\\&&\LS{\B{out}}
           \land \LS{\ell_g} \land \ado b \land \LUPD{\ell_g}{out}
           ; \LS{out} \land \Skip \lor {}
\\&&
     \LS{\ell_g} \land \Skip
     ; \LS{\B{out}}
       \land \LS{\ell_g} \land \ado b \land \LUPD{\ell_g}{out} \lor{}
\\&& \LS{\B{\ell_g}}
           \land \LS{in} \land \ado a \land \LUPD{in}{\ell_g}
            ; \LS{\B{out}}
       \land \LS{\ell_g} \land \ado b \land \LUPD{\ell_g}{out}
\EQ{with care\dots}
\\&&\LS{\ell_g} \land\LS{out} \land \Skip
\\&& \LS{\B{\ell_g}}
           \land \LS{in} \land \ado a \land \LUPD{in}{\ell_g}
            \land \LSd{out} \lor {}
\\&& \LS{out} \land \Skip  \lor {}
\\&&\LS{\B{out}}
           \land \LS{\ell_g} \land \ado b \land \LUPD{\ell_g}{out}
           \lor {}
\\&& \LS{\B{out}}
       \land \LS{\ell_g} \land \ado b \land \LUPD{\ell_g}{out} \lor{}
\\&& \LS{\B{\ell_g}} \land \LS{in} \land \LS{\B{out}}
    \land \ado{a;b}\land \LUPD{in}{out}
\EQ{\ldots tidy-up}
\\&& \LS{out} \land \Skip \lor {}
\\&& \LS{\B{\ell_g}}
           \land \LS{in} \land \ado a \land \LUPD{in}{\ell_g}
            \land \LSd{out} \lor {}
\\&& \LS{\B{out}}
       \land \LS{\ell_g} \land \ado b \land \LUPD{\ell_g}{out} \lor{}
\\&& \LS{\B{\ell_g}} \land \LS{in} \land \LS{\B{out}}
    \land \ado{a;b}\land \LUPD{in}{out}
\end{eqnarray*} Hmmm\dots

We look at $S_;^2 ; L_;$ now.
\RLEQNS{
   S^2_; ; L_; &=&
     \LS{\ell_g} \land \Skip \lor{}
      & \ltype{\ell_g,\ell'_g}
\\&& \LS{\B{\ell_g}}
           \land \LS{in} \land \ado a \land \LUPD{in}{\ell_g} \lor {}
      & \ltype{\B{\ell_g},in,\B{in'}, \ell'_g}
\\&& \LS{out} \land \Skip \lor {}
     & \ltype{out,out'}
\\&&\LS{\B{out}}
           \land \LS{\ell_g} \land \ado b \land \LUPD{\ell_g}{out}
           & \ltype{\B{out},\ell_g,\B{\ell'_g},out'}
\\&& ;
\\&& \LS{out} \land \Skip \lor {}
\\&& \LS{\B{\ell_g}}
           \land \LS{in} \land \ado a \land \LUPD{in}{\ell_g}
            \land \LSd{out} \lor {}
\\&& \LS{\B{out}}
       \land \LS{\ell_g} \land \ado b \land \LUPD{\ell_g}{out} \lor{}
\\&& \LS{\B{\ell_g}} \land \LS{in} \land \LS{\B{out}}
    \land \ado{a;b}\land \LUPD{in}{out}
}

\subsection{Notation, Notation}

We find essentially just two idioms here,
where $L$, $I$, $O$, $R$ and $A$ are lists of labels,
with $I \cap O = \emptyset$ and $R \cap A = \emptyset$:
\begin{eqnarray*}
   D(L) &\defs&  \LS{L} \land \Skip
\\ &=& ls(L) \land s'=s \land ls'=ls
\\ &=& ls(L) \land s'=s \land ls'=ls \land ls'(L)
\\ &=& \LS{L} \land \Skip \land \LSd{L}
\\ &:&  \ltype{L,L'}
\\ A(I,O,as,R,A,L)
   &\defs& \LS{I} \land \LS{\B O} 
           \land \ado{as} 
           \land \LUPD{R}{A} \land \LSd{L}
\\ &=& ls(I) \land ls(\B O) \land \ado{as}
       \land ls'=ls\ominus(R|A) \land ls'(L)
\\ &:& \ltype{I,\B O,\B{R'\setminus A'},L',A'}
\\ &=& I \cap O = \emptyset \land ls(I) \land ls(\B O)
\\ && {} \land \ado{as}
\\ && {} \land ls'=ls\ominus(R|A) \land ls'(L)
         \land (R\setminus A) \cap L = \emptyset
\end{eqnarray*}
The first and second lines above in each definition/expansion
are the so-called ``implicit'' forms,
in that we have a minimal complete description,
but without any explicit identification of situations that force
the predicate to evaluate to false.
The last version of $A$ above is the so-called ``explicit-form'',
which states the relationships on parameters $I$, $O$, $R$, $A$ and $L$
that most hold in order for the predicate not to be false everywhere.



We get the following laws (implicit form):
\begin{eqnarray*}
   D(L_1) ; D(L_2) &=& D(L_1 \cup L_2)
%
\\ D(L_1) ;  A(I,O,as,R,A,L_2)
   &=&
   A(L_1\cup I,O,as,R,A,L_2)
%
\\  A(I,O,as,R,A,L_1) ; D(L_2)
   &=&
   A(I,O,as,R,A,L_1\cup L_2)
%
\\ A(I_1,O_1,as,R_1,A_1,L_1) ; {}
\\ A(I_2,O_2,bs,R_2,A_2,L_2)
   &=& \dots
\end{eqnarray*}

\subsubsection{Proofs}

Full forms
\begin{eqnarray*}
   D(L) 
   &\defs& \LS{L} \land \Skip
\\ &  =  & ls(L) \land s'=s \land ls'=ls
\\
\\ A(I,O,as,R,A,L)
   &\defs& 
   \LS{I} \land \LS{\B O} \land \ado{as} \land \LUPD{R}{A} \land \LSd{L}
\\ &  =  & ls(I) \land ls(\B O) \land \ado{as} \land \lupd R A \land ls'(L)
\end{eqnarray*}

\begin{eqnarray*}
  && D(L_1) \seq D(L_2)
\EQ{Defn. $D$}
\\&& ls(L_1) \land s'=s \land ls'=ls \seq ls(L_2) \land s'=s \land ls'=ls
\EQ{Defn. $\seq$}
\\&& \exists s_m,ls_m \bullet
    ls(L_1) \land s_m=s \land ls_m=ls
    \land ls_m(L_2) \land s'=s_m \land ls'=ls_m
\EQ{One-point, $s_m,ls_m = s,ls$}
\\&& ls(L_1) \land ls(L_2) \land s'=s \land ls'=ls
\EQ{$A \subseteq S \land B \subseteq S = (A \cup B) \subseteq S$}
\\&& ls(L_1 \cup L_2) \land s'=s \land ls'=ls
\EQ{Defn. $D$, fold}
\\&& D(L_1 \cup L_2)
\end{eqnarray*}

\begin{eqnarray*}
  && D(L_1) \seq A(I,O,as,R,A,L_2)
\EQ{Defn. of $D$ and $A$.}
\\&& ls(L_1) \land s'=s \land ls'=ls
     \seq
     ls(I) \land ls(\B O) \land \ado{as} \land \lupd R A \land ls'(L_2)
\EQ{Defn. of $\seq$.}
\\&& \exists s_m,ls_m \bullet ls(L_1) \land s_m=s \land ls_m=ls
\\&& \qquad {} \land
     ls_m(I) \land ls_m(\B O) \land \ado{as}[s_m/s]
     \land ls'=ls_m \ominus (R|A) \land ls'(L_2)
\EQ{One-point, $s_m,ls_m = s,ls$}
\\&& ls(L_1) \land
     ls(I) \land ls(\B O) \land \ado{as}
     \land ls'=ls \ominus (R|A) \land ls'(L_2)
\EQ{$A \subseteq S \land B \subseteq S = (A \cup B) \subseteq S$}
\\&& ls(L_1 \cup I) \land ls(\B O) \land \ado{as}
     \land ls'=ls \ominus (R|A) \land ls'(L_2)
\EQ{Defn. $A$, fold}
\\&& A(L_1 \cup I,O,as,R,A,L_2)
\end{eqnarray*}

\begin{eqnarray*}
  && A(I,O,as,R,A,L_1) \seq D(L_2)
\EQ{Defn. of $A$ and $D$.}
\\&& ls(I) \land ls(\B O) \land \ado{as} \land \lupd R A \land ls'(L_1)
     \seq
     ls(L_2) \land s'=s \land ls'=ls
\EQ{Defn. $\seq$}
\\&& \exists s_m,ls_m \bullet
\\&& \quad
     ls(I) \land ls(\B O) \land \ado{as}[s_m/s'] 
     \land ls_m=ls \ominus (R|A) \land ls_m(L_1)
\\&& {} \land
     ls_m(L_2) \land s'=s_m \land ls'=ls_m
\EQ{One-point, $s_m,ls_m = s',ls'$}
\\&& ls(I) \land ls(\B O) \land \ado{as} 
\land ls'=ls \ominus (R|A) \land ls'(L_1)
     \land ls'(L_2)
\EQ{$A \subseteq S \land B \subseteq S = (A \cup B) \subseteq S$}
\\&& ls(I) \land ls(\B O) \land \ado{as} \land ls'=ls \ominus (R|A)
     \land ls'(L_1 \cup L_2)
\EQ{Defn. $A$, fold}
\\&& A(I,O,as,R,A,L_1 \cup L_2)
\end{eqnarray*}

\begin{eqnarray*}
  && A(I_1,O_1,as,R_1,A_1) \seq A(I_2,O_2,bs,R_2,A_2)
\EQ{Defn $A$, twice}
\\&& ls(I_1) \land ls(\B{O_1}) \land \ado{as} 
     \land \lupd{R_1}{A_1} \land ls'(L_1)
\\&& {} \seq
     ls(I_2) \land ls(\B{O_2}) \land \ado{bs} 
     \land \lupd{R_2}{A_2} \land ls'(L_2)
\EQ{Defn $\seq$.}
\\&& \exists s_m,ls_m \bullet
\\&& \qquad ls(I_1) \land ls(\B{O_1}) \land \ado{as}[s_m/s']
     \land ls_m = ls\ominus(R_1|A_1) \land ls_m(L_1)
\\&& \quad {} \land
     ls_m(I_2) \land ls_m(\B{O_2}) \land \ado{bs}[s_m/s]
     \land ls' = ls_m\ominus(R_2|A_2) \land ls'(L_2)
\EQ{$A \subseteq S \land B \subseteq S = (A \cup B) \subseteq S$, re-arrange}
\\&& \exists s_m,ls_m \bullet
\\&& \qquad ls(I_1) \land ls(\B{O_1}) \land ls_m = ls\ominus(R_1|A_1)
     \land ls_m(L_1 \cup I_2)
\\&& \quad {}\land ls_m(\B{O_2}) \land ls' = ls_m\ominus(R_2|A_2) \land ls'(L_2)
     \land \ado{as}[s_m/s'] \land \ado{bs}[s_m/s]
\EQ{One-point, $ls_m = ls\ominus(R_1|A_1)$}
\\&& \exists s_m \bullet ls(I_1) \land ls(\B{O_1})
     \land (ls\ominus(R_1|A_1))(L_1 \cup I_2)
\\&& \qquad {}\land (ls\ominus(R_1|A_1))(\B{O_2})
      \land ls' = (ls\ominus(R_1|A_1))\ominus(R_2|A_2) \land ls'(L_2)
\\&& \qquad {} \land \ado{as}[s_m/s'] \land \ado{bs}[s_m/s]
\EQ{Push quantifier in}
\\&& ls(I_1)  \land ls(\B{O_1})
     \land (ls\ominus(R_1|A_1))(L_1 \cup I_2)
\\&& {}\land (ls\ominus(R_1|A_1))(\B{O_2})
      \land ls' = (ls\ominus(R_1|A_1))\ominus(R_2|A_2) \land ls'(L_2)
\\&& {}\land \exists s_m \bullet \ado{as}[s_m/s'] \land \ado{bs}[s_m/s]
\EQ{Defn. of $\seq$ for atomic $as$,$bs$.}
\\&& ls(I_1)  \land ls(\B{O_1})
     \land (ls\ominus(R_1|A_1))(L_1 \cup I_2)
\\&& {}\land (ls\ominus(R_1|A_1))(\B{O_2})
      \land ls' = (ls\ominus(R_1|A_1))\ominus(R_2|A_2) \land ls'(L_2)
\\&& {}\land (\ado{as \seq bs})
\end{eqnarray*}
We now take stock, and seek to simplify some stuff.

\begin{eqnarray*}
  && (ls\ominus(R_1|A_1))(L_1 \cup I_2)
\\&=& (L_1\cup I_2) \subseteq (ls \setminus R_1) \cup A_1
\end{eqnarray*}

\begin{eqnarray*}
  && (ls\ominus(R_1|A_1))(\B{O_2})
\\&=& O_2 \cap ((ls \setminus R_1) \cup A_1) = \emptyset
\end{eqnarray*}

\begin{eqnarray*}
  && (ls\ominus(R_1|A_1))\ominus(R_2|A_2)
\\&=& (((ls \setminus R_1) \cup A_1) \setminus R_2) \cup A_1
\end{eqnarray*}

Hmmmm\dots  
We may need to assume explicit false-avoiding and minimalist conditions,
e.g. for
\[ 
   A(I,O,as,R,A,L) 
\]
the false avoiding conditions are
$I \cap O = \emptyset$ 
and $L \cap (R \setminus A) = \emptyset$,
while the minimalist condition is $R \cap A = \emptyset$.

The basic atomic action $a$ has semantics
\[
  D(out) \lor A(in,out,a,in,out,\emptyset)
\]


































