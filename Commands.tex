\section{Commands}

We do a quick run-down of the Commands\cite{conf/popl/Dinsdale-YoungBGPY13}.

\subsection{Syntax}

\begin{eqnarray*}
   a &\in& \Atom
\\ C &::=& a \mid \cskip \mid C \cseq C \mid C+C \mid C \parallel C \mid C^*
\\ g &:& Gen
\\ \ell &:& Lbl
\\ G &::=&  g \mid G_{:} \mid G_1 \mid G_2
\\ L &::=& \ell_G
\end{eqnarray*}

\subsection{Domains}
\begin{eqnarray*}
   s &\in& \mathcal S
\\ \sem{-} &:& \Atom \fun \mathcal S \fun \mathcal P(\mathcal S)
\\ S \ominus (T|V)
   &\defs& (S \setminus T) \cup V
\end{eqnarray*}

\subsection{Alphabet}

\begin{eqnarray*}
   s, s' &:& \mathcal S
\\ ls, ls' &:& \mathcal P (Lbl)
\\ g &:& Gen
\\ in, out &:& Lbl
\end{eqnarray*}
\subsection{``Standard'' UTP Constructs}

\begin{eqnarray*}
   P \cond c Q
   &\defs&
   c \land P \lor \lnot c \land Q
\\ P ; Q
   &\defs&
   \exists s_m,ls_m \bullet P[s_m,ls_m/s',ls'] \land Q[s_m,ls_m/s,ls]
\\ c * P
   &=&
   P ; c * P \cond c \Skip
\end{eqnarray*}

\subsection{Simple Semantics}

This follows the TASE2016 Submission style,
with a simpler version of $run$.

\begin{eqnarray*}
   a
   &\defs&
   ls(in) \land s' \in \sem a s \land ls'=ls \ominus (in|out)
\\ \cskip
   &\defs&
   ls(in) \land s' = s \land ls'=ls \ominus (in|out)
\\ C \cseq D
   &\defs&
   C[g_{:1},\ell_g/g,out] \lor D[g_{:2},\ell_g/g,in]
\\ C + D
   &\defs&
   ls(in)
   \land s' = s
   \land ls' \in \{ ls \ominus (in,\ell_{g2}|\ell_{g1})
                  , ls \ominus (in,\ell_{g1}|\ell_{g2})\}
\\ &\lor&
   C[g_{1:},\ell_{g1}/g,in] \lor D[g_{2:},\ell_{g2}/g,in]
\\ C \parallel D
   &\defs&
   ls(in)
   \land s' = s
   \land ls'= ls \ominus (in|\ell_{g1},\ell_{g2})
\\ &\lor&
   C[g_{1::},\ell_{g1},\ell_{g1:}/g,in,out]
   \lor D[g_{2::},\ell_{g2},\ell_{g2:}/g,in,out]
\\ &\lor&
   ls(\ell_{g1:},\ell_{g2:})
   \land s' = s
   \land ls'= ls \ominus (\ell_{g1:},\ell_{g2:}|out)
\\ C^*
   &\defs&
   ls(in)
   \land s' = s
   \land ls' \in \{ ls \ominus (in|out), ls \ominus (in|\ell_g)\}
\\ &\lor&
   C[g_{:},\ell_g,in/g,in,out]
\\ run(C)
   &\defs&
   ( \lnot ls(out) * C)[in/ls]
\end{eqnarray*}


\newpage
\subsection{Wheels-within-Wheels}

This is the new all-shiny fully compositional form%
\footnote{We hope !}%
.


\subsubsection{WwW Semantic Definitions}

We introduce the following shorthands:
\begin{eqnarray*}
   \ado a &\defs& \atomdo a
\\ \LUPD S T &\defs& \lupd S T
\\ \LS S &\defs& ls(S)
\\ \LSd S &\defs& ls'(S)
\end{eqnarray*}

The last three shorthands satisfy the following laws:
\begin{eqnarray*}
   \LUPD S T &=& \LUPD S T \land \LSd T
\\ P \land \LSd T ; Q
   &=&
   P ; \LS T \land Q
\\ P \land \LUPD S T ; Q
   &=&
   P \land \LUPD S T ; \LS T \land Q
\end{eqnarray*}

The definitions:
\begin{eqnarray*}
   \W(C)
   &\defs&
   (\lnot ls(out) * C)
\\ a
   &\defs&
   \W(ls(in) \land \ado a \land \lupd{in}{out})
\\ &=& \W(~\LS{in} \land \ado a \land \LUPD{in}{out}~)
\\ \cskip
   &\defs&
   \W(ls(in) \land \ss \land \lupd{in}{out})
\\ C \cseq D
   &\defs&
   \W(C[g_{:1},\ell_g/g,out] \lor D[g_{:2},\ell_g/g,in])
\\ C + D
   &\defs&
   \W(~ls(in)
   \land \ss
   \land ls' \in \{ \lexc{in,\ell_{g2}}{\ell_{g1}}
                  , \lexc{in,\ell_{g1}}{\ell_{g2}}\}
\\ &\lor&
   C[g_{1:},\ell_{g1}/g,in] \lor D[g_{2:},\ell_{g2}/g,in]~)
\\ C \parallel D
   &\defs&
   \W(~ls(in)
   \land \ss
   \land \lupd{in}{\ell_{g1},\ell_{g2}})
\\ &\lor&
   C[g_{1::},\ell_{g1},\ell_{g1:}/g,in,out]
   \lor D[g_{2::},\ell_{g2},\ell_{g2:}/g,in,out]
\\ &\lor&
   ls(\ell_{g1:},\ell_{g2:})
   \land \ss
   \land \lupd{\ell_{g1:},\ell_{g2:}}{out}~)
\\ C^*
   &\defs&
   \W(~ls(in)
   \land \ss
   \land ls' \in \{ \lexc{in}{out}), \lexc{in}{\ell_g}\}
\\ &\lor&
   C[g_{:},\ell_g,in/g,in,out]~)
\end{eqnarray*}

\subsubsection{WwW Calculations/Results}

We will start by explaining a calculation method
that should help structure our reasoning about loops.
We consider a generic iteration $c*P$,
and note the following identity,
obtained by repeated application of the loop unrolling law
coupled with expansion of the definition of conditionals:
\begin{equation}\label{eqn:unroll-n-times}
   c * P
   \quad=\quad
   \bigvee_{i=0}^{n-1} ( (c \land P)^i ; \lnot c \land \Skip)
   \;\lor\;
   (c \land P)^n ; c * P
\end{equation}
From this we define the following shorthands,
and suggest two important calcualtions:
\begin{eqnarray*}
   W &\defs& c * P
\\ D &\defs& \lnot c \land \Skip \EOLC{Done}
\\ S &\defs& c \land P \EOLC{Step}
\\ L &\defs& S ; D \EOLC{Last}
\\ T &\defs& D ; D \EOLC{Two-Step}
\end{eqnarray*}

We now illustrate with the definition of $a$,
using subscripts indicating that we are instantiating the above shorthands
w.r.t. the semantics of $a$:

\begin{eqnarray*}
   a &\defs& \W( \LS{in} \land \ado a \land \LUPD{in}{out} )
\\ &=& W_a
\\ D &\defs& \D \EOLC{Independent of $a$.}
\\ S_a &\defs& \lnot \LS{out}
               \land \LS{in} \land \ado a \land \LUPD{in}{out}
\\ L_a &\defs& S_a ; D
\\ &=& \lnot \LS{out}
               \land \LS{in} \land \ado a \land \LUPD{in}{out}
\\ && {} ; \LS{out} \land \Skip
\EQ{shorthand law, and $\Skip$ is $;$-identity}
\\ && \lnot \LS{out}
               \land \LS{in} \land \ado a \land \LUPD{in}{out}
\\ T_a &\defs& S_a ; S_a
\\ &=& \lnot \LS{out}
               \land \LS{in} \land \ado a \land \LUPD{in}{out}
\\ && {} ; \lnot \LS{out}
               \land \LS{in} \land \ado a \land \LUPD{in}{out}
\EQ{shorthand law}
\\ && \lnot \LS{out}
               \land \LS{in} \land \ado a \land \LUPD{in}{out}
\\ && {} ; \LS{out} \land \lnot \LS{out}
               \land \LS{in} \land \ado a \land \LUPD{in}{out}
\\ &=& false
\end{eqnarray*}
We see that all terms $S_a^i$ for $i \geq 2$ just vanish, so
\begin{eqnarray*}
   a &=& D \lor L_a
\\ &=& \D \;\lor\; \lnot \LS{out}
               \land \LS{in} \land \ado a \land \LUPD{in}{out}
\end{eqnarray*}


We overload sequential composition as follows:
\begin{eqnarray*}
   \ado{a;b} &=& \ado a ; \ado b
\end{eqnarray*}

