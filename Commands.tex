\section{Commands}

We do a quick run-down of the Commands\cite{conf/popl/Dinsdale-YoungBGPY13}.

\subsection{Syntax}

\begin{eqnarray*}
   a &\in& \Atom
\\ C &::=& a \mid \cskip \mid C \cseq C \mid C+C \mid C \parallel C \mid C^*
\\ g &:& Gen
\\ \ell &:& Lbl
\\ G &::=&  g \mid G_{:} \mid G_1 \mid G_2
\\ L &::=& \ell_G
\end{eqnarray*}

\subsection{Domains}
\begin{eqnarray*}
   s &\in& \mathcal S
\\ \sem{-} &:& \Atom \fun \mathcal S \fun \mathcal P(\mathcal S)
\\ S \ominus (T|V)
   &\defs& (S \setminus T) \cup V
\end{eqnarray*}

\subsection{Alphabet}

\begin{eqnarray*}
   s, s' &:& \mathcal S
\\ ls, ls' &:& \mathcal P (Lbl)
\\ g &:& Gen
\\ in, out &:& Lbl
\end{eqnarray*}

\subsection{``Standard'' UTP Constructs}

\begin{eqnarray*}
   P \cond c Q
   &\defs&
   c \land P \lor \lnot c \land Q
\\ P ; Q
   &\defs&
   \exists s_m,ls_m \bullet P[s_m,ls_m/s',ls'] \land Q[s_m,ls_m/s,ls]
\\ c * P
   &=&
   P ; c * P \cond c \Skip
\end{eqnarray*}

\subsection{Shorthands}

\begin{eqnarray*}
   ls(\ell) &\defs& \ell \in ls
\\ ls(L) &\defs& L \subseteq ls
\\ ls(\B\ell) &\defs& \ell \notin ls
\\ ls(\B L) &\defs& L \cap ls = \emptyset
\end{eqnarray*}

\subsection{Simple Semantics}

This follows the TASE2016 Submission style,
with a simpler version of $run$.

\begin{eqnarray*}
   a
   &\defs&
   ls(in) \land s' \in \sem a s \land ls'=ls \ominus (in|out)
\\ \cskip
   &\defs&
   ls(in) \land s' = s \land ls'=ls \ominus (in|out)
\\ C \cseq D
   &\defs&
   C[g_{:1},\ell_g/g,out] \lor D[g_{:2},\ell_g/g,in]
\\ C + D
   &\defs&
   ls(in)
   \land s' = s
   \land ls' \in \{ ls \ominus (in,\ell_{g2}|\ell_{g1})
                  , ls \ominus (in,\ell_{g1}|\ell_{g2})\}
\\ &\lor&
   C[g_{1:},\ell_{g1}/g,in] \lor D[g_{2:},\ell_{g2}/g,in]
\\ C \parallel D
   &\defs&
   ls(in)
   \land s' = s
   \land ls'= ls \ominus (in|\ell_{g1},\ell_{g2})
\\ &\lor&
   C[g_{1::},\ell_{g1},\ell_{g1:}/g,in,out]
   \lor D[g_{2::},\ell_{g2},\ell_{g2:}/g,in,out]
\\ &\lor&
   ls(\ell_{g1:},\ell_{g2:})
   \land s' = s
   \land ls'= ls \ominus (\ell_{g1:},\ell_{g2:}|out)
\\ C^*
   &\defs&
   ls(in)
   \land s' = s
   \land ls' \in \{ ls \ominus (in|out), ls \ominus (in|\ell_g)\}
\\ &\lor&
   C[g_{:},\ell_g,in/g,in,out]
\\ run(C)
   &\defs&
   ( ls(\B{out}) * C)[in/ls]
\end{eqnarray*}

\subsection{Operational Semantics}

As per \cite{conf/popl/Dinsdale-YoungBGPY13}.
\begin{mathpar}
\inferrule{C_1 \arr\alpha C'_1}{C_1 \cseq C_2 \arr\alpha C'_1 \cseq C_2}
\and
\inferrule{ }{\cskip \cseq C_2 \arr{ii} C_2}
\and
\inferrule{ }{C_1 + C_2 \arr{ii} C_i} i \in \setof{1,2}
\and
\inferrule{ }{C^* \arr{ii} C\cseq C^*}
\and
\inferrule{ }{C^* \arr{ii} \cskip}
\and
\inferrule{ }{\cskip \parallel C_2 \arr{ii} C_2}
\and
\inferrule{ }{C_1 \parallel \cskip \arr{ii} C_1}
\and
\inferrule{ }{a \arr{a} \cskip}
\and
\inferrule{C_1 \arr\alpha C'_1}{C_1 \parallel C_2 \arr\alpha C'_1 \parallel C_2}
\and
\inferrule{C_2 \arr\alpha C'_2}{C_1 \parallel C_2 \arr\alpha C_1 \parallel C'_2}
\\\\
\inferrule{ }{C,s \arrs C,s}
\and
\inferrule
 {C_1 \arr\alpha C_2
  \\
  s_2 \in \sem\alpha(s_1)
  \\
  C_2,s_2 \arrs C_3,s_3
 }
 {C_1,s_1 \arrs C_3,s_3}
\end{mathpar}

An interesting exercise is to try and give a meaning
to the transitions above in terms of our semantic model.

\subsection{Induced Transitions}

We shall write an atomic behavior
that is enabled only when $E \subset ls$,
does a state change described by predicate $a$ over $s$ and $s'$
and then removes all of $E$ from $ls$ and adds in the labels in $N$
with the suggestive notation:
\[
 E \arr a N
\]
Here we shall assume that both $E$ and $N$
contain only expressions over $in$, $out$ and $\ell_G$
where $G$ is a generator expression over $g$ only.
Written according to our semantics,
this corresponds to the predicate
\[
  ls(E) \land s' \in \sem a \land ls'=ls\ominus(E|N)
\]

If we define $ii$ as being  $s'=s$,
then we can re-write our semantics as follows:
\begin{eqnarray*}
   a
   &\defs&
   in \arr a out
\\ \cskip
   &\defs&
   in \arr{ii} out
\\ C \cseq D
   &\defs&
   C[g_{:1},\ell_g/g,out] \lor D[g_{:2},\ell_g/g,in]
\\ C + D
   &\defs&
   in \arr{ii} \ell_{g1} \lor in \arr{ii} \ell_{g2}
\\ &\lor&
   C[g_{1:},\ell_{g1}/g,in] \lor D[g_{2:},\ell_{g2}/g,in]
\\ C \parallel D
   &\defs&
   in \arr{ii} \setof{\ell_{g1},\ell_{g2}}
\\ &\lor&
   C[g_{1::},\ell_{g1},\ell_{g1:}/g,in,out]
   \lor D[g_{2::},\ell_{g2},\ell_{g2:}/g,in,out]
\\ &\lor&
   \setof{\ell_{g1:},\ell_{g2:}}
   \arr{ii}
   out
\\ C^*
   &\defs&
   in \arr{ii} out
   \lor
   in \arr{ii} \ell_g
\\ &\lor&
   C[g_{:},\ell_g,in/g,in,out]
\\ run(C)
   &\defs&
   ( ls(\B{out}) * C)[in/ls]
\end{eqnarray*}
We note the following substitution law:
\begin{eqnarray*}
  && (E \arr a N)[G,L,M/g,in,out]
\\&=& (ls(E) \land s' \in \sem a \land ls'=ls\ominus(E|N))[G,L,M/g,in,out]
\\&=& ls(E)[G,L,M/g,in,out]
       \land s' \in \sem a
\\&& {}\land ls'=ls\ominus(E[G,L,M/g,in,out]|N[G,L,M/g,in,out]))
\\&=& E[G,L,M/g,in,out] \arr a N[G,L,M/g,in,out]
\end{eqnarray*}
