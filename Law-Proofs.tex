\section{Law Proofs}\label{sec:law-proofs}

For now we take each law and expand lhs and rhs as far as is practical.


\subsection{Proof of Law ;;-l-unit}
\RLEQNS{
   \cskip \cseq P &=& P & \eref{;;-l-unit}
}
LHS:
\RLEQNS{
  && \cskip \cseq P
\EQ{\eref{sem:seq},\eref{nf-skip}}
\\&& [in|\ell_g|out]
     \land
     \W(~
         ([in|out] \land (\Skip \lor A(in|ii|out)))[g_{:1},\ell_g/g,out]
         \lor
         P[g_{:2},\ell_g/g,in]
     ~)
\EQ{Substitution, \eref{I-gamma-subst},\eref{W-gamma-subst}}
\\&& [in|\ell_g|out]
     \land
     \W(~
         ([in|\ell_g] \land (\Skip \lor A(in|ii|\ell_g)))
         \lor
         P[g_{:2},\ell_g/g,in]
     ~)
\EQ{$\land$-$\lor$-distr, \eref{I-W-distr}}
\\&&\W(~
         [in|\ell_g|out] \land [in|\ell_g] \land \Skip
\\&&\qquad \lor {}
         [in|\ell_g|out] \land [in|\ell_g] \land A(in|ii|\ell_g)
\\&&\qquad \lor {}
         [in|\ell_g|out] \land P[g_{:2},\ell_g/g,in]
     ~)
\EQ{\eref{I-subsumption}, shorthand $\gamma=[g_{:2},\ell_g/g,in]$}
\\&&\W(~
         [in|\ell_g|out] \land \Skip
\\&&\qquad \lor {}
         [in|\ell_g|out]  \land A(in|ii|\ell_g)
\\&&\qquad \lor {}
         [in|\ell_g|out] \land P\gamma
     ~)
\EQ{$\land$-$\lor$-distr, \eref{I-W-distr}}
\\&&[in|\ell_g|out]
     \land
     \W(~
         \Skip
         \lor
         A(in|ii|\ell_g)
         \lor
         P\gamma
     ~)
\EQ{\eref{W-as-NDC}, pull $\Skip$ out}
\\&&[in|\ell_g|out]
     \land
     (\Skip
       \lor \bigvee_{i \in 1,\dots}(~
         A(in|ii|\ell_g)
         \lor
         P\gamma~)^i
     )
\EQ{\eref{UTCP-NF}, shorthand form,substitution}
\\&&[in|\ell_g|out]
     \land
     (\Skip
       \lor \bigvee_{i \in 1,\dots}(~
         A(in|ii|\ell_g)
         \lor
         (\bigvee_p X_p\gamma)~)^i
     )
}
Now we break out to explore the sequencing, starting with $i=2$,
and noting that the invariant is $[in|\ell_g|out]$
\RLEQNS{
  && (A(in|ii|\ell_g) \lor (\bigvee_p X_p\gamma))^2
\EQ{\eref{NF+NF-squared}}
\\    && A(in|ii|\ell_g)^2
\\&\lor&  \bigvee_{p_1,p_2}(X_{p_1}\gamma \seq X_{p_2}\gamma)
\\&\lor&  \bigvee_p(A(in|ii|\ell_g) \seq X_p)
\\&\lor&  \bigvee_p(X_p\gamma \seq A(in|ii|\ell_g))
\EQ{$A^2$ vanishes (easy exercise)}
\\    &&  \bigvee_{p_1,p_2}(X_{p_1}\gamma \seq X_{p_2}\gamma)
\\&\lor&  \bigvee_p(A(in|ii|\ell_g) \seq X_p\gamma)
\\&\lor&  \bigvee_p(X_p\gamma \seq A(in|ii|\ell_g))
\EQ{$X\seq A$ vanishes (Lemma 1 below)}
\\    &&  \bigvee_{p_1,p_2}(X_{p_1}\gamma \seq X_{p_2}\gamma)
\\&\lor&  \bigvee_p(A(in|ii|\ell_g) \seq X_p\gamma)
\EQ{Lemma 2 below}
\\    &&  \bigvee_{p_1,p_2}(X_{p_1}\gamma \seq X_{p_2}\gamma)
\\&\lor&  \bigvee_p X_p, \textbf{ where } E_p = in
\EQ{\eref{NF-squared}}
\\&& (\bigvee_p X_p)^2 \lor \bigvee_r X_r, \textbf{ where } E_r = in
}

\subsubsection{Lemma 1}

Actions in $X_p\gamma$ are enabled by labels in $\ell_g,\g{:2}$,
and the adding new labels in $\g{:2},out$.
The top-level Disjoint Labels invariant says $[in|g|out]$,
so $in$ cannot be in $ls$ when any $X_p\gamma$ action is enabled.
So once it has executed it will not add in $in$ so the $A$ action
will not be enabled. These mumbling steps cannot happen.

We need a lemma here about the interaction of an instance of $\cskip$
with the $A$ terms of $P$.


\subsubsection{Lemma 2}

Action $A$ is enabled by $in$, when all actions in $X_p\gamma$
are disabled, as a result of the invariant.
It enables actions that require $\ell_g$
and disables those with other enablers.
So we only get mumblings with $X$-terms of the form:
\[
  X(\ell_g,\ell_G,...|a|R\gamma|A\gamma)
\]
where $G \subset \g{:2}$.
Let us calculate:
\RLEQNS{
  && A(in|ii|\ell_g) \seq X(\ell_g,\ell_G,...|a|R\gamma|A\gamma)
\EQ{Disjoint Label invariant under $\gamma$ is $[\ell_g|\g{:2}|out]$}
\\&& A(in|ii|\ell_g) \seq X(\ell_g|a|R\gamma|A\gamma)
\EQ{\eref{A-then-X}}
\\&& X(in\cup (\ell_g\setminus \ell_g)
      | ii\seq a
      | \ell_g,R\gamma
      | (\ell_g \setminus \ell_g)\cup A\gamma)
\\&& {} \land (\ell_g\setminus \ell_g) \cap in = \emptyset
\EQ{Simplify}
\\&& X(in
      | a
      | \ell_g,R\gamma
      | A\gamma)
\EQ{$\gamma=[\g{:2},\ell_g/g,in]$, invariant means $ls$ cannot contain $\ell_g$ when $in$ is present}
\\&& X(in
      | a
      | R\gamma
      | A\gamma)
\EQ{invariant means $\ell_g \notin R\gamma$, so $in \notin R$.}
\\&& X(in
      | a
      | R
      | A\gamma)
\EQ{Program $P$ does not add in $in$}
\\&& X(in
      | a
      | R
      | A)
}
So we have demonstrated that
\[
 A(in|ii|\ell_g) \seq X(in,\ell_g,...|a|R|A)\gamma
 = X(in|a|R|A)
\]
\subsection{Proof of Law ;;-r-unit}
\RLEQNS{
   P \cseq \cskip &=& P & \eref{;;-r-unit}
}
LHS:
\RLEQNS{
  && P \cseq \cskip
\EQ{careful look at LHS calculation for \ecite{;;-l-unit} above.}
\\&&[in|\ell_g|out]
     \land
     (\Skip \lor
        \bigvee_{i \in 0,\dots}(~
          P[g_{:1},\ell_g/g,out]
         \lor
         A(\ell_g|ii|out)~)^i
     )
}

\subsection{Proof of Law XXX}
\RLEQNS{
   P \cseq ( Q \cseq R) &=& ( P \cseq Q ) \cseq R & \eref{;;-assoc}
}

\subsection{Proof of Law XXX}
\RLEQNS{
   P \parallel Q &=& Q \parallel P & \eref{||-comm}
}

\subsection{Proof of Law XXX}
\RLEQNS{
   P \parallel ( Q \parallel R) &=& ( P \parallel Q ) \parallel R & \eref{||-assoc}
}

\subsection{Proof of Law XXX}
\RLEQNS{
   P + Q &=& Q+P
}

\subsection{Proof of Law XXX}
\RLEQNS{
   P + ( Q + R) &=& ( P + Q ) + R
}

\subsection{Proof of Law XXX}
\RLEQNS{
   P + Q &\refinedby& P
}

\subsection{Proof of Law XXX}
\RLEQNS{
   P \sqsubseteq Q &\equiv& (P + Q) = Q
}

\subsection{Proof of Law XXX}
\RLEQNS{
   P^* &=& \cskip + P \cseq P^*
}

\subsection{Proof of Law XXX}
\RLEQNS{
   P \dcond{true} Q &=& P & \eref{cond-true}
}

\subsection{Proof of Law XXX}
\RLEQNS{
   P \dcond{false} Q &=& Q & \eref{cond-false}
}

\subsection{Proof of Law XXX}
\RLEQNS{
   (P_1 \dcond c P_2) \cseq P_3 &=& (P_1 \cseq P_3) \dcond c (P_2 \cseq P_3)
}

\subsection{Proof of Law XXX}
\RLEQNS{
   c \ddo P &=& (P \cseq c \ddo P) \dcond c \cskip & \eref{wdo-unroll}
}

\subsection{Proof of Law XXX}
\RLEQNS{
   P_1 \cseq (P_2 \parallel P_3) &\sqsubseteq& (P_1 \cseq P_2) \parallel P_3
}

\subsection{Proof of Law XXX}
\RLEQNS{
   P_1 \cseq P_2 &\sqsubseteq& P_1 \parallel P_2
}

\subsection{Proof of Law XXX}
\RLEQNS{
   P_1 \dcond{c_1 \land c_2} P_2 &\sqsubseteq& (P_1 \dcond{c_2} P_2) \dcond{c_1} P_2
}
