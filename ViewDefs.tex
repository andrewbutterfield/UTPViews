\section{View Definitions}\label{sec:viewdefs}

We list all the generic view definitions\cite{conf/popl/Dinsdale-YoungBGPY13}.


\subsection{Universal Definitions}

\setcounter{secnumdepth}{0}

\subsubsection{Parameter A}(Atomic Commands)
Assume a set of (syntactic)
\emph{atomic commands} \Atom{}, ranged over by $a$.

\subsubsection{Definition 1} (Language Syntax)
The set of (syntactic) \emph{commands},
\Comm, ranged over by $C$, is defined by the following grammar:
\begin{eqnarray*}
   C &::=& a \mid \Vskip \mid C ; C \mid C+C \mid C \parallel C \mid C^*
\end{eqnarray*}

\subsubsection{Parameter B}(Machine States)
Assume a set of \emph{machine states} $\SS{}$,
ranged over by $s$.

\subsubsection{Parameter C}(Interpretation of Atomic Commands)
Assume a function $\sem{-} : \Atom{} \fun \SS{} \fun \power(\SS{})$
that associates each atomic
command with a non-deterministic state transformer. (Where necessary,
we lift non-deterministic state transformers to sets of states:
for $S \in \power(\SS{}), \sem a(A) = \bigcup\setof{\sem a(a) \mid s \in S}$.)


\subsubsection{Definition 2} (Transition Labels)
The set of transition labels, $\Label \vwdef \Atom{} \uplus \setof\vid$,
extends the set of atomic commands with a
designated identity label, \vid.
Labels are ranged over by $\alpha$.
$\sem{-}$ is extended to \Label\ by defining $\sem\vid \vwdef \lambda s .\setof s$.

\subsubsection{Definition 3}
(Labelled Transition System and Operational Semantics).
The labelled transition relation
$\_ \arr{\_} \_ : \Comm \times \Label \times \Comm$
is defined by the following rules:
\begin{mathpar}
\inferrule{C_1 \arr\alpha C'_1}{C_1 \seq C_2 \arr\alpha C'_1 \seq C_2}
\and
\inferrule{ }{\Vskip \seq C_2 \arr{ii} C_2}
\and
\inferrule{ }{C_1 + C_2 \arr{ii} C_i} i \in \setof{1,2}
\and
\inferrule{ }{C^* \arr{ii} C\cseq C^*}
\and
\inferrule{ }{C^* \arr{ii} \Vskip}
\and
\inferrule{ }{\Vskip \parallel C_2 \arr{ii} C_2}
\and
\inferrule{ }{C_1 \parallel \Vskip \arr{ii} C_1}
\and
\inferrule{ }{a \arr{a} \Vskip}
\and
\inferrule{C_1 \arr\alpha C'_1}{C_1 \parallel C_2 \arr\alpha C'_1 \parallel C_2}
\and
\inferrule{C_2 \arr\alpha C'_2}{C_1 \parallel C_2 \arr\alpha C_1 \parallel C'_2}
\end{mathpar}
The multi-step operational transition relation
$\_,\_ \arrs \_,\_ : (\Comm\times\SS{}) \times (\Comm\times\SS{})$
is defined by the following rules:
\begin{mathpar}
\inferrule{ }{C,s \arrs C,s}
\and
\inferrule
 {C_1 \arr\alpha C_2
  \\
  s_2 \in \sem\alpha(s_1)
  \\
  C_2,s_2 \arrs C_3,s_3
 }
 {C_1,s_1 \arrs C_3,s_3}
\end{mathpar}

An interesting exercise is to try and give a meaning
to the transitions above in terms of our semantic model.

\subsubsection{Parameter D} (View Commutative Semigroup).
Assume a commutative
semigroup $(\View,*)$. The variables $p$, $q$, $r$ are used to denote
elements of \View.

If the semigroup has a unit, $u$, then we call it a \emph{view monoid},
written $(\View,*,u)$.

\subsubsection{Parameter E} (Axiomatisation).
Assume a set of axioms $\Axiom \subseteq \View \times \Label \times \View$.

\subsubsection{Definition 7}
(Entailment).
The entailment relation
$\models \subseteq \View \times \View$
is defined by:
\[
  p\models q \eqvdef (p,\vid,q) \in \Axiom
\]

\subsubsection{Definition 8} (Program Logic).
The program logic’s judgements
are of the form
$\plogic p C q$,
where $p, q \in \View$ provide the
precondition and postcondition of command $C \in \Comm$. The
proof rules for these judgements are as follows:
\begin{mathpar}
\inferrule
  {(p,a,q) \in \Axiom}
     {\plogic p a q}
\and \inferrule
    {\plogic p C q}
  {\plogic{p*r}C{q*r}}
\and \inferrule
          { }
  {\plogic p \Vskip p}
\and \inferrule
  {\plogic p {C_1} q \\ \plogic p {C_2} q}
        {\plogic p {C_1 + C_2} q}
\and \inferrule
    {\plogic p C p}
  {\plogic p {C^*} P}
\and \inferrule
   { \plogic p {C_1} r \\ \plogic r {C_2} q }
        {\plogic p {C_1 ; C_2} q}
\and \inferrule
  {\plogic{p_1}{C_1}{q_1} \\ \plogic{p_2}{C_2}{q_2}}
  {\plogic{p_1 * p_2}{C_1\parallel C_2}{q_1 * q_2}}
\and \inferrule
  {p \models p' \\ \plogic{p'} C q}
          {\plogic p C q}
\and \inferrule
  {\plogic p C {q'} \\ q' \models q}
          {\plogic p C q}
\end{mathpar}

\subsubsection{Parameter F} (Reification).
Assume a reification function
$\reify{-} : \View \fun \power(\SS{})$
which maps views to sets of machine states.

\subsubsection{Definition 11} (Action Judgement).
The \emph{action judgement} is defined
as:
\begin{eqnarray*}
   \ajudge \alpha p q
   &\eqvdef&
   \sem\alpha(\reify p) \subseteq \reify q
   \land
   \forall r \in \View . \sem\alpha(\reify{p*r}) \subseteq \reify{q*r}
\end{eqnarray*}
For a view monoid, this definition simplifies to:
\begin{eqnarray*}
   \ajudge \alpha p q
   &\eqvdef&
   \forall r \in \View . \sem\alpha(\reify{p*r}) \subseteq \reify{q*r}
\end{eqnarray*}

\subsubsection{Definition 12} (Semantic Entailment)
$p \entails q  \eqvdef \ajudge \vid p q$.

\subsubsection{Property G} (Axiom Soundness).
For every $(p,\alpha,q) \in \Axiom)$
\[
  \ajudge \alpha p q
\]

\setcounter{secnumdepth}{4}

\subsection{Specialized Definitions}

The definitions are specialisations
specific to groups of similar instances of views.

\setcounter{secnumdepth}{0}

\subsubsection{More To Come \ldots}

\setcounter{secnumdepth}{4}
