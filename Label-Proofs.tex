\section{Label Proofs}\label{sec:label-proofs}

\subsection{Proof of labs-fully-disjoint}

\[
  \setof{\ell_G}
  \quad
  labs(G_{:})
  \quad
  labs(G_1)
  \quad
  labs(G_2)
  \qquad
  \eref{labs-fully-disjoint}
\]

TBD

\subsection{Proof of prefix-lab-subset}

\RLEQNS{
       labs(\g{\rho}) \subseteq labs(\g{\varrho}) &\equiv& \varrho \leq \rho
       & \eref{prefix-lab-subset}
    }

TBD

\subsection{Exclusivity Satisfaction}\label{ssec:exclusive-lsat}


We start by defining a general abstract way of specifying
sets with valid combinations
of values drawn from a parameter type $\tau$.
\RLEQNS{
   i \in I_\tau &::=& \tau   &  \mbox{can contain this value}
\\ &\mid& \otimes(i,\ldots,i) & \mbox{at most one of these allowed to contain}
\\ &\mid& \cup (i,\ldots,i) & \mbox{any of these allowed to contain}
}

In effect we identify the occurrences of labels within this $I$-structure,
and then check the multiplicity constraints:
\RLEQNS{
   L \textbf{ lsat } I_{Lbl} &\defs& res=ok
\\ \textbf{where} && (res,\_) = occChk (occ_L I_{Lbl})
}
The $occ$ function takes a set $L$ of $\tau$ and an $I_\tau$ and returns a $I_\Bool$
that records if the corresponding element of $\tau$ is present in $L$.
\RLEQNS{
   occ &:& \power \tau \fun I_\tau \fun I_\Bool
\\ occ_L~\ell &\defs& \ell \in L
\\ occ_L~\otimes(i_1,\ldots,i_n)
   &\defs&
   \otimes(occ_L~i_1,\ldots,occ_L~i_n)
\\ occ_L~\cup(i_1,\ldots,i_n)
   &\defs&
   \cup(occ_L~i_1,\ldots,occ_L~i_n)
}
The $occChk$ function pattern matches across the boolean values to see if
constraints are satisfied.
The first component of the result is an overall ok/fail indicator,
while the second boolean component indicates if values are present
in any component.
\RLEQNS{
   occChk &:& I_\Bool \fun (\setof{ok,fail}\times \Bool)
\\ occChk(b) &\defs& (ok,b)
\\ occChk(\cup(i_1,\ldots,i_n))
   &\defs&
   (fail,\_),
   \textbf{ if }\exists j @ occChk(i_j) = (fail,\_)
\\ && (ok,b_1 \lor \dots \lor b_n),
   \textbf{ if }\forall j @ occChk(i_j) = (ok,b_j)
\\ occChk(\otimes(i_1,\ldots,i_n))
   &\defs&
   (fail,\_),
   \textbf{ if }\exists j @ occChk(i_j) = (fail,\_)
\\&& (fail,\_) \mbox{ if more than one $(ok,true)$}
\\&& (ok,false) \mbox{ if all are $(ok,false)$}
\\&& (ok,true) \mbox{ if  exactly one $(ok,true)$}
}
We note, as a consequence of the above definitions, that
\RLEQNS{
  \emptyset &  \textbf{lsat} &  I & \elabel{emp-lsat-I}
}
for any label-set invariant $I$.

We introduce a shorthand for invariants illustrated as follows.
\RLEQNS{
  ~ [ a,b,c | d,e | f ]
   &=&
   ls \textbf{ lsat } \otimes(\cup(a,b,c),\cup(d,e),f)
\\ ~[ a | (b|c),(d|e) | f ]
  &=&
  ls \textbf{ lsat } \otimes(a,\cup(\otimes(b,c),\otimes(d,e)),f)
}
In effect we make the involvement of $ls$ implicit,
and use bar ($|$) and comma ($,$) to replace $\otimes$ and $\cup$ respectively.
We also have a shorthand that just denotes
the non-intersecting nature of the arguments.
E.g.,
$\setof{A|B|C}$ asserts that $A$, $B$ and $C$ are mutually disjoint,
without any reference to $ls$ or any other set.


\subsection{Proof of DL-perm}

TBD

\subsection{Proof of LE-perm}

TBD

\subsection{Proof of DL-drop}

TBD

\subsection{Proof of LE-drop}

TBD

\subsection{Proof of DL-subset}

TBD

\subsection{Proof of DL-subset}

TBD

\subsection{Proof of LE-out-of-scope}

TBD

\subsection{Proof of LE-trivial}

TBD

\subsection{Proof of ???}

TBD
