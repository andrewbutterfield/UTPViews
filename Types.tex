\section{Types}\label{sec:types}

\subsection{Types as a View}

\begin{eqnarray*}
  \tau \in \Type &=& \valt \mid \reft \tau
\\ \Gamma &:& \Var \pfun \Type
\end{eqnarray*}

\textbf{Definition 9} (Simple Type Views) : \textbf{Parameter D}.

View monoid: $((\Var \pfun \Type)\uplus\setof\bot,\cup_\bot,\emptyset)$

\textbf{Definition 10} (Simple Type Axioms) : \textbf{Parameter E}.

Type axioms:
\begin{mathpar}
\vx : \tau, \vy : \tau \vdash \vx := \vy
\and
x: \reft \valt \vdash [x]:=v
\\
\vx:\tau, \vy:\reft\tau \vdash \vx := [\vy]
\and
\vx : \reft\tau, \vy:\tau \vdash \vx := \reft \vy
\\
\vx : \reft\tau, y:\tau \vdash [\vx] := \vy
\end{mathpar}

\def\op{\mathbin{\mathsf{op}}}
Type inference
\begin{mathpar}
\inferrule{ }
 {\Gamma \vdash \cskip}
\and
\inferrule
{\Gamma \vdash C_1 \and \Gamma \vdash C_2 \and \op \in \setof{\cseq,+} }
               {\Gamma \vdash C_1 \op C_2}
\\
\inferrule{\Gamma_1 \vdash C_1 \and \Gamma_2 \vdash C_2}
           {\Gamma_1,\Gamma_2 \vdash C_1 \parallel C_2}
\and
\inferrule{\Gamma \vdash C}
         {\Gamma \vdash C^*}
\and
\inferrule{\Gamma \vdash C}
    {\Gamma,\Gamma' \vdash C}
\end{mathpar}

\textbf{Definition 13} (State Typing).

The state typing judgement $\Gamma;\Theta \vdash s$,
where $\Gamma : \Var \pfun \Type$, $s \in \SS{H}$
and $\Theta : \Loc \pfun \Type$ ranges
over heap typing contexts, is defined as follows:
\begin{eqnarray*}
   \Gamma;\Theta \vdash s
   &\eqvdef&
   \forall \vx \in \dom(\Gamma) . \Theta \vdash s(\vx) : \Gamma(\vx)
\\ && {} \land \forall l \in \dom(\Theta) . \Theta \vdash s(l) : \Theta(l)
\end{eqnarray*}
where $\Theta \vdash v : \tau \eqvdef \tau=\valt \lor \tau=\reft (\Theta(v))$.


\textbf{Definition 14} (Simple Type Reification) : \textbf{Parameter F}.

The simple type reification,
$\reify{-}_{\TS} : \View_{\mathcal M \TS} \fun \power(\SS{H})$,
is defined as follows:
\[
  \reify{\Gamma}_{\TS}
  \vwdef \setof{ s \in \SS H \mid \exists \Theta . \Gamma;\Theta \vdash s}
\]

\subsection{View Types in UTP}

We first note that types are defined over the same heap model
used for atomic heaps.

We first elaborate the view a little.
\begin{eqnarray*}
   \View_{TS} &\defs& (\Var \pfun \Type)\uplus\setof\bot
\\ \bot \uts \Gamma &\defs& \bot
\\ \Gamma \uts \bot &\defs& \bot
\\ \Gamma \uts \Theta
     &\defs&
       \Gamma \cup \Theta
       \cond{~\Gamma \sim \Theta~}
       \bot
\\ \Gamma \sim \Theta &\defs& (\dom(\Gamma)\lhd \Theta) = (\dom(\Theta)\lhd\Gamma)
\end{eqnarray*}
Here we define a shorthand $\sim$ that asserts that two type contexts
are compatible

The type assertions $\Gamma \vdash C$
are shorthand for $\plogic \Gamma C \Gamma$,
and are effectively UTP conditions.
We can still express them as pre/post relations as shown
by the following two versions of the first type axiom:

\begin{eqnarray*}
   \vx : \tau, \vy : \tau &\vdash& \vx := \vy
\\ \vx := \vy &\defs& \Gamma = \setof{\vx : \tau, \vy : \tau} \land \Gamma'=\Gamma
\end{eqnarray*}
Actually, this proves to be too strong.
In effect we need to build in the weakening inference rule
from \textbf{Definition 10} here, by replacing equality with a form of
superset relation:
\begin{eqnarray*}
   \Gamma \supts \bot &\defs& \false
\\ \bot \supts \Theta &\defs& \false
\\ \Gamma \supts \Theta
   &\defs&
   \dom(\Theta)\subseteq \dom(\Gamma)
   \land
   \forall v \in \dom(\Theta) \bullet \Theta(v) = \Gamma(v)
\end{eqnarray*}
We can use a healthiness condition here
(which effectively asserts that we are doing ``static'' analysis).
\begin{eqnarray*}
  \HTS(P) &\defs& P \land \Gamma'=\Gamma
  \\ \vx := \vy     &\defs& \HTS(\Gamma \supts \setof{\vx : \tau, \vy : \tau})
\\ ~[x]:=v          &\defs& \HTS(\Gamma \supts \setof{x: \reft \valt})
\\ \vx := [\vy]     &\defs& \HTS(\Gamma \supts \setof{\vx:\tau, \vy:\reft\tau})
\\ \vx := \reft \vy &\defs& \HTS(\Gamma \supts \setof{\vx : \reft\tau, \vy:\tau})
\\ ~[\vx] := \vy    &\defs& \HTS(\Gamma \supts \setof{\vx : \reft\tau, y:\tau})
\end{eqnarray*}

Now, how does the semantics of commands play out here?
As we observed in the Commands section (\S\ref{sec:commands}),
the outcomes of computing $run(C)$
are of the form
\[
  atomicExpr \land ls'=out
\]
where $atomicExpr$ is an expression built up from atomic actions
$a$ and $b$, using only disjunction and sequential composition
over the shared state only.

All our atomic actions in this type theory
are of the form $t \land \Gamma'=\Gamma$,
where $t$ is a condition, a predicate over $\Gamma$ only.

So we can calculate the sequential composition of two type `actions'
as follows:
\begin{eqnarray*}
  && (t_1 \land \Gamma'=\Gamma) \seq (t_2 \land \Gamma'=\Gamma)
\EQ{Defn of $\seq$}
\\&& \exists \Gamma_m \bullet
     t_1[\Gamma_m/\Gamma'] \land \Gamma_m=\Gamma
     \land
     t_2[\Gamma_m/\Gamma] \land \Gamma'=\Gamma_m
\EQ{$\Gamma'$ not in $t_1$, One-point Rule, $\Gamma_m=\Gamma$}
\\&& t_1
     \land
     t_2[\Gamma_m/\Gamma][\Gamma/\Gamma_m] \land \Gamma'=\Gamma
\EQ{Tidy-up}
\\&& t_1 \land t_2 \land \Gamma'=\Gamma
\end{eqnarray*}
So, sequential composition turns into a conjunction of the
relevant type context assertions.
In terms of our healthiness condition:
\[
  \HTS(t_1) \seq \HTS(t_2)  = \HTS(t_1 \land t_2)
\]
It is also worth doing the following calculation,
where neither of $\Gamma_i$ are $\bot$:
\begin{eqnarray*}
  && \HTS(\Gamma \supts \Gamma_1 \uts \Gamma_2)
\EQ{Defn $\uts$}
\\&& \HTS(~\Gamma \supts ( \Gamma_1 \cup \Gamma_2
                   \cond{\Gamma_1\sim\Gamma_2}
                   \bot)~)
\EQ{lift conditional}
\\&& \HTS(~(\Gamma \supts \Gamma_1 \cup \Gamma_2)
            \cond{\Gamma_1\sim\Gamma_2}
            (\Gamma \supts \bot))
\EQ{else-case is false, defn of $\cond{-}$}
\\&& \HTS(~(\Gamma \supts \Gamma_1 \cup \Gamma_2)
            \land (\Gamma_1\sim\Gamma_2))
\EQ{pull conjunct out, noting it ensures $\supts$ is like map-$\supseteq$}
\\&& \HTS(\Gamma \supseteq ( \Gamma_1 \cup \Gamma_2))
            \land \Gamma_1\sim\Gamma_2
\EQ{superset law}
\\&& \HTS(\Gamma \supseteq \Gamma_1 \land \Gamma \supseteq \Gamma_2))
            \land \Gamma_1\sim\Gamma_2
\EQ{$\HTS$ and conjunction distribute}
\\&& \HTS(\Gamma \supseteq \Gamma_1)
     \land \HTS(\Gamma \supseteq \Gamma_2)
            \land \Gamma_1\sim\Gamma_2
\end{eqnarray*}
So, if the type contexts are incompatible,
both the following, from the command semantics calculations,
\[
  \HTS(\Gamma\supts\Gamma_1) \seq \HTS(\Gamma\supts\Gamma_2)
\]
and this, capturing a common case in the type inference rules,
\[
  \HTS(\Gamma\supts(\Gamma_1\uts\Gamma_2))
\]
reduce to $\false$.
If they are compatible, they reduce to the assertion
\[
  \HTS(\Gamma\supts(\Gamma_1\cup\Gamma_2)).
\]
