\section{Labels}\label{sec:labels}

\subsection{Overview}


Concurrent programs are a mix of control constructs ($\cseq,\parallel,+,\dots$)
and atomic actions $a$ that modify \emph{program state} $s$.
This is all that is visible to the programmer.



Our denotational UTP semantics
is inspired by that done for UTPP\cite{DBLP:conf/icfem/WoodcockH02},
itself based on a UTP theory of \emph{action systems}.
First, we add a new state component, not visible in the program text,
which is a set of labels ($ls$).
These labels are used to orchestrate the flow of control.
Given an atomic action $a$ that only knows about program state $s$,
we use notation $\catom a$ to describe a lifted version
of $a$ that takes account of the contents of the label-set $ls$.
In effect a lifted atomic state action is disabled until
a particular label is present in that set.
A disabled action does not modify the state any way
and simply monitors the label-set.
Once an action observes the presence/arrival of its label,
it then becomes active.
An active lifted action $\catom a$ atomically modifies the extended state as follows:
\begin{itemize}
  \item It makes the changes to $s$ as determined by its underlying action $a$.
  \item It removes the enabling label from the label-set.
  \item It adds in appropriate ``continuation'' labels to $ls$.
\end{itemize}
In \cite{DBLP:conf/icfem/WoodcockH02},
the language syntax required explicit (starting) labels
for every atomic state change action,
and these are used as lifted action enablers.
The continuation labels were taken to the the enabling labels
of whatever other actions occurred immediately after, in the program.
This means that the resulting semantics is not compositional.

The approach we adopt is to adopt the notion of \emph{label generators}
to create the labels, so ensuring they are unique,
but in such a way as make the generation a compositional process.
In addition we explicitly associate two labels with every program construct:
one that marks the entry into the construct,
and another that indicates the exit.

\subsubsection{On Notation}

We make liberal use of a number of shorthand notations,
largely to reduce expression size and clutter.
Typically these reduce the use of infix operators,
and in one notable case, collapse large conjunctions
of implications into a simple form with string visual cues.
Each shorthand is introduced by a subsection similar to this one.

\subsection{Label Generation}

INTRODUCE

\RLEQNS{
   g &:& Gen = \setof g  & \mbox{only one variable!}
\\ \ell &:& Gen \fun Lbl
\\ G \in Gen &::=&  g     & \mbox{the ``root'' generator}
\\           &\mid& G_{:} & \mbox{resulting generator after label produced}
\\           &\mid& G_1   & \mbox{first generator after generator split}
\\           &\mid& G_2   & \mbox{second generator after generator split}
\\ L \in Lbl &::=& \ell_G & \mbox{label produced by generator}
}
Note that our generator expression language has only one variable, $g$,
which denotes the ``root'' generator.
All generator expressions have the form $g$ followed by zero
or more uses of $:$, $1$ and $2$.
We say the  ``root'' generator because this notion is local to a
particular context, and we can relativise things by replacing $g$
by an arbitrary $G$ expression.
We define the notion of the set of labels produced by a generator,
which must satisfy the following laws:
\RLEQNS{
   labs &:& Gen \fun \power Lbl
\\ labs(G) &=& \setof{\ell_G} \cup labs(G_{:}) \cup labs(G_1) \cup labs(G_2)
  &\elabel{labs-def}
\\ \ell_G &\notin& labs(G_{:})
   &\elabel{labs-unique}
\\ \emptyset &=& labs(G_1) \cap labs(G_2)
  &\elabel{labs-split-disjoint}
}
The conceptually simplest model of such a generator
is one where labels are simply the strings of $:$, $1$ and $2$
that designate how its generator was produced from the root.
With this model we also get the following law:
For any generator expression $G$,
the following four sets are mutually disjoint:
\[
  \setof{\ell_G}
  \quad
  labs(G_{:})
  \quad
  labs(G_1)
  \quad
  labs(G_2)
  \qquad
  \elabel{labs-fully-disjoint}
\]

\includegraphics{images/new-label}

\includegraphics{images/split-gen}


\subsection{Label-Set Invariants}


We want to distinguish between disjointness only:
\[ \{ in | g | out \} \]
from disjointness with label-set exclusion:
\[ [ in | g | out ] \]

We want to be able to specify that certain combinations
of labels should never appear simultaneously
in the global label set ($ls$).
We do this by defining a label-set invariant $I$
which has alphabet $ls,in,out,g$.

We start by defining a general abstract way of specifying
sets with valid combinations
of values drawn from a parameter type $\tau$.
\RLEQNS{
   i \in I_\tau &::=& \tau   &  \mbox{can contain this value}
\\ &\mid& \otimes(i,\ldots,i) & \mbox{at most one of these allowed to contain}
\\ &\mid& \cup (i,\ldots,i) & \mbox{any of these allowed to contain}
}

We define what it means for an atomic action invocation
to satisfy an invariant parameterised on the label type ($Lbl$).
\RLEQNS{
  ls \textbf{ lsat } I \land A(E|a|N) &\implies& ls' \textbf{ lsat } I
}
We can re-formulate this as a following equivalent test:
\RLEQNS{
   A(E|a|N) \textbf{ sat } I_{Lbl}
   &\defs&
   E \textbf{ lsat } I_{Lbl} \land N \textbf{ lsat } I_{Lbl}
}
In effect we identify the occurrences of labels within this $I$-structure,
and then check the multiplicity constraints:
\RLEQNS{
   L \textbf{ lsat } I_{Lbl} &\defs& res=ok
\\ \textbf{where} && (res,\_) = occChk (occ_L I_{Lbl})
}
The $occ$ function takes a set $L$ of $\tau$ and an $I_\tau$ and returns a $I_\Bool$
that records if the corresponding element of $\tau$ is present in $L$.
\RLEQNS{
   occ &:& \power \tau \fun I_\tau \fun I_\Bool
\\ occ_L~\ell &\defs& \ell \in L
\\ occ_L~\otimes(i_1,\ldots,i_n)
   &\defs&
   \otimes(occ_L~i_1,\ldots,occ_L~i_n)
\\ occ_L~\cup(i_1,\ldots,i_n)
   &\defs&
   \cup(occ_L~i_1,\ldots,occ_L~i_n)
}
The $occChk$ function pattern matches across the boolean values to see if
constraints are satisfied.
The first component of the result is an overall ok/fail indicator,
while the second boolean component indicates if values are present
in any component.
\RLEQNS{
   occChk &:& I_\Bool \fun (\setof{ok,fail}\times \Bool)
\\ occChk(b) &\defs& (ok,b)
\\ occChk(\cup(i_1,\ldots,i_n))
   &\defs&
   (fail,\_),
   \textbf{ if }\exists j @ occChk(i_j) = (fail,\_)
\\ && (ok,b_1 \lor \dots \lor b_n),
   \textbf{ if }\forall j @ occChk(i_j) = (ok,b_j)
\\ occChk(\otimes(i_1,\ldots,i_n))
   &\defs&
   (fail,\_),
   \textbf{ if }\exists j @ occChk(i_j) = (fail,\_)
\\&& (fail,\_) \mbox{ if more than one $(ok,true)$}
\\&& (ok,false) \mbox{ if all are $(ok,false)$}
\\&& (ok,true) \mbox{ if  exactly one $(ok,true)$}
}
We note, as a consequence of the above definitions, that
\RLEQNS{
  \emptyset &  \textbf{lsat} &  I & \elabel{emp-lsat-I}
}
for any label-set invariant $I$.

We introduce a shorthand for invariants illustrated as follows.
\RLEQNS{
  ~ [ a,b,c | d,e | f ]
   &=&
   ls \textbf{ lsat } \otimes(\cup(a,b,c),\cup(d,e),f)
\\ ~[ a | (b|c),(d|e) | f ]
  &=&
  ls \textbf{ lsat } \otimes(a,\cup(\otimes(b,c),\otimes(d,e)),f)
}
In effect we make the involvement of $ls$ implicit,
and use bar ($|$) and comma ($,$) to replace $\otimes$ and $\cup$ respectively.
We also have a shorthand that just denotes
the non-intersecting nature of the arguments.
E.g.,
$\setof{A|B|C}$ asserts that $A$, $B$ and $C$ are mutually disjoint,
without any reference to $ls$ or any other set.

\subsubsection{Invariant Examples}

The following examples show how various instances of $ls \textbf{ lsat } I$
get expanded:
\RLEQNS{
   ~[in|out]
   &=&
   \setof{in} \cap \setof{out} = \emptyset
\\ &\land&
  \setof{in} \subseteq ls \implies \setof{out}\cap ls = \emptyset
\\ &\land&
  \setof{out} \subseteq ls \implies \setof{in}\cap ls = \emptyset
\\
\\ ~[in|\ell|out]
   &=&
   \setof{in} \cap \setof{\ell,out} = \emptyset
\\ &\land&
   \setof{\ell} \cap \setof{in,out} = \emptyset
\\ &\land&
   \setof{out} \cap \setof{in,\ell} = \emptyset
\\ &\land&
  \setof{in} \subseteq ls \implies \setof{\ell,out}\cap ls = \emptyset
\\ &\land&
  \setof{\ell} \subseteq ls \implies \setof{in,out}\cap ls = \emptyset
\\ &\land&
  \setof{out} \subseteq ls \implies \setof{in,\ell}\cap ls = \emptyset
\\
\\ ~[ a | (b|c),(d|e) | f ]
   &=&
   \setof{a} \cap \setof{b,c,d,e,f} = \emptyset
\\ &\land&
   \setof{b,c,d,e} \cap \setof{a,f} = \emptyset
\\ &\land&
   \setof{f} \cap \setof{a,b,c,d} = \emptyset
\\ &\land&
  \setof{a} \subseteq ls \implies \setof{b,c,d,e,f}\cap ls = \emptyset
\\ &\land&
  \setof{b,c,d,e} \subseteq ls \implies \setof{a,f}\cap ls = \emptyset
\\ &\land&
  \setof{f} \subseteq ls \implies \setof{a,b,c,d,e}\cap ls = \emptyset
\\ &\land&
   \setof{b} \cap \setof{c} = \emptyset
\\ &\land&
   \setof{d} \cap \setof{e} = \emptyset
\\ &\land&
  \setof{b} \subseteq ls \implies \setof{c}\cap ls = \emptyset
\\ &\land&
  \setof{c} \subseteq ls \implies \setof{b}\cap ls = \emptyset
\\ &\land&
  \setof{d} \subseteq ls \implies \setof{e}\cap ls = \emptyset
\\ &\land&
  \setof{e} \subseteq ls \implies \setof{d}\cap ls = \emptyset
}
Here one of $b$ or $c$ may occur in $ls$ along with one of  $d$ or $e$.
In this theory,
we use the label generators to ensure that the disjointness
conditions of the invariants are always true, by construction.

\subsubsection{Notation}

\RLEQNS{
   ls(\ell) &\defs& \ell \in ls
\\ ls(L) &\defs& L \subseteq ls
\\ ls(\B\ell) &\defs& \ell \notin ls
\\ ls(\B L) &\defs& L \cap ls = \emptyset
}
