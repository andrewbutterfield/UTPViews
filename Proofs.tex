\section{Proofs}\label{sec:proofs}

\subsection{Proof of \textsf{UTCP-NF}}

We take the normal form
\[
  I_P \land \Skip
      \lor
      \bigvee_{n \in 0\dots}
      (I_P \land I_n \land X(E_n|a_n|R_n|A_n) \land S_n)
\]
and  assume that the values ranged over by index $n$
include a ``zero'' value $n_0$,
and we define:
\RLEQNS{
   I_{n_0} &\defs& \true
\\ E_{n_0} &\defs& \setof{in}
\\ a_{n_0} &\defs& ii
\\ R_{n_0} &\defs& \setof{in}
\\ A_{n_0} &\defs& \setof{out}
\\ S_{n_0} &\defs& \true
}
then we have:
\RLEQNS{
   I_P \land \Skip
   &=&
   I_P \land I_{n_0} \land X(E_{n_0}|a_{n_0}|R_{n_0}|A_{n_0}) \land S_{n_0}
}
So, we can simplify the normal form to the more uniform:
\[
\bigvee_n (I_{P_n} \land X(E_n|a_n|R_n|A_n) \land S_n)
\]
where $I_{P_n} = I_P \land I_n$.
For brevity, we shall use $X_n$ to denote $ I_{P*} \land X(E_n|a_n|R_n|A_n)$,
or just $X(E_n|a_n|R_n|A_n)$, as appropriate in context.

So we want to show for all programs $P$
that
\[
 P = \bigvee_n X_n
\]
for $n$ drawn from some index-set, and appropriate $X_n$.
We do so by induction over the program syntax.

\subsubsection{Proof of \textsf{UTCP-NF}, Base Case}

\RLEQNS{
  && a
\EQ{\eref{sem:atomic}}
\\&& [in|out] \land \W(A(in|a|out)
\EQ{\eref{nf-atomic}, Calculation done elsewhere}
\\&& [in|out] \land (\Skip \lor A(in|a|out))
\EQ{\eref{X-def}}
\\&& [in|out] \land (\Skip \lor X(in|a|in|out))
}
So, we have the following correspondence:
\RLEQNS{
   I_{a^*} &=& [in|out]
\\ E_{n_1} &=& \setof{in}
\\ a_{n_1} &=& a
\\ R_{n_1} &=& \setof{in}
\\ A_{n_1} &=& \setof{out}
\\ a &=& X_{n_0} \lor X_{n_1}
\\ &=& \bigvee_{n\in \setof{n_0,n_1}}  X_n
}

\subsubsection{Proof of \textsf{UTCP-NF}, Inductive Step}

We assume the following as inductive hypotheses,
where $n$ and $m$ range over different index sets,
and we use $Y$ and $Z$ instead of $X$ to reduce confusion.
\RLEQNS{
   P &=& \bigvee_p Y_p & \elabel{UTCP-NF-hyp-P}
\\ Y_{p_0} &=& \Skip \land I_{P^*} & \mbox{(reminder)}
\\ Q &=& \bigvee_q Z_q & \elabel{UTCP-NF-hyp-Q}
\\ Z_{q_0} &=& \Skip \land I_{Q^*} & \mbox{(reminder)}
}
Frequently,
we need to determine the form of $((\bigvee_p Y_p) \lor (\bigvee_q Z_q))^i$
for $i \geq 2$.
We start with $i=2$
\RLEQNS{
  && ((\bigvee_p Y_p) \lor (\bigvee_q Z_q))^2
\EQ{Defn. of $P^2$}
\\&& ((\bigvee_p Y_p) \lor (\bigvee_q Z_q))\seq ((\bigvee_p Y_p) \lor (\bigvee_q Z_q))
\EQ{$\lor$-$\seq$-distr}
\\&&     (\bigvee_p Y_p) \!\seq\! (\bigvee_p Y_p)
   \lor (\bigvee_p Y_p) \!\seq\! (\bigvee_q Z_q)
   \lor (\bigvee_q Z_q) \!\seq\! (\bigvee_p Y_p)
   \lor (\bigvee_q Z_q) \!\seq\! (\bigvee_q Z_q)
\EQ{$\lor$-$\seq$-distr}
\\&&     \bigvee_{p_a,p_b} (Y_{p_a} \!\seq\!  Y_{p_b})
   \lor \bigvee_{p,q}     (Y_p \!\seq\!  Z_q)
   \lor \bigvee_{q,p}     (Z_q \!\seq\! Y_p)
   \lor \bigvee_{q_a,q_b} (Z_{q_a} \!\seq\! Z_{q_b})
\EQ{term $Y_{p_0} \seq Z_{q_0}$ occurs, so pull out explicitly}
\\&&     Y_{p_0} \seq Z_{q_0}
\\&\lor&\bigvee_{p_a,p_b} (Y_{p_a} \!\seq\!  Y_{p_b})
   \lor \bigvee_{p,q}     (Y_p \!\seq\!  Z_q)
   \lor \bigvee_{q,p}     (Z_q \!\seq\! Y_p)
   \lor \bigvee_{q_a,q_b} (Z_{q_a} \!\seq\! Z_{q_b})
\EQ{$\Skip \seq \Skip = \Skip$}
\\&&     \Skip
   \lor \bigvee_{p_a,p_b} (Y_{p_a} \!\seq\!  Y_{p_b})
   \lor \bigvee_{p,q}     (Y_p \!\seq\!  Z_q)
   \lor \bigvee_{q,p}     (Z_q \!\seq\! Y_p)
   \lor \bigvee_{q_a,q_b} (Z_{q_a} \!\seq\! Z_{q_b})
\EQ{\eref{X-then-X}, with $f$ being an injective index function}
\\&&     \Skip
\\&\lor& \bigvee_{p_a,p_b} X_m \textbf{ where } m = f(p_a,p_b)
\\&\lor& \bigvee_{p,q}     X_m \textbf{ where } m = f(p,q)
\\&\lor& \bigvee_{q,p}     X_n \textbf{ where } n = f(q,p)
\\&\lor& \bigvee_{q_a,q_b} X_n \textbf{ where } n = f(q_a,q_b)
\EQ{We have the form, so re-index}
\\&& \bigvee_m X_m \lor \bigvee_n X_n
}
So $((\bigvee_p Y_p) \lor (\bigvee_q Z_q))^2$
has the form $\bigvee_m X_m \lor \bigvee_n X_n$
for appropriate new indexing.
Each time we compose we get the same result.

So, we
have that, for all $i \geq 1$, that
\[
  ((\bigvee_p Y_p) \lor (\bigvee_q Z_q))^i
\]
has the form
\[
 \bigvee_{m_i} X_{m_i} \lor \bigvee_{n_i} X_{n_i}
\]
From this we can conclude
that
\[
 \bigvee_i (\bigvee_p Y_p \lor \bigvee_q Z_q)^i
\]
has the form
\[
  \bigvee_i (\bigvee_{p_i} X_{p_i} \lor \bigvee_{q_i} X_{q_i})
\]
This can be collapsed down to one big disjunction:
\[
  \bigvee_x X_x
\]
We just need to show the details for each construct,
and check what happens to the invariants.

\subsubsection{Proof of \textsf{UTCP-NF}, Parallel Case}

We now look at the parallel case in more detail.
We want to show that $P \parallel Q$ has a normal form
if $P$ and $Q$ do.
\RLEQNS{
  && P \parallel Q
\EQ{\eref{sem:par}}
\\&& [in|(\ell_{g1}|\ell_{g1:}),(\ell_{g2}|\ell_{g2:})|out] \land {}
\\&& \W(\quad\phlor A(in|ii|\ell_{g1},\ell_{g2})
\\ && \qquad {}\lor
   P[g_{1::},\ell_{g1},\ell_{g1:}/g,in,out]
\\ && \qquad {}\lor
    Q[g_{2::},\ell_{g2},\ell_{g2:}/g,in,out]
\\ && \qquad {}\lor
   A(\ell_{g1:},\ell_{g2:}|ii|out)~)
\EQ{\eref{UTCP-NF-hyp-P}, \eref{UTCP-NF-hyp-Q}}
\\&& [in|(\ell_{g1}|\ell_{g1:}),(\ell_{g2}|\ell_{g2:})|out] \land {}
\\&& \W(\quad\phlor A(in|ii|\ell_{g1},\ell_{g2})
\\ && \qquad {}\lor
   (\bigvee_p Y_p)[g_{1::},\ell_{g1},\ell_{g1:}/g,in,out]
\\ && \qquad {}\lor
    (\bigvee_q Z_q)[g_{2::},\ell_{g2},\ell_{g2:}/g,in,out]
\\ && \qquad {}\lor
   A(\ell_{g1:},\ell_{g2:}|ii|out)~)
\EQ{substitution}
\\&& [in|(\ell_{g1}|\ell_{g1:}),(\ell_{g2}|\ell_{g2:})|out] \land {}
\\&& \W(\quad\phlor A(in|ii|\ell_{g1},\ell_{g2})
\\ && \qquad {}\lor
   (\bigvee_p Y_p[g_{1::},\ell_{g1},\ell_{g1:}/g,in,out])
\\ && \qquad {}\lor
    (\bigvee_q Z_q[g_{2::},\ell_{g2},\ell_{g2:}/g,in,out])
\\ && \qquad {}\lor
   A(\ell_{g1:},\ell_{g2:}|ii|out)~)
\EQ{\eref{W-prog-stutters}}
\\&& [in|(\ell_{g1}|\ell_{g1:}),(\ell_{g2}|\ell_{g2:})|out] \land {}
\\&& \W(\quad\phlor \Skip
\\ && \qquad {}\lor A(in|ii|\ell_{g1},\ell_{g2})
\\ && \qquad {}\lor
   (\bigvee_p Y_p[g_{1::},\ell_{g1},\ell_{g1:}/g,in,out])
\\ && \qquad {}\lor
    (\bigvee_q Z_q[g_{2::},\ell_{g2},\ell_{g2:}/g,in,out])
\\ && \qquad {}\lor
   A(\ell_{g1:},\ell_{g2:}|ii|out)~)
}
We pause here, and note that the $Y_p$ and $Z_q$ terms contain
conjunctions of $X$-actions with invariants:
\RLEQNS{
   Y_p &=&  I_{P^*} \land X(E_p|a_p|R_p|A_p)
\\ Z_q &=&  I_{Q^*} \land X(E_q|a_q|R_q|A_q)
}
What we want to do is to pull these out
and push them to the top.


\newpage
\subsection{Proof of \textsf{HL-X-red}}
\RLEQNS{
  && \HL(X(E|a|R|A))
\EQ{\eref{HL-def}}
\\&& \exists ls,ls' @ X(E|a|R|A)
\EQ{\eref{X-def}}
\\&& \exists ls,ls' @
     ls(E) \land s' \in \sem a s \land ls' = (ls \setminus R) \cup A
\EQ{1pt-law, $ls'$, noting that $E$ is ground}
\\&& \exists ls @
     ls(E) \land s' \in \sem a s
\EQ{reduce $ls$ scope}
\\&& (\exists ls @ ls(E)) \land s' \in \sem a s
\EQ{witness: $ls=E$}
\\&& s' \in \sem a s
}

\newpage
\subsection{Proof of \textsf{HL-X-sub-red}}

We start with a lemma \elabel{X-gnd-subst}:
\RLEQNS{
  && X(E|a|R|A)[G,I,O/g,in,out]
\EQ{\eref{X-def}}
\\&& (ls(E) \land s' \in \sem a s \land ls' = (ls \setminus R) \cup A)
     [G,I,O/g,in,out]
\EQ{substitution}
\\&& ls(E[G,I,O/g,in,out])
    \land s' \in \sem a s
\\ &\land& ls' = (ls \setminus R[G,I,O/g,in,out]) \cup A[G,I,O/g,in,out]
\EQ{\eref{X-def}}
\\&& X(E[G,I,O/g,in,out]|a|R[G,I,O/g,in,out]|A[G,I,O/g,in,out])
}

\RLEQNS{
  && \HL(X(E|a|R|A)[G,I,O/g,in,out])
\EQ{\eref{X-gnd-subst}}
\\&& \HL(X(E[G,I,O/g,in,out]|a|R[G,I,O/g,in,out]|A[G,I,O/g,in,out]))
\EQ{\eref{HL-X-red}, provided $E[G,I,O/g,in,out]$ is ground.}
\\&& s' \in \sem a s
}

\newpage
\subsection{Proof of \textsf{HL-or-distr}}
\RLEQNS{
  && \HL(A \lor B)
\EQ{\eref{HL-def}}
\\&& \exists ls,ls' @ A \lor B
\EQ{$\exists$-$\lor$-distr}
\\&& (\exists ls,ls' @ A) \lor (\exists ls,ls' @ B)
\EQ{\eref{HL-def}}
\\&& \HL(A) \lor \HL{B}
}

\newpage
\subsection{Proof of \textsf{HL-I-and-skip}}
\RLEQNS{
  && \HL(I \land \Skip)
\EQ{\eref{UTP-skip-def}}
\\&& \HL(I \land ls'=ls \land s'=s)
\EQ{\eref{HL-def}}
\\&& \exists ls,ls' @ I \land ls'=ls \land s'=s
\EQ{1-pt, $ls'=ls$}
\\&& \exists ls @ I[ls/ls'] \land s'=s
\EQ{$ls'$ not in $I$}
\\&& \exists ls @ I \land s'=s
\EQ{shrink scope of $ls$, expand $I$ shorthand (a bit)}
\\&& (\exists ls @ ls \textbf{ lsat } I) \land s'=s
\EQ{witness, $ls=\emptyset$}
\\&& s'=s
\EQ{\eref{ii-def}}
\\&& ii
}
We note that we can deduce the following lemma,
that generalises to the use of ground substitutions $\sigma$:
\RLEQNS{
  && \exists ls @ I\sigma
\EQ{expand $I$ shorthand (a bit)}
\\&& (\exists ls @ ls \textbf{ lsat } I\sigma)
\EQ{witness, $ls=\emptyset$, substitution irrelevant}
\\&& \true
}
It also works if we have a conjunction of invariants:
\RLEQNS{
  \exists ls @ (I \land J)\sigma &=& \true
}
The one witness satisfies all such invariants.


\newpage
\subsection{Proof of \textsf{HL-I-and-X}}
\RLEQNS{
  && \HL(I \land X(E|a|R|A) \land S)
\EQ{\eref{X-def}}
\\&& \HL(I \land ls(E) \land a \land ls'=(ls\setminus R)\cup A \land S)
\EQ{\eref{HL-def}}
\\&& \exists ls,ls' @ I \land ls(E) \land a \land ls'=(ls\setminus R)\cup A \land S)
\EQ{1-pt, $ls'=(ls\setminus R)\cup A$, noting $E$, $S$ are ground}
\\&& \exists ls @ I \land ls(E) \land a \land S)
\EQ{shrink scope of $ls$}
\\&& (\exists ls @ I \land ls(E)) \land a \land S)
\EQ{witness $ls=E$, assuming that $E \textbf{ lsat } I$}
\\&& a \land S
}

\subsection{Proof of \textsf{HL-I-and-A}}

Law \ecite{HL-I-and-A} is an easy consequence of \ecite{HL-I-and-X},
with $R=E$, $A=N$ and $S=\true$.

\newpage
\subsection{Proof of \textsf{HL-nf}}


\RLEQNS{
  && \HL\left (I \land \left( \Skip \lor \bigvee A(a_i) \right) \right)
\EQ{$\land$-$\lor$-distr}
\\&& \HL\left( I \land \Skip \lor \bigvee (I \land A(a_i)) \right)
\EQ{\eref{HL-or-distr}}
\\&& \HL(I \land \Skip) \lor \bigvee \HL(I \land A(a_i))
\EQ{\eref{HL-I-and-skip} and \eref{HL-I-and-A}}
\\&& ii \lor \bigvee a_i
}

\newpage
\subsection{Proof of \textsf{HL-subs-indep}}

We want to prove $\HL(P\sigma_1)=\HL(P\sigma_2)$, where
\begin{itemize}
  \item $P$ is a predicate that can be put in normal form.
  \item $\sigma_i$ is a substitution of the form $[G_i,I_i,O_i/g,in,out]$.
  \item $G_i$, $I_i$ and $O_i$ are ground.
\end{itemize}
We note the following lemmas that hold given such substitutions
(easy proofs as exercise):
\RLEQNS{
   \Skip\sigma    &=& \Skip                & \elabel{gnd-sub-skip}
\\ A(E|a|N)\sigma &=& A(E\sigma|a|N\sigma) & \elabel{gnd-sub-A}
}

We do this proof by induction over the UTCP syntax.

\subsubsection{\textsf{HL-subs-indep}, Base Case}

Here $\sigma=[G,I,O/g,in,out]$.
\RLEQNS{
  && \HL(a\sigma)
\EQ{\eref{nf-atomic}}
\\&& \HL(([in|out] \land (\Skip \lor A(in|a|out)))\sigma)
\EQ{$\land$-$\lor$-distr}
\\&& \HL(([in|out] \land \Skip \lor [in|out] \land A(in|a|out)))\sigma)
\EQ{substitution, using $\eref{gnd-sub-skip}$ and  $\eref{gnd-sub-A}$}
\\&& \HL([in\sigma|out\sigma] \land \Skip
         \lor
         [in\sigma|out\sigma] \land A(in\sigma|a|out\sigma)))
\EQ{Apply $\sigma$}
\\&& \HL([I|O] \land \Skip
         \lor
         [I|O] \land A(I|a|O)))
\EQ{\eref{HL-or-distr}}
\\&& \HL([I|O] \land \Skip)
         \lor
         \HL([I|O] \land A(I|a|O)))
\EQ{\eref{HL-I-and-skip} and \eref{HL-I-and-A}}
\\&& ii \lor a
}
We have $\HL(a\sigma) = ii \lor a$ for arbitrary ground $\sigma$,
so our result is immediate.

\newpage
\subsubsection{\textsf{HL-subs-indep}, Inductive Hypothesis}

For each inductive step case below we assume,
for \emph{arbitrary} $\sigma_1$, $\sigma_2$.
\RLEQNS{
   \HL(P\sigma_1) &=& \HL(P\sigma_2) & \elabel{HL-subs-indep-hyp-P}
\\ \HL(Q\sigma_1) &=& \HL(Q\sigma_2) & \elabel{HL-subs-indep-hyp-Q}
}
We also assume the following normal forms,
where we distribute the invariant in.
\RLEQNS{
   P &=& I \land \Skip \lor \bigvee_j (I \land  A(a_j))
   & \elabel{HL-subs-indep-nf-P}
\\ Q &=& J \land \Skip \lor \bigvee_k  (J \land  A(b_k))
   & \elabel{HL-subs-indep-nf-Q}
}
An important lemma is one that states that these ground substitutions
distribute through UTP sequential composition:
\RLEQNS{
  (P \seq Q)\sigma &=& P\sigma \seq Q\sigma & \elabel{gnd-subs-seq-distr}
}

\subsubsection{\textsf{HL-subs-indep}, Step: Sequence}

\RLEQNS{
  && \HL((P \cseq Q)\sigma)
\EQ{\eref{sem:seq}}
\\&& \HL(([in|\ell_g|out]\land \W(P[g_{:1},\ell_g/g,out] \lor Q[g_{:2},\ell_g/g,in]))\sigma)
\EQ{\eref{W-as-NDC}}
\\&& \HL(([in|\ell_g|out]
          \land
          (\Skip
           \lor
           \bigvee_i (P[g_{:1},\ell_g/g,out]
                      \lor
                      Q[g_{:2},\ell_g/g,in])^i))\sigma)
\EQ{substitution, noting \eref{gnd-subs-seq-distr}}
\\&& \HL([in|\ell_g|out]\sigma
          \land
          (\Skip\sigma
           \lor
           \bigvee_i (P[g_{:1},\ell_g/g,out]\sigma
                      \lor
                      Q[g_{:2},\ell_g/g,in]\sigma)^i))
\EQ{\eref{gnd-sub-skip}, $\land$-$\lor$-distr}
\\&& \HL( [in|\ell_g|out]\sigma \land \Skip
          \lor
          [in|\ell_g|out]\sigma \land
            \bigvee_i (P[g_{:1},\ell_g/g,out]\sigma
                      \lor
                      Q[g_{:2},\ell_g/g,in]\sigma)^i)
\EQ{\eref{HL-or-distr}}
\\&& \HL( [in|\ell_g|out]\sigma \land \Skip)
          \lor
     \HL([in|\ell_g|out]\sigma \land
            \bigvee_i (P[g_{:1},\ell_g/g,out]\sigma
                      \lor
                      Q[g_{:2},\ell_g/g,in]\sigma)^i)
\EQ{\eref{HL-I-and-skip}, substitution, subst. composition}
\\&& \say{$\sigma_1 = [g_{:1},\ell_g/g,out]\sigma$}
\\&& \say{$\sigma_2 = [g_{:2},\ell_g/g,in]\sigma$}
\\&& ii \lor
     \HL([I|\ell_G|O] \land
            \bigvee_i (P\sigma_1
                      \lor
                      Q\sigma_2)^i)
\EQ{normal forms}
\\&& ii \lor
     \HL([I|\ell_G|O] \land
            \bigvee_i ((I \land \Skip \lor \bigvee_j (I \land  A(a_j)))\sigma_1
                      \lor
                      (J \land \Skip \lor \bigvee_k (J \land  A(b_k)))\sigma_2)^i)
\EQ{substitution, \eref{gnd-sub-skip}}
\\&& ii \lor
     \HL([I|\ell_G|O]
\\ && {} \land
            \bigvee_i ( I\sigma_1 \land \Skip \lor \bigvee_j (I\sigma_1 \land  A(a_j)\sigma_1)
                        \lor
                        J\sigma_2 \land \Skip \lor \bigvee_k (J\sigma_2 \land  A(b_k)\sigma_2))^i)
\EQ{$\land$-$\lor$-distr}
\\&& ii \lor
     \HL(\bigvee_i (
         [I|\ell_G|O] \land I\sigma_1 \land \Skip
         \lor
         \bigvee_j ([I|\ell_G|O] \land I\sigma_1 \land  A(a_j)\sigma_1)
\\&& \qquad\qquad\, {}\lor [I|\ell_G|O] \land J\sigma_2 \land \Skip
           \lor \bigvee_k ([I|\ell_G|O] \land J\sigma_2 \land  A(b_k)\sigma_2)
           )^i)
\EQ{HL-or-distr}
\\&& ii \lor
    \bigvee_i (
          \HL([I|\ell_G|O] \land I\sigma_1 \land \Skip)
         \lor
         \bigvee_j  \HL([I|\ell_G|O] \land I\sigma_1 \land  A(a_j)\sigma_1)
\\&& \qquad\qquad\, {}\lor  \HL([I|\ell_G|O] \land J\sigma_2 \land \Skip)
           \lor \bigvee_k  \HL([I|\ell_G|O] \land J\sigma_2 \land  A(b_k)\sigma_2)
           )^i
\EQ{\eref{HL-I-and-skip},\eref{HL-I-and-A}, with invariant conjuncts}
\\&& ii \lor
    \bigvee_i (
          ii
         \lor
         \bigvee_j  a_j
\lor  ii
           \lor \bigvee_k  b_k
           )^i
\EQ{tidy-up (not strictly necessary, but nice)}
\\&& ii \lor (\bigvee_j  a_j \lor \bigvee_k  b_k)^i
}
We see that $\HL((P\cseq Q)\sigma)$ is independent of $\sigma$,
provided $P$ and $Q$ have normal forms.
So we don't need induction here.
It is needed to show that our definitions do have these normal forms, though.

\newpage
\subsection{Lemma \textsf{gnd-subs-seq-distr}}

\RLEQNS{
  && (P\seq Q)\sigma
\EQ{\eref{UTP-seq-def}}
\\&& (\exists s_m,ls_m @
       P[s_m,ls_m/s',ls']
       \land
       Q[s_m,ls_m/s,ls])\sigma
\EQ{$\sigma$ does not mention $s_m$, $ls_m$}
\\&& \exists s_m,ls_m @
       P[s_m,ls_m/s',ls']\sigma
       \land
       Q[s_m,ls_m/s,ls]\sigma
\EQ{$\sigma$ is ground, does not mention $s,s',ls,ls'$}
\\&& \exists s_m,ls_m @
       (P\sigma)[s_m,ls_m/s',ls']
       \land
       (Q\sigma)[s_m,ls_m/s,ls]
\EQ{\eref{UTP-seq-def}}
\\&& (P\sigma) \seq (Q\sigma)
}
In particular, this means that $(P^i)\sigma = (P\sigma)^i$.
