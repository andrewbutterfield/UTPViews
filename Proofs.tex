\section{Proofs}\label{sec:proofs}

\subsection{Proof of \textsf{UTCP-NF}}

We take the normal form
\[
  I_P \land \Skip
      \lor
      \bigvee_{n \in 0\dots}
      (I_P \land I_n \land X(E_n|a_n|R_n|A_n) \land S_n)
\]
and  assume that the values ranged over by index $n$
include a ``zero'' value $n_0$,
and we define:
\RLEQNS{
   I_{n_0} &\defs& \true
\\ E_{n_0} &\defs& \setof{in}
\\ a_{n_0} &\defs& ii
\\ R_{n_0} &\defs& \setof{in}
\\ A_{n_0} &\defs& \setof{out}
\\ S_{n_0} &\defs& \true
}
then we have:
\RLEQNS{
   I_P \land \Skip
   &=&
   I_P \land I_{n_0} \land X(E_{n_0}|a_{n_0}|R_{n_0}|A_{n_0}) \land S_{n_0}
}
So, we can simplify the normal form to the more uniform:
\[
\bigvee_n (I_{P_n} \land X(E_n|a_n|R_n|A_n) \land S_n)
\]
where $I_{P_n} = I_P \land I_n$.
For brevity, we shall use $X_n$ to denote
$I_{P_n} \land X(E_n|a_n|R_n|A_n) \land S_n$,
or just $X(E_n|a_n|R_n|A_n)$, as appropriate in context.

So we want to show for all programs $P$
that
\[
 P = \bigvee_n X_n
\]
for $n$ drawn from some index-set, and appropriate $X_n$.
We do so by induction over the program syntax.

\newpage
\subsubsection{Proof of \textsf{UTCP-NF}, Base Case}

\RLEQNS{
  && \catom a
\EQ{\eref{sem:atomic}}
\\&& [in|out] \land \W(A(in|a|out)
\EQ{\eref{nf-atomic}, Calculation done elsewhere}
\\&& [in|out] \land (\Skip \lor A(in|a|out))
\EQ{\eref{X-def}}
\\&& [in|out] \land (\Skip \lor X(in|a|in|out))
}
So, we have the following correspondence:
\RLEQNS{
   I_{a^*} &=& [in|out]
\\ E_{n_1} &=& \setof{in}
\\ a_{n_1} &=& a
\\ R_{n_1} &=& \setof{in}
\\ A_{n_1} &=& \setof{out}
\\ \catom a &=& X_{n_0} \lor X_{n_1}
\\ &=& \bigvee_{n\in \setof{n_0,n_1}}  X_n
}

\newpage
\subsubsection{Proof of \textsf{UTCP-NF}, Inductive Step}

We assume the following as inductive hypotheses,
where $n$ and $m$ range over different index sets,
and we use $Y$ and $Z$ instead of $X$ to reduce confusion.
\RLEQNS{
   P &=& \bigvee_p Y_p & \elabel{UTCP-NF-hyp-P}
\\ Y_{p_0} &=& \Skip \land I_{P^*} & \mbox{(reminder)}
\\ Q &=& \bigvee_q Z_q & \elabel{UTCP-NF-hyp-Q}
\\ Z_{q_0} &=& \Skip \land I_{Q^*} & \mbox{(reminder)}
}


We shall look at parallel case in detail.
All the other composites have a similar argument.
We want to show that $P \parallel Q$ has a normal form
if $P$ and $Q$ do.
\RLEQNS{
  && P \parallel Q
\EQ{\eref{sem:par}}
\\&& [in|(\ell_{g1}|\ell_{g1:}),(\ell_{g2}|\ell_{g2:})|out] \land {}
\\&& \W(\quad\phlor A(in|ii|\ell_{g1},\ell_{g2})
\\ && \qquad {}\lor
   P[g_{1::},\ell_{g1},\ell_{g1:}/g,in,out]
\\ && \qquad {}\lor
    Q[g_{2::},\ell_{g2},\ell_{g2:}/g,in,out]
\\ && \qquad {}\lor
   A(\ell_{g1:},\ell_{g2:}|ii|out)~)
\EQ{\eref{UTCP-NF-hyp-P}, \eref{UTCP-NF-hyp-Q}}
\\&& [in|(\ell_{g1}|\ell_{g1:}),(\ell_{g2}|\ell_{g2:})|out] \land {}
\\&& \W(\quad\phlor A(in|ii|\ell_{g1},\ell_{g2})
\\ && \qquad {}\lor
   (\bigvee_p Y_p)[g_{1::},\ell_{g1},\ell_{g1:}/g,in,out]
\\ && \qquad {}\lor
    (\bigvee_q Z_q)[g_{2::},\ell_{g2},\ell_{g2:}/g,in,out]
\\ && \qquad {}\lor
   A(\ell_{g1:},\ell_{g2:}|ii|out)~)
\EQ{substitution, $I_\parallel$ abbreviates invariant}
\\&& I_\parallel \land {}
\\&& \W(\quad\phlor A(in|ii|\ell_{g1},\ell_{g2})
\\ && \qquad {}\lor
   (\bigvee_p Y_p[g_{1::},\ell_{g1},\ell_{g1:}/g,in,out])
\\ && \qquad {}\lor
    (\bigvee_q Z_q[g_{2::},\ell_{g2},\ell_{g2:}/g,in,out])
\\ && \qquad {}\lor
   A(\ell_{g1:},\ell_{g2:}|ii|out)~)
\EQ{push invariant in, fuse disjuncts and re-index}
\\&& \W(\bigvee_r X_r), \quad r \in r_1,r_2,\dots
\\&& \textbf{where}\\&& X_{r_1} = I_\parallel \land A(in|ii|\ell_{g1},\ell_{g2})
\\&& X_{r_2} = I_\parallel \land A(\ell_{g1:},\ell_{g2:}|ii|out)
\\&& X_r \textbf{ ranges over all } I_\parallel \land Y_p, I_\parallel \land Z_q \textbf{ otherwise}
\EQ{\eref{W-prog-stutters}}
\\&& \W(\Skip \lor\bigvee_r X_r), \quad r \in r_1,r_2,\dots
\EQ{\eref{W-as-NDC}}
\\&& \bigvee_{i \in 0,\dots} \Skip \seq (\bigvee_r X_r)^i
\EQ{\eref{W-NDC-preserves-NF}}
\\&& \bigvee_n X_n
}

\subsubsection{Proof of \textsf{UTCP-NF}, Lemma \textsf{W-NDC-preserves-NF}}

We show that $\W$ preserves normal forms
\RLEQNS{
   \bigvee_{i \in 0,\dots} \Skip \seq (\bigvee_r X_r)^i
   &=&
   \bigvee_n X_n
   & \elabel{W-NDC-preserves-NF}
}
We start by showing that sequential composition preserves normal forms:
\RLEQNS{
  && (\bigvee_p Y_p) \seq (\bigvee_q Z_q)
\EQ{$\lor$-$\seq$-distr}
\\&& \bigvee_{p,q} (Y_p \seq Z_q)
\EQ{\eref{X-then-X}, $f$ injective}
\\&& \bigvee_f(p,q) X_{f(p,q)} \textbf{ where } X_{f(p,q)} =  (Y_p \seq Z_q)
\EQ{Let $s \in \rng f$}
\\&& \bigvee_s X_s
}
So we can deduce that $(\bigvee_r X_r)^2$ is in normal form.
A simple induction argument shows this holds for $i=3$ and upwards.
So
\[
  \bigvee_{i \in 0,\dots} \Skip \seq (\bigvee_r X_r)^i
\]
is a disjunction of normal forms,
i.e., disjunctions of $X$s,
so is itself a disjunction of $X$s and hence a normal form.

%Frequently,
%we need to determine the form of $((\bigvee_p Y_p) \lor (\bigvee_q Z_q))^i$
%for $i \geq 2$.
%We start with $i=2$
%\RLEQNS{
%  && ((\bigvee_p Y_p) \lor (\bigvee_q Z_q))^2
%\EQ{Defn. of $P^2$}
%\\&& ((\bigvee_p Y_p) \lor (\bigvee_q Z_q))\seq ((\bigvee_p Y_p) \lor (\bigvee_q Z_q))
%\EQ{$\lor$-$\seq$-distr}
%\\&&     (\bigvee_p Y_p) \!\seq\! (\bigvee_p Y_p)
%   \lor (\bigvee_p Y_p) \!\seq\! (\bigvee_q Z_q)
%   \lor (\bigvee_q Z_q) \!\seq\! (\bigvee_p Y_p)
%   \lor (\bigvee_q Z_q) \!\seq\! (\bigvee_q Z_q)
%\EQ{$\lor$-$\seq$-distr}
%\\&&     \bigvee_{p_a,p_b} (Y_{p_a} \!\seq\!  Y_{p_b})
%   \lor \bigvee_{p,q}     (Y_p \!\seq\!  Z_q)
%   \lor \bigvee_{q,p}     (Z_q \!\seq\! Y_p)
%   \lor \bigvee_{q_a,q_b} (Z_{q_a} \!\seq\! Z_{q_b})
%\EQ{term $Y_{p_0} \seq Z_{q_0}$ occurs, so pull out explicitly}
%\\&&     Y_{p_0} \seq Z_{q_0}
%\\&\lor&\bigvee_{p_a,p_b} (Y_{p_a} \!\seq\!  Y_{p_b})
%   \lor \bigvee_{p,q}     (Y_p \!\seq\!  Z_q)
%   \lor \bigvee_{q,p}     (Z_q \!\seq\! Y_p)
%   \lor \bigvee_{q_a,q_b} (Z_{q_a} \!\seq\! Z_{q_b})
%\EQ{$\Skip \seq \Skip = \Skip$}
%\\&&     \Skip
%   \lor \bigvee_{p_a,p_b} (Y_{p_a} \!\seq\!  Y_{p_b})
%   \lor \bigvee_{p,q}     (Y_p \!\seq\!  Z_q)
%   \lor \bigvee_{q,p}     (Z_q \!\seq\! Y_p)
%   \lor \bigvee_{q_a,q_b} (Z_{q_a} \!\seq\! Z_{q_b})
%\EQ{\eref{X-then-X}, with $f$ being an injective index function}
%\\&&     \Skip
%\\&\lor& \bigvee_{p_a,p_b} X_m \textbf{ where } m = f(p_a,p_b)
%\\&\lor& \bigvee_{p,q}     X_m \textbf{ where } m = f(p,q)
%\\&\lor& \bigvee_{q,p}     X_n \textbf{ where } n = f(q,p)
%\\&\lor& \bigvee_{q_a,q_b} X_n \textbf{ where } n = f(q_a,q_b)
%\EQ{We have the form, so re-index}
%\\&& \bigvee_m X_m \lor \bigvee_n X_n
%}
%So $((\bigvee_p Y_p) \lor (\bigvee_q Z_q))^2$
%has the form $\bigvee_m X_m \lor \bigvee_n X_n$
%for appropriate new indexing.
%Each time we compose we get the same result.
%
%So, we
%have that, for all $i \geq 1$, that
%\[
%  ((\bigvee_p Y_p) \lor (\bigvee_q Z_q))^i
%\]
%has the form
%\[
% \bigvee_{m_i} X_{m_i} \lor \bigvee_{n_i} X_{n_i}
%\]
%From this we can conclude
%that
%\[
% \bigvee_i (\bigvee_p Y_p \lor \bigvee_q Z_q)^i
%\]
%has the form
%\[
%  \bigvee_i (\bigvee_{p_i} X_{p_i} \lor \bigvee_{q_i} X_{q_i})
%\]
%This can be collapsed down to one big disjunction:
%\[
%  \bigvee_x X_x
%\]
%We just need to show the details for each construct,
%and check what happens to the invariants.

\newpage
\subsection{Proof of \textsf{HL-X-red}}
\RLEQNS{
  && \HL(X(E|a|R|A))
\EQ{\eref{HL-def}}
\\&& \exists ls,ls' @ X(E|a|R|A)
\EQ{\eref{X-def}}
\\&& \exists ls,ls' @
     ls(E) \land s' \in \sem a s \land ls' = (ls \setminus R) \cup A
\EQ{1pt-law, $ls'$, noting that $E$ is ground}
\\&& \exists ls @
     ls(E) \land s' \in \sem a s
\EQ{reduce $ls$ scope}
\\&& (\exists ls @ ls(E)) \land s' \in \sem a s
\EQ{witness: $ls=E$}
\\&& s' \in \sem a s
}

\newpage
\subsection{Proof of \textsf{HL-X-sub-red}}

We start with a lemma \elabel{X-gnd-subst}:
\RLEQNS{
  && X(E|a|R|A)[G,I,O/g,in,out]
\EQ{\eref{X-def}}
\\&& (ls(E) \land s' \in \sem a s \land ls' = (ls \setminus R) \cup A)
     [G,I,O/g,in,out]
\EQ{substitution}
\\&& ls(E[G,I,O/g,in,out])
    \land s' \in \sem a s
\\ &\land& ls' = (ls \setminus R[G,I,O/g,in,out]) \cup A[G,I,O/g,in,out]
\EQ{\eref{X-def}}
\\&& X(E[G,I,O/g,in,out]|a|R[G,I,O/g,in,out]|A[G,I,O/g,in,out])
}

\RLEQNS{
  && \HL(X(E|a|R|A)[G,I,O/g,in,out])
\EQ{\eref{X-gnd-subst}}
\\&& \HL(X(E[G,I,O/g,in,out]|a|R[G,I,O/g,in,out]|A[G,I,O/g,in,out]))
\EQ{\eref{HL-X-red}, provided $E[G,I,O/g,in,out]$ is ground.}
\\&& s' \in \sem a s
}

\newpage
\subsection{Proof of \textsf{HL-or-distr}}
\RLEQNS{
  && \HL(A \lor B)
\EQ{\eref{HL-def}}
\\&& \exists ls,ls' @ A \lor B
\EQ{$\exists$-$\lor$-distr}
\\&& (\exists ls,ls' @ A) \lor (\exists ls,ls' @ B)
\EQ{\eref{HL-def}}
\\&& \HL(A) \lor \HL{B}
}

\newpage
\subsection{Proof of \textsf{HL-I-and-skip}}
\RLEQNS{
  && \HL(I \land \Skip)
\EQ{\eref{UTP-skip-def}}
\\&& \HL(I \land ls'=ls \land s'=s)
\EQ{\eref{HL-def}}
\\&& \exists ls,ls' @ I \land ls'=ls \land s'=s
\EQ{1-pt, $ls'=ls$}
\\&& \exists ls @ I[ls/ls'] \land s'=s
\EQ{$ls'$ not in $I$}
\\&& \exists ls @ I \land s'=s
\EQ{shrink scope of $ls$, expand $I$ shorthand (a bit)}
\\&& (\exists ls @ ls \textbf{ lsat } I) \land s'=s
\EQ{witness, $ls=\emptyset$}
\\&& s'=s
\EQ{\eref{ii-def}}
\\&& ii
}
We note that we can deduce the following lemma,
that generalises to the use of ground substitutions $\sigma$:
\RLEQNS{
  && \exists ls @ I\sigma
\EQ{expand $I$ shorthand (a bit)}
\\&& (\exists ls @ ls \textbf{ lsat } I\sigma)
\EQ{witness, $ls=\emptyset$, substitution irrelevant}
\\&& \true
}
It also works if we have a conjunction of invariants:
\RLEQNS{
  \exists ls @ (I \land J)\sigma &=& \true
}
The one witness satisfies all such invariants.


\newpage
\subsection{Proof of \textsf{HL-I-and-X}}
\RLEQNS{
  && \HL(I \land X(E|a|R|A) \land S)
\EQ{\eref{X-def}}
\\&& \HL(I \land ls(E) \land a \land ls'=(ls\setminus R)\cup A \land S)
\EQ{\eref{HL-def}}
\\&& \exists ls,ls' @ I \land ls(E) \land a \land ls'=(ls\setminus R)\cup A \land S)
\EQ{1-pt, $ls'=(ls\setminus R)\cup A$, noting $E$, $S$ are ground}
\\&& \exists ls @ I \land ls(E) \land a \land S)
\EQ{shrink scope of $ls$}
\\&& (\exists ls @ I \land ls(E)) \land a \land S)
\EQ{witness $ls=E$, assuming that $E \textbf{ lsat } I$}
\\&& a \land S
}

\subsection{Proof of \textsf{HL-I-and-A}}

Law \ecite{HL-I-and-A} is an easy consequence of \ecite{HL-I-and-X},
with $R=E$, $A=N$ and $S=\true$.

\newpage
\subsection{Proof of \textsf{HL-nf}}


\RLEQNS{
  && \HL\left (I \land \left( \Skip \lor \bigvee A(a_i) \right) \right)
\EQ{$\land$-$\lor$-distr}
\\&& \HL\left( I \land \Skip \lor \bigvee (I \land A(a_i)) \right)
\EQ{\eref{HL-or-distr}}
\\&& \HL(I \land \Skip) \lor \bigvee \HL(I \land A(a_i))
\EQ{\eref{HL-I-and-skip} and \eref{HL-I-and-A}}
\\&& ii \lor \bigvee a_i
}

\newpage
\subsection{Proof of \textsf{HL-subs-indep}}

We want to prove $\HL(P\sigma_1)=\HL(P\sigma_2)$, where
\begin{itemize}
  \item $P$ is a predicate that can be put in normal form.
  \item $\sigma_i$ is a substitution of the form $[G_i,I_i,O_i/g,in,out]$.
  \item $G_i$, $I_i$ and $O_i$ are ground.
\end{itemize}
We note the following lemmas that hold given such substitutions
(easy proofs as exercise):
\RLEQNS{
   \Skip\sigma    &=& \Skip                & \elabel{gnd-sub-skip}
\\ A(E|a|N)\sigma &=& A(E\sigma|a|N\sigma) & \elabel{gnd-sub-A}
\\ X(E|a|R|A)\sigma &=& X(E\sigma|a|R\sigma|N\sigma) & \elabel{gnd-sub-X}
}
Consider predicate $P$ that gives the semantics
of a command program:
\RLEQNS{
  && \HL(P\sigma)
\EQ{\eref{UTCP-NF}, short form}
\\&& \HL((\bigvee_n X_n)\sigma)
\EQ{substitution}
\\&& \HL(\bigvee_n (X_n\sigma))
\EQ{\eref{HL-or-distr}}
\\&& \bigvee_n \HL(X_n\sigma)
\EQ{Expand $X$}
\\&& \bigvee_n \HL((I_n \land X(E_n|a_n|R_n|A_n)))\sigma)
\EQ{substitution, \eref{gnd-sub-X}}
\\&& \bigvee_n \HL(I_n\sigma \land X(E_n\sigma|a_n|R_n\sigma|A_n\sigma))
\EQ{substitution outcomes ground, \eref{HL-I-and-X}}
\\&& \bigvee_n a_n
}
The overall outcome is independent of $\sigma$,
and so the law follows.


\newpage
\subsection{Lemma \textsf{gnd-subs-seq-distr}}

\RLEQNS{
  && (P\seq Q)\sigma
\EQ{\eref{UTP-seq-def}}
\\&& (\exists s_m,ls_m @
       P[s_m,ls_m/s',ls']
       \land
       Q[s_m,ls_m/s,ls])\sigma
\EQ{$\sigma$ does not mention $s_m$, $ls_m$}
\\&& \exists s_m,ls_m @
       P[s_m,ls_m/s',ls']\sigma
       \land
       Q[s_m,ls_m/s,ls]\sigma
\EQ{$\sigma$ is ground, does not mention $s,s',ls,ls'$}
\\&& \exists s_m,ls_m @
       (P\sigma)[s_m,ls_m/s',ls']
       \land
       (Q\sigma)[s_m,ls_m/s,ls]
\EQ{\eref{UTP-seq-def}}
\\&& (P\sigma) \seq (Q\sigma)
}
In particular, this means that $(P^i)\sigma = (P\sigma)^i$.


\newpage
\subsection{Proof of \textsf{;;-l-unit}}

We shall start with an easy proof to smoke out initial issues
\[
  \cskip \cseq P = P
\]
This is the same as
\[
  \HL(\cskip \cseq P) = \HL(P)
\]

We start with the LHS:
\RLEQNS{
  && \HL(\cskip \cseq P)
\EQ{\eref{sem:seq}}
\\&& \HL( [in|\ell_g|out]
           \land ( \cskip[\g{:1},\ell_g/g,out]
                   \lor
                   P[\g{:2},\ell_g/g,in]
                 )
        )
\EQ{\eref{NF-skip}}
\\&& \HL( [in|\ell_g|out]
           \land ( ([in|out] \land A(in|ii|out))[\g{:1},\ell_g/g,out]
                   \lor
                   P[\g{:2},\ell_g/g,in]
                 )
        )
\EQ{substitution}
\\&& \HL( [in|\ell_g|out]
           \land ( ([in|\ell_g] \land A(in|ii|\ell_g))
                   \lor
                   P[\g{:2},\ell_g/g,in]
                 )
     )
\EQ{$\land$-$\lor$-distr}
\\&& \HL( [in|\ell_g|out] \land [in|\ell_g] \land A(in|ii|\ell_g)
           \lor
          [in|\ell_g|out] \land P[\g{:2},\ell_g/g,in]
        )
\EQ{$[in|\ell_g|out] \implies [in|\ell_g]$}
\\&& \HL( [in|\ell_g|out] \land A(in|ii|\ell_g)
           \lor
          [in|\ell_g|out] \land P[\g{:2},\ell_g/g,in]
        )
\EQ{\eref{HL-or-distr}}
\\&& \HL( [in|\ell_g|out] \land A(in|ii|\ell_g) )
     \lor
     \HL( [in|\ell_g|out] \land P[\g{:2},\ell_g/g,in] )
\EQ{\eref{HL-I-and-A}}
\\&& ii
     \lor
     \HL( [in|\ell_g|out] \land P[\g{:2},\ell_g/g,in] )
\EQ{Normal form for $P$, substitution, $\land$-$\lor$-distr}
\\&& ii
     \lor
     \HL( \bigvee_p ([in|\ell_g|out] \land X_p[\g{:2},\ell_g/g,in] ))
\EQ{\eref{HL-or-distr}}
\\&& ii
     \lor
     \bigvee_p \HL( [in|\ell_g|out] \land X_p[\g{:2},\ell_g/g,in] )
}
Useful Lemmas:
\RLEQNS{
  \HL(P) &=& ii \lor \HL(P)
\\ \HL(X\gamma_1) &=& \HL(X\gamma_2)
}


\newpage
\subsection{Proof: Parallel is commutative}

\RLEQNS{
   P \parallel Q
   &=&
   Q \parallel P
   & \elabel{$||$-comm}
}

We apply $\HL$ to each side and calculate.

LHS:
\RLEQNS{
  && \HL(P \parallel Q)
\EQ{\eref{sem:par}, using shorthand for invariant and substitutions}
\\&& \HL(~I_\parallel \land {}
\\&& \quad\W(\quad\phlor A(in|ii|\ell_{g1},\ell_{g2})
\\ && \quad\qquad {}\lor
   P\gamma_1
\\ && \quad\qquad {}\lor
    Q\gamma_2
\\ && \quad\qquad {}\lor
   A(\ell_{g1:},\ell_{g2:}|ii|out)~)~)
\EQ{\eref{W-as-NDC}, splitting out stuttering}
\\&& \HL(~I_\parallel  \land {}
\\&& \quad(~\Skip \lor\bigvee_{i \in 1\dots}
\\ && \qquad\qquad (~ A(in|ii|\ell_{g1},\ell_{g2})
\\ && \qquad\qquad {}\lor
   P\gamma_1
\\ && \qquad\qquad {}\lor
    Q\gamma_2
\\ && \qquad\qquad {}\lor
   A(\ell_{g1:},\ell_{g2:}|ii|out)~)^i~)~)
\EQ{Drive invariant in}
\\&& \HL(~ I_\parallel  \land\Skip
      \lor\bigvee_{i \in 1\dots}
\\ && \qquad\qquad I_\parallel  \land(~ A(in|ii|\ell_{g1},\ell_{g2})
\\ && \qquad\qquad {}\lor
   P\gamma_1
\\ && \qquad\qquad {}\lor
    Q\gamma_2
\\ && \qquad\qquad {}\lor
   A(\ell_{g1:},\ell_{g2:}|ii|out)~)^i~)~)
\EQ{Normal forms for $P$ and $Q$, with substitutions pushed inside}
\\&& \HL(~ I_\parallel  \land\Skip
      \lor\bigvee_{i \in 1\dots}
\\ && \qquad\qquad I_\parallel  \land(~ A(in|ii|\ell_{g1},\ell_{g2})
\\ && \qquad\qquad {}\lor
   \bigvee_p (Y_p\gamma_1)
\\ && \qquad\qquad {}\lor
    \bigvee_q (Z_q\gamma_2)
\\ && \qquad\qquad {}\lor
   A(\ell_{g1:},\ell_{g2:}|ii|out)~)^i~)~)
}
We pause to note that $Y_{p_0} I_P \land \Skip$
and $Z_{q_0} = I_Q \land \Skip$.
So the $(\dots)^i$ term collapses into a big disjunction
of $X$s.
Some of these $X$s will result from sequential compositions
of several $Y_p\gamma_1$ and $Z_q\gamma_2$, for possibly
many different $p$ and $q$ indices.
Consider such an arbitrary $X_n$, other than terms of the form $I\land\Skip$.
It will be the result of sequentially composing one or more $Y_p\gamma_1$,
$Z_q\gamma_q$, $A(in|ii|\ell_{g1},\ell_{g2})$ and $A(\ell_{g1:},\ell_{g2:}|ii|out)$.
We need to show that we can get exactly the same overall results
if we swap $\gamma_1$ and $\gamma_2$.

We can abstract such a composition as a sequence of indices,
where we use indices $a_s$ and $a_m$ to denote the control flow actions.
So sequence $a_s,p_1,q_3,p_1$
denotes the composition
\[
A(in|ii|\ell_{g1},\ell_{g2}) ; Y_{p_1}\gamma_1 ; Z_{q_3}\gamma_2 ; Y_{p_1}\gamma_1.
\]
We need to show that nothing changes if we swap these two gammas,
but that other gammas can break things.

\subsection{STOP PRESS}

The key property we may need to prove the laws is the following:

For any program $P$, during its execution:
\begin{itemize}
  \item the only labels it removes from $ls$ are in $\setof{in} \cup labs(g)$
  \item the only labels it adds to $ls$ are in $labs(g) \cup{out}$
\end{itemize}
This should be provable by induction.
