\section{Laws}\label{sec:laws}

We expect our language to obey the following laws
(\textbf{\emph{SCRAPED FROM fPAISEAN hUTCP-TODO}}):

\def\E{\mathcal{E}}
\def\B{\mathcal{B}}
\def\M{\mathcal{M}}
\def\bpre{\sqsubseteq_{\M}}
\def\bequiv{\equiv_{\M}}
\def\spre{\leq_{\M}}
\def\sequiv{=_{\M}}
\def\trans{\rightarrow}
\def\rtrans{\trans^{*}}
\def\T{\mathcal{T}}
\def\ClsTrc{\power^\dagger(\Sigma^+)}
\def\tclose#1{\setof{{#1}}^{\dagger}}
\def\TT#1{\T\sem{{#1}}}
\def\override{\oplus}
\def\tpre{\sqsubseteq_{\T}}
\def\tequiv{\equiv_{\T}}
\def\wtrans{\trans^\omega}

\def\bor{\textbf{ or }}

\def\wdo{\circledast}
\def\lseq{\mathrel{\mathbf{;\!;}}}
\def\lcond#1{\mathrel{\blacktriangleleft  {#1}  \blacktriangleright}}
\def\llawnamebr{\mbox{\raisebox{2pt}{${\scriptscriptstyle\langle\!\langle\cdot}$}}}
\def\rlawnamebr{\mbox{\raisebox{2pt}{${\scriptscriptstyle\cdot\rangle\!\rangle}$}}}
\def\mkelabel#1{\textsf{\llawnamebr{#1}\rlawnamebr}}
\def\elab#1#2{\mkelabel{#1}\label{#2}}
\def\elabel#1{\elab{#1}{#1}}
\def\ecite#1{\mkelabel{#1}}
\def\eref#1{\mkelabel{#1, p\pageref{#1}}}

\RLEQNS{
   P \lseq ( Q \lseq R) &=& ( P \lseq Q ) \lseq R
   & \elab{$\lseq$-assoc}{;;-assoc}
\\ Idle \lseq P &=& P & \elab{$\lseq$-l-unit}{;;-l-unit}
\\ P \lseq Idle &=& P & \elab{$\lseq$-Q-unit}{;;-r-unit}
\\ P \parallel Q &=& Q \parallel P & \elab{$\parallel$-comm}{||-comm}
\\ P \parallel ( Q \parallel R) &=& ( P \parallel Q ) \parallel R
   & \elab{$\parallel$-assoc}{||-assoc}
\\ P \lcond{true} Q &=& P & \elab{$\lcond{true}$}{lcond-true}
\\ P \lcond{false} Q &=& Q & \elab{$\lcond{false}$}{lcond-false}
\\ c \wdo P &=& (P \lseq c \wdo P) \lcond c Idle
   & \elab{$\wdo$-unroll}{wdo-unroll}
}

\def\bnew#1#2{\nu {#1}={#2} @}
\def\hide{\backslash}
\def\bskip{skip}
\def\bseq{\lseq}
\def\bpar{\parallel}
\def\bcond#1{\lcond{#1}}
\def\bdo{\wdo}
\def\bawait{\mathbin{!}}

Here $=$ and $\sqsubseteq$ correspond to $\tequiv$ and $\tpre$, respectively,
and assuming that assignment is atomic:
\RLEQNS{
   \bskip &=& i:=i
\\ \bskip \bseq c           & = c = & c \bseq skip
\\ (c_1 \bseq c_2) \bseq c_3 &=& c_1 \bseq (c_2 \bseq c_3)
\\ c \bpar \bskip &=& c
\\ c_1 \bpar c_2 &=& c_2 \bpar c_1
\\ (c_1 \bpar c_2) \bpar c_3 &=& c_1 \bpar (c_2 \bpar c_3)
\\ c_1 \bseq (c_2 \bpar c_3) &\sqsubseteq& (c_1 \bseq c_2) \bpar c_3
\\ c_1 \bseq c_2 &\sqsubseteq& c_1 \bpar c_2
\\ (c_1 \bcond b c_2) \bseq c_3 &=& (c_1 \bseq c_3) \bcond c (c_2 \bseq c_3)
\\ c_1 \bcond{b_1 \land b_2} c_2
   &\sqsubseteq&
   (c_1 \bcond{b_2} c_2) \bcond{b_1} c_2
\\ b \bdo c &=& (c \bseq b \bdo c) \cond b \bskip
\\ (b_1 \land b_2) \bawait c &=& b_1 \bawait (b_2 \bawait c)
\\ false \bawait c &\sqsubseteq& c'
\\ false \bawait c &=& true \bdo \bskip
}
If expression evaluation is deterministic:
\RLEQNS{
   i:=e_2[e_1/i] &\sqsubseteq& i:=e_1 \bseq i:=e_2
}
If assignment is \emph{not} atomic:
\RLEQNS{
   \bskip &\sqsubseteq& i:=i
}
If we allow infinite \emph{fair} traces then some of the laws above break/alter as follows:
\RLEQNS{
   c_1 \bseq (c_2 \bpar c_3) &\sqsubseteq& (c_1 \bseq c_2) \bpar c_3
   & \say{only if $c_1$ is loop-free}
\\ false \bawait c &\neq& true \bdo \bskip
\\ true \bdo \bskip &\not\sqsubseteq& c'
}
A loop for $c_1$ in the first case results in inequality because
of the fairness assumption which will force some $c_3$ events to occur
on the rhs, even if $c_1$ is non-terminating,
whereas the lhs will only ever produce $c_1$'s ever-growing trace.

\RLEQNS{
   \TT{c_1 \bor c_2} &=& \TT{c_1} \cup \TT{c_2}
\\ c \sqsubseteq c' &\equiv& (c \bor c') = c'
\\ i_1:=e_1 \bpar i_2:=e_2
   &=&
   (i_1:=e_1 \bseq i_2:=e_2) \bor (i_2:=e_2 \bseq i_1:=e_1)
}
The latter holding for atomic assignment.

\subsection{Finding Bisimulations}


Idea: define the desired bisimulations in such a way that
we can manipulate them with the same kind of substitutions
as found in the semantics.

\subsubsection{\protect{$a \parallel b = b \parallel a$}}

Consider a simple example: prove that $a \parallel b = b \parallel a$,

\RLEQNS{
  && a \parallel b
\EQ{defn of $\parallel$ }
\\&&      in \arr{ii} \setof{\ell_{g1},\ell_{g2}}
     \lor \setof{\ell_{g1:},\ell_{g2:}} \arr{ii} out
\\ &\lor&
   a[g_{1::},\ell_{g1},\ell_{g1:}/g,in,out]
   \lor b[g_{2::},\ell_{g2},\ell_{g2:}/g,in,out]
\EQ{defn of $a$ and $b$}
\\&&      in \arr{ii} \setof{\ell_{g1},\ell_{g2}}
     \lor \setof{\ell_{g1:},\ell_{g2:}} \arr{ii} out
\\ &\lor&
   (in \arr a out)[g_{1::},\ell_{g1},\ell_{g1:}/g,in,out]
   \lor (in \arr b out)[g_{2::},\ell_{g2},\ell_{g2:}/g,in,out]
\EQ{substitutions, as per above}
\\&& in \arr{ii} \setof{\ell_{g1},\ell_{g2}}
     \lor
     \ell_{g1} \arr a \ell_{g1:}
     \lor
     \ell_{g2} \arr b \ell_{g2:}
     \lor
     \setof{\ell_{g1:},\ell_{g2:}} \arr{ii} out
\\
\\&& b \parallel a
\EQ{Above calculations with $a$ and $b$ swapped}
\\&& in \arr{ii} \setof{\ell_{g1},\ell_{g2}}
     \lor
     \ell_{g1} \arr b \ell_{g1:}
     \lor
     \ell_{g2} \arr a \ell_{g2:}
     \lor
     \setof{\ell_{g1:},\ell_{g2:}} \arr{ii} out
}

If we define%
\footnote{If is a label is not mentioned, it maps to itself}
the following bijection on labels:
\[
  \ell_{g1} \mapsto \ell{g2}
, \ell_{g1:} \mapsto \ell{g2:}
, \ell_{g2} \mapsto \ell{g1}
, \ell_{g2:} \mapsto \ell{g1:}
\]
we can show that the two sides are isomorphic modulo that bijection.
Interestingly, this isomorphism works if we define the following
generator bijection $g1 \rightleftharpoons g2$.

\subsubsection{\protect{$(a \cseq b)\parallel c = c \parallel (a \cseq b)$}}

Now, a bit more ambitious---consider
\[ (a \cseq b)\parallel c
    = c \parallel (a \cseq b)
\]

LHS:
\RLEQNS{
   && (a \cseq b)\parallel c
\EQ{defn $\parallel$}
\\&& in \arr{ii} \setof{\ell_{g1},\ell_{g2}}
      \lor \setof{\ell_{g1:},\ell_{g2:}} \arr{ii} out \lor {}
\\ &&
   (a \cseq b)[g_{1::},\ell_{g1},\ell_{g1:}/g,in,out]
   \lor c[g_{2::},\ell_{g2},\ell_{g2:}/g,in,out]
\EQ{defn $\cseq$}
\\&& in \arr{ii} \setof{\ell_{g1},\ell_{g2}}
      \lor \setof{\ell_{g1:},\ell_{g2:}} \arr{ii} out \lor {}
\\ &&
   (a[g_{:1},\ell_g/g,out] \lor b[g_{:2},\ell_g/g,in])[g_{1::},\ell_{g1},\ell_{g1:}/g,in,out]
\\&& {}\lor c[g_{2::},\ell_{g2},\ell_{g2:}/g,in,out]
\EQ{atomic substitution}
\\&& in \arr{ii} \setof{\ell_{g1},\ell_{g2}}
      \lor \setof{\ell_{g1:},\ell_{g2:}} \arr{ii} out \lor {}
\\ &&
   (in \arr a \ell_g \lor \ell_g \arr b out)[g_{1::},\ell_{g1},\ell_{g1:}/g,in,out]
\\&& {}\lor \ell_{g2} \arr c\ell_{g2:}
\EQ{substitution}
\\&& in \arr{ii} \setof{\ell_{g1},\ell_{g2}}
      \lor \setof{\ell_{g1:},\ell_{g2:}} \arr{ii} out \lor {}
\\ &&
   \ell_{g1} \arr a \ell_{g1::}
   \lor \ell_{g1::} \arr b \ell_{g1:}
   \lor \ell_{g2} \arr c\ell_{g2:}
}

RHS:
\RLEQNS{
  && c \parallel (a \cseq b)
\EQ{defn $\parallel$}
\\&& in \arr{ii} \setof{\ell_{g1},\ell_{g2}}
      \lor \setof{\ell_{g1:},\ell_{g2:}} \arr{ii} out \lor {}
\\ &&
   c[g_{1::},\ell_{g1},\ell_{g1:}/g,in,out]
   \lor (a \cseq b)[g_{2::},\ell_{g2},\ell_{g2:}/g,in,out]
\EQ{defn $\cseq$}
\\&& in \arr{ii} \setof{\ell_{g1},\ell_{g2}}
      \lor \setof{\ell_{g1:},\ell_{g2:}} \arr{ii} out \lor {}
\\ &&
   c[g_{1::},\ell_{g1},\ell_{g1:}/g,in,out] \lor {}
\\&& (a[g_{:1},\ell_g/g,out] \lor b[g_{:2},\ell_g/g,in])[g_{2::},\ell_{g2},\ell_{g2:}/g,in,out]
\EQ{atomic substitution}
\\&& in \arr{ii} \setof{\ell_{g1},\ell_{g2}}
      \lor \setof{\ell_{g1:},\ell_{g2:}} \arr{ii} out \lor {}
\\ &&
   \ell_{g1} \arr c \ell_{g1:} \lor {}
\\&& (in \arr a \ell_g \lor \ell_g \arr b out)[g_{2::},\ell_{g2},\ell_{g2:}/g,in,out]
\EQ{substitution}
\\&& in \arr{ii} \setof{\ell_{g1},\ell_{g2}}
      \lor \setof{\ell_{g1:},\ell_{g2:}} \arr{ii} out \lor {}
\\ &&
   \ell_{g1} \arr c \ell_{g1:}
   \lor \ell_{g2} \arr a \ell_{g2::}
   \lor \ell_{g2::} \arr b \ell_{g2:}
}

Reprise LHS:
\[
in \arr{ii} \setof{\ell_{g1},\ell_{g2}} \lor
\setof{\ell_{g1:},\ell_{g2:}} \arr{ii} out \lor
\ell_{g1} \arr a \ell_{g1::} \lor
\ell_{g1::} \arr b \ell_{g1:} \lor
\ell_{g2} \arr c\ell_{g2:}
\]
Reprise RHS:
\[
in \arr{ii} \setof{\ell_{g1},\ell_{g2}} \lor
\setof{\ell_{g1:},\ell_{g2:}} \arr{ii} out \lor
\ell_{g1} \arr c \ell_{g1:} \lor
\ell_{g2} \arr a \ell_{g2::} \lor
\ell_{g2::} \arr b \ell_{g2:}
\]
Reorder LHS:
\[
in \arr{ii} \setof{\ell_{g2},\ell_{g1}} \lor
\setof{\ell_{g2:},\ell_{g1:}} \arr{ii} out \lor
\ell_{g2} \arr c\ell_{g2:} \lor
\ell_{g1} \arr a \ell_{g1::} \lor
\ell_{g1::} \arr b \ell_{g1:}
\]
We immediately see the following bijection: $g1 \leftrightharpoons g2$.

\subsubsection{\protect{$(a \cseq b) \cseq c   = a \cseq (b \cseq c)$}}

Now we look at the following:
\[
  (a \cseq b) \cseq c   = a \cseq (b \cseq c)
\]

LHS:
\RLEQNS{
  && (a \cseq b) \cseq c
\EQ{defn $\cseq$}
\\&& (a \cseq b)[g_{:1},\ell_g/g,out] \lor c[g_{:2},\ell_g/g,in]
\EQ{defn $\cseq$}
\\&& (a[g_{:1},\ell_g/g,out] \lor b[g_{:2},\ell_g/g,in])[g_{:1},\ell_g/g,out]
     \lor c[g_{:2},\ell_g/g,in]
\EQ{atomic substitution}
\\&& (in \arr a out \lor \ell_g \arr b out)[g_{:1},\ell_g/g,out]
     \lor \ell_g \arr c out
\EQ{substitution}
\\&& in \arr a \ell_{g:1} \lor \ell_{g:1} \arr b \ell_g \lor \ell_g \arr c out
}

% C[g_{:1},\ell_g/g,out] \lor D[g_{:2},\ell_g/g,in]
RHS:
\RLEQNS{
  && a \cseq (b \cseq c)
\EQ{defn $\cseq$}
\\&& a[g_{:1},\ell_g/g,out] \lor (b \cseq c)[g_{:2},\ell_g/g,in]
\EQ{defn $\cseq$}
\\&& a[g_{:1},\ell_g/g,out] \lor
  (b[g_{:1},\ell_g/g,out] \lor c[g_{:2},\ell_g/g,in])[g_{:2},\ell_g/g,in]
\EQ{atomic substitution}
\\&& in \arr a \ell_g \lor
  (in \arr b \ell_g \lor \ell_g \arr c out)[g_{:2},\ell_g/g,in]
\EQ{substitution}
\\&& in \arr a \ell_g \lor \ell_g \arr b \ell_{g:2} \lor \ell_{g:2} \arr c out
}

We see the following bijection:
$\ell_{g:1} \leftrightharpoons \ell_{g},
 \ell_{g}   \leftrightharpoons \ell_{g:2}
$

\subsubsection{Bijective Substitutions}

Idea: aren't all our substitutions actually bijections?
\begin{eqnarray*}
   \cseq
   &&
   [g_{:1},\ell_g/g,out]
   [g_{:2},\ell_g/g,in]
\\ +
   &&
   [g_{1:},\ell_{g1}/g,in]
   [g_{2:},\ell_{g2}/g,in]
\\ \parallel
   &&
   [g_{1::},\ell_{g1},\ell_{g1:}/g,in,out]
   [g_{2::},\ell_{g2},\ell_{g2:}/g,in,out]
\\ {}^*
   &&
   [g_{:},\ell_g,in/g,in,out]
\end{eqnarray*}
Under reasonable assumptions about the structure and use of everything,
at least.
