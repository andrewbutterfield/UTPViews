\section{Laws}\label{sec:laws}

\subsection{Equality}\label{ssec:equality}

\subsubsection{Hiding Label-Sets}\label{sssec:HL}

For now, we define equality and refinement as follows,
where we have to ``Hide Labels'':
\RLEQNS{
   \HL(P) &\defs& \exists ls,ls' @ P & \elabel{HL-def}
\\ P=Q &\defs& [ \HL(P) \equiv \HL(Q) ] & \elabel{HL-EQ-def}
\\ P \refinedby Q &\defs& [ \HL(Q) \implies \HL(P) ] & \elabel{HL-RFBY-def}
}
As $ls$ and $ls'$ are no longer free in the result of applying $\HL$,
we can consider it as removing those two variables from the alphabet.
The following law is an immediate consequence of the definition:
\RLEQNS{
   \HL(A \lor B) &=& \HL(A) \lor \HL(B) & \elabel{HL-or-distr}
}

Provided $E$ is ground, then:
\RLEQNS{
   \HL(X(E|a|R|A)) &=& a
   & \elabel{HL-X-red}
\\ \HL(A(E|a|N)) &=& a
   & \elabel{HL-A-red}
}
Provided $E[G,I,O/g,in,out]$ is ground, then:
\RLEQNS{
   \HL(X(E|a|R|A)[G,I,O/g,in,out]) &=& a
   & \elabel{HL-X-sub-red}
\\ \HL(A(E|a|N)[G,I,O/g,in,out]) &=& a
   & \elabel{HL-A-sub-red}
}
As all the language semantic definitions boil down to a large
disjunction of $X$ or $A$ predicates and substitutions satisfying
the two groundedness side-conditions above,
we see that the effect of $\HL$ is not just to remove
all free occurrences of $ls$ and $ls'$,
but also has the effect of removing all uses of $g$, $in$ and $out$.
All that remains are the underlying actions on program state.

Considering invariants in the mix, assuming $E$ is ground
and that $E \textbf{ lsat } I$:
\RLEQNS{
   \HL(I \land \Skip) &=&  ii & \elabel{HL-I-and-skip}
\\ \HL(I \land X(E|a|R|A) \land S)
   &=& a \land S
   & \elabel{HL-I-and-X}
\\ \HL(I \land A(E|a|N))
   &=& a
   & \elabel{HL-I-and-A}
}
Here $S$ is a predicate over ground label-set expressions.
We intend to use this law on $X$-components calculated in the
context of invariant $I$, so we always expect $E$ to satisfy $I$.

Assume that $P$ has a normal form as follows,
where $A(a_i)$ is shorthand for $A(E_i|a_i|N_i)$:
\[
   I \land \left( \Skip \lor \bigvee A(a_i) \right)
\]
then
\RLEQNS{
  \HL(P) &=& ii \lor \bigvee a_i
  & \elabel{HL-nf}
}
All the above results generalise to the case were we have a conjunction
of invariants:
\RLEQNS{
   \HL(I \land J \land \Skip) &=& ii
\\ \HL(I \land J \land A(a)) &=& a
}


The key to proving the language laws is the following law of $\HL$
that captures its independence under any \emph{valid} substitution.
\RLEQNS{
   \HL(P[G_1,I_1,O_1/g,in,out])
   &=&
   \HL(P[G_2,I_2,O_2/g,in,out])
   & \elabel{HL-subs-indep}
}
Such substitutions are \emph{valid} provided they satisfy the invariant
$(I_i | labs(G_i) | O_i)$ for $i=1,2$.
This excludes pathological substitutions that mix up input, output
and internal labels, which would alter the resulting normal form.

\subsubsection{Language Laws}\label{ssec:lang-laws}
\RLEQNS{
   \cskip \cseq P &=& P & \elab{$\cseq$-l-unit}{;;-l-unit}
\\ P \cseq \cskip &=& P & \elab{$\cseq$-Q-unit}{;;-r-unit}
\\ P \cseq ( Q \cseq R) &=& ( P \cseq Q ) \cseq R
   & \elab{$\cseq$-assoc}{;;-assoc}
\\ P \parallel Q &=& Q \parallel P & \elab{$\parallel$-comm}{||-comm}
\\ P \parallel ( Q \parallel R) &=& ( P \parallel Q ) \parallel R
   & \elab{$\parallel$-assoc}{||-assoc}
\\ P + Q &=& Q+P
\\ P + ( Q + R) &=& ( P + Q ) + R
\\ P + Q &\refinedby& P
\\ P \sqsubseteq Q &\equiv& (P + Q) = Q
\\ P^* &=& \cskip + P \cseq P^*
\\ P \dcond{true} Q &=& P & \elab{$\dcond{true}$}{lcond-true}
\\ P \dcond{false} Q &=& Q & \elab{$\dcond{false}$}{lcond-false}
\\ (P_1 \dcond c P_2) \cseq P_3 &=& (P_1 \cseq P_3) \dcond c (P_2 \cseq P_3)
\\ c \ddo P &=& (P \cseq c \ddo P) \dcond c \cskip
   & \elab{$\ddo$-unroll}{wdo-unroll}
\\ P_1 \cseq (P_2 \parallel P_3) &\sqsubseteq& (P_1 \cseq P_2) \parallel P_3
\\ P_1 \cseq P_2 &\sqsubseteq& P_1 \parallel P_2
\\ P_1 \dcond{c_1 \land c_2} P_2
   &\sqsubseteq&
   (P_1 \dcond{c_2} P_2) \dcond{c_1} P_2
}

If we have assignments $x:=e$, then the following apply:
\RLEQNS{
   \cskip &=& i:=i & \mbox{atomic assignment}
\\ \cskip &\sqsubseteq& x:=x &\mbox{non-atomic assignment}
\\ x:=e_2[e_1/x] &\sqsubseteq& x:=e_1 \cseq x:=e_2 & e_i \mbox{ deterministic}
\\ x_1:=e_1 \parallel x_2:=e_2
   &=&
   (x_1:=e_1 \cseq x_2:=e_2)
\\ && {} + (x_2:=e_2 \cseq x_1:=e_1)
}
The latter holding for atomic assignment.
