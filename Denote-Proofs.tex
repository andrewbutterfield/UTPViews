\section{Denotation Proofs}\label{sec:denote-proofs}


\subsection{Proof of seq-unit}
\subsection{Proof of seq-assoc}
\subsection{Proof of seq-or-distr}
\subsection{Proof of lor-seq-distr}
\subsection{Proof of skip-or-twice}
\subsection{Proof of GND-and-seq-L}
\subsection{Proof of GND-and-seq-R}
\subsection{Proof of GND-and-seq-both}
\subsection{Proof of GND-seq-GND-is-GND}
\subsection{Proof of seq-s-unit}
\subsection{Proof of seq-gnd-distr}
\subsection{Proof of skip-gamma}
\subsection{Proof of DL-gamma-subst}
\subsection{Proof of LE-gamma-subst}
\subsection{Proof of gnd-sub-closure}
\subsection{Proof of gnd-sub-comp}
\subsection{Proof of sound-sub-closure}

\newpage
\subsection{Proof of DL-seq-distr}

\RLEQNS{
  && \DL(P)\seq \DL(Q)
\EQ{\eref{DL-def}, $D=\setof{in|g|out}$}
\\&& (P \land D) \seq (Q \land D)
\EQ{\eref{UTP-seq-def}}
\\&& \exists ls_m,s_m @
     (P \land D)[ls_m,s_m/ls',s']
     \land
     (Q \land D)[ls_m,s_m/ls,s]
\EQ{substitution}
\\&& \exists ls_m,s_m @
     P[ls_m,s_m/ls',s'] \land D[ls_m,s_m/ls',s']
     \land
     Q[ls_m,s_m/ls,s] \land D[ls_m,s_m/ls,s]
\EQ{$D$ is ground: $\setof{ls,ls',s,s'} \cap \textrm{fv}(D) = \emptyset$}
\\&& \exists ls_m,s_m @
     P[ls_m,s_m/ls',s'] \land D
     \land
     Q[ls_m,s_m/ls,s] \land D
\EQ{shrink scope, $\land$-idem}
\\&& (\exists ls_m,s_m @
       P[ls_m,s_m/ls',s'] \land Q[ls_m,s_m/ls,s]) \land D
\EQ{\eref{UTP-seq-def}}
\\&& (P \seq  Q) \land D
\EQ{\eref{DL-def}}
\\&& \DL(P \seq  Q)
}

\subsection{Proof of loop-as-NDC}
\subsection{Proof of W-monotonic}
\subsection{Proof of W-idempotent}

\newpage
\subsection{Proof of W-seq-W-is-W}

\RLEQNS{
  && \W(P)\seq\W(P)
\EQ{\eref{W-as-NDC}}
\\&& (\bigvee_{i \in \Nat} P^i) \seq (\bigvee_{i \in \Nat} P^i)
\EQ{unroll iterated disjunctions}
\\&& (P^0 \lor P^1 \lor P^2 \lor P^3 \lor \dots)
     \seq
     (P^0 \lor P^1 \lor P^2 \lor P^3 \lor \dots)
\EQ{\eref{seq-0},\eref{seq-i-plus-1}}
\\&& (\Skip \lor P \lor P^2 \lor P^3 \lor \dots)
     \seq
     (\Skip \lor P \lor P^2 \lor P^3 \lor \dots)
\EQ{$\lor$-$\seq$-distr}
\\&&     \Skip \seq (\Skip \lor P \lor P^2 \lor P^3 \lor \dots)
\\&\lor&     P \seq (\Skip \lor P \lor P^2 \lor P^3 \lor \dots)
\\&\lor&   P^2 \seq (\Skip \lor P \lor P^2 \lor P^3 \lor \dots)
\\&\lor&   P^3 \seq (\Skip \lor P \lor P^2 \lor P^3 \lor \dots)
\\&\lor&   \dots
\EQ{$\seq$-$\lor$-distr}
\\&&     \Skip \seq \Skip \lor \Skip \seq P \lor \Skip \seq P^2 \lor \Skip \seq P^3 \lor \dots
\\&\lor&      P \seq \Skip \lor P \seq P \lor P \seq P^2 \lor P \seq P^3 \lor \dots
\\&\lor&   P^2 \seq \Skip \lor P^2 \seq P \lor P^2 \seq P^2 \lor P^2 \seq P^3 \lor \dots
\\&\lor&   P^3 \seq \Skip \lor P^3 \seq P \lor P^3 \seq P^2 \lor P^3 \seq P^3 \lor \dots
\\&\lor&   \dots
\EQ{\eref{seq-unit}, \eref{seq-assoc}, \eref{seq-0}, \eref{seq-i-plus-1}}
\\&&     \Skip \lor P \lor P^2 \lor P^3 \lor \dots
\\&\lor&      P \lor P^2 \seq P^3 \seq P^4 \lor \dots
\\&\lor&   P^2 \lor P^3 \lor P^4 \lor P^5 \lor \dots
\\&\lor&   P^3 \lor P^4 \lor P^5 \lor P^6 \lor \dots
\\&\lor&   \dots
\EQ{$\lor$-assoc, $\lor$-idem}
\\&& \Skip \lor P \lor P^2 \lor P^3 \lor P^4 \lor P^5 \lor P^6 \lor \dots
\EQ{roll repeated disjunctions, \eref{W-as-NDC}}
\\&& \W(P)
}
Note: this proof shows why the stuttering step $\Skip$ is crucial.
Without it we would have that
\[
  (\W(P))^k  = \bigvee_{i \in k \dots} P^i
\]
Whereas an easy corollary of our result is that
\RLEQNS{
  (\W(P))^k  &=& \W(P) & \elabel{W-star-is-W}
}

\subsection{Proof of W-gnd-subst}
\subsection{Proof of W-gnd-and-distr}
\subsection{Proof of A-gamma-subs}




\subsection{Proof of atom-inv-ok}

\RLEQNS{
   \{in|out\}\varsigma
   \land [in|out]\varsigma
   \land A(in|a|out)\varsigma &\implies& [in|out]'\varsigma
   & \elabel{atom-inv-ok}
}

We shall expand the definitions
\RLEQNS{
   \left(\begin{array}{c}
     ls(in) \land a \land ls'=(ls\setminus in\cup out)
  \\\land
  \\ (ls(in)\implies \lnot ls(out))
  \\\land
  \\(ls(out)\implies \lnot ls(in))
  \end{array}\right)
  &\implies&
  \left(\begin{array}{c}
     (ls'(in)\implies \lnot ls'(out))
  \\\land
  \\(ls'(out)\implies \lnot ls'(in))
  \end{array}\right)
}
We note that $(A \implies \lnot B) \land (B \implies \lnot A)$
is equivalent to $(\lnot A \lor \lnot B)$
\RLEQNS{
   \left(\begin{array}{c}
     ls(in) \land a \land ls'=(ls\setminus in\cup out)
  \\\land
  \\ (\lnot  ls(in) \lor  \lnot ls(out))
  \end{array}\right)
  &\implies&
  (\lnot ls'(in) \lor \lnot ls'(out))
}
We now point out that the antecedent is $\false$ if $ls(in)$
is not true, so we assume $ls(in)$ and simplify:
\RLEQNS{
   \left(\begin{array}{c}
     a \land ls'=(ls\setminus in\cup out)
  \\\land
  \\ \lnot ls(out)
  \end{array}\right)
  &\implies&
  (\lnot ls'(in) \lor \lnot ls'(out))
}
We now replace $ls'$ on the RHS with its value from the RHS:
\RLEQNS{
   \left(\begin{array}{c}
     a \land ls'=(ls\setminus in\cup out)
  \\\land
  \\ \lnot ls(out)
  \end{array}\right)
  &\implies&
  \left(\begin{array}{c}
     \lnot ((ls\setminus in\cup out))(in)
  \\ \lor
  \\\lnot((ls\setminus in\cup out))(out)
  \end{array}\right)
}
Set theory
\RLEQNS{
   \left(\begin{array}{c}
     a \land ls'=(ls\setminus in\cup out)
  \\\land
  \\ \lnot ls(out)
  \end{array}\right)
  &\implies&
  \left(\begin{array}{c}
     \lnot \false
  \\ \lor
  \\\lnot \true
  \end{array}\right)
}
QED

\textbf{WE REPEAT ABOVE WITH $\gamma$ TO DEMONSTRATE SUBSTITUTION INDEPENDENCE}


\subsection{Proof of split-inv-ok}

\RLEQNS{
   \{in|g|out\}\varsigma
   \land [in|(\ell_{g1}|\ell_{g1:}),(\ell_{g2}|\ell_{g2:})|out]\varsigma
   && \elabel{split-inv-ok}
\\ {} \land A(in|ii|\ell_{g1},\ell_{g2})\varsigma
   &\implies&
   [in|(\ell_{g1}|\ell_{g1:}),(\ell_{g2}|\ell_{g2:})|out]'\varsigma
}

\subsection{Proof of merge-inv-ok}

\RLEQNS{
   \{in|g|out\}\varsigma
   \land [in|(\ell_{g1}|\ell_{g1:}),(\ell_{g2}|\ell_{g2:})|out]\varsigma
   && \elabel{merge-inv-ok}
\\ {} \land A(\ell_{g1:},\ell_{g2:}|ii|out)\varsigma
   &\implies&
   [in|(\ell_{g1}|\ell_{g1:}),(\ell_{g2}|\ell_{g2:})|out]'\varsigma
}

\subsection{Proof of ndc-1-inv-ok}

\RLEQNS{
   \{in|g|out\}\varsigma
   \land [in|\ell_{g1}|\ell_{g2}|out]\varsigma
   && \elabel{ndc-1-inv-ok}
\\ {} \land A(in|ii|\ell_{g1})\varsigma
   &\implies&
   [in|\ell_{g1}|\ell_{g2}|out]'\varsigma
}

\subsection{Proof of ndc-2-inv-ok}

\RLEQNS{
   \{in|g|out\}\varsigma
   \land [in|\ell_{g1}|\ell_{g2}|out]\varsigma
   && \elabel{ndc-2-inv-ok}
\\ {} \land A(in|ii|\ell_{g2})\varsigma
   &\implies&
   [in|\ell_{g1}|\ell_{g2}|out]'\varsigma
}


\subsection{Proof of ndc-exit-inv-ok}

\RLEQNS{
   \{in|g|out\}\varsigma
   \land [in|\ell_{g1}|out]\varsigma
   && \elabel{ndc-exit-inv-ok}
\\ {} \land A(in|ii|out)\varsigma
   &\implies&
   [in|\ell_{g1}|out]'\varsigma
}

\subsection{Proof of ndc-loop-inv-ok}

\RLEQNS{
   \{in|g|out\}\varsigma
   \land [in|\ell_{g1}|out]\varsigma
   && \elabel{ndc-loop-inv-ok}
\\ {} \land A(in|ii|\ell_{g1})\varsigma
   &\implies&
   [in|\ell_{g1}|out]'\varsigma
}



\subsection{Proof of cond-1-inv-ok}

\RLEQNS{
   \{in|g|out\}\varsigma
   \land [in|\ell_{g1}|\ell_{g2}|out]\varsigma
   && \elabel{cond-1-inv-ok}
\\ {} \land A(in|c \land ii|\ell_{g1})\varsigma
   &\implies&
   [in|\ell_{g1}|\ell_{g2}|out]'\varsigma
}

\subsection{Proof of cond-2-inv-ok}

\RLEQNS{
   \{in|g|out\}\varsigma
   \land [in|\ell_{g1}|\ell_{g2}|out]\varsigma
   && \elabel{cond-2-inv-ok}
\\ {} \land A(in|\lnot c \land ii|\ell_{g2})\varsigma
   &\implies&
   [in|\ell_{g1}|\ell_{g2}|out]'\varsigma
}

\subsection{Proof of cond-exit-inv-ok}

\RLEQNS{
   \{in|g|out\}\varsigma
   \land [in|\ell_{g1}|out]\varsigma
   && \elabel{cond-exit-inv-ok}
\\ {} \land A(in|\lnot c \land ii|out)\varsigma
   &\implies&
   [in|\ell_{g1}|out]'\varsigma
}

\subsection{Proof of cond-loop-inv-ok}

\RLEQNS{
   \{in|g|out\}\varsigma
   \land [in|\ell_{g1}|out]\varsigma
   && \elabel{cond-loop-inv-ok}
\\ {} \land A(in|c \land ii|\ell_{g1})\varsigma
   &\implies&
   [in|\ell_{g1}|out]'\varsigma
}
