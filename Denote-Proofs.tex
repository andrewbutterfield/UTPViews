\section{Denotation Proofs}\label{sec:denote-proofs}


\subsection{Proof of seq-unit}
\subsection{Proof of seq-assoc}
\subsection{Proof of seq-or-distr}
\subsection{Proof of lor-seq-distr}
\subsection{Proof of skip-or-twice}
\subsection{Proof of GND-and-seq-L}
\subsection{Proof of GND-and-seq-R}
\subsection{Proof of GND-and-seq-both}
\subsection{Proof of GND-seq-GND-is-GND}
\subsection{Proof of seq-s-unit}
\subsection{Proof of seq-gnd-distr}
\subsection{Proof of skip-gamma}
\subsection{Proof of DL-gamma-subst}
\subsection{Proof of LE-gamma-subst}
\subsection{Proof of gnd-sub-closure}
\subsection{Proof of gnd-sub-comp}
\subsection{Proof of sound-sub-closure}
\subsection{Proof of loop-as-NDC}
\subsection{Proof of W-monotonic}
\subsection{Proof of W-idempotent}
\subsection{Proof of W-seq-W-is-W}
\subsection{Proof of W-gnd-subst}
\subsection{Proof of W-gnd-and-distr}
\subsection{Proof of A-gamma-subs}


\subsection{Stuff}
This appendix looks like the best place to introduce these shorthands
\RLEQNS{
   ls(\ell) &\defs& \ell \in ls
\\ ls(L) &\defs& L \subseteq ls
\\ ls(\B\ell) &\defs& \ell \notin ls
\\ ls(\B L) &\defs& L \cap ls = \emptyset
}
We define what it means for an atomic action invocation
to satisfy an invariant parameterised on the label type ($Lbl$).
\RLEQNS{
  ls \textbf{ lsat } I \land A(E|a|N) &\implies& ls' \textbf{ lsat } I
}
We can re-formulate this as a following equivalent test:
\RLEQNS{
   A(E|a|N) \textbf{ sat } I_{Lbl}
   &\defs&
   E \textbf{ lsat } I_{Lbl} \land N \textbf{ lsat } I_{Lbl}
}


\subsection{Proof of atom-inv-ok}

\RLEQNS{
   \{in|out\}\varsigma
   \land [in|out]\varsigma
   \land A(in|a|out)\varsigma &\implies& [in|out]'\varsigma
   & \elabel{atom-inv-ok}
}

We shall expand the definitions
\RLEQNS{
   \left(\begin{array}{c}
     ls(in) \land a \land ls'=(ls\setminus in\cup out)
  \\\land
  \\ (ls(in)\implies \lnot ls(out))
  \\\land
  \\(ls(out)\implies \lnot ls(in))
  \end{array}\right)
  &\implies&
  \left(\begin{array}{c}
     (ls'(in)\implies \lnot ls'(out))
  \\\land
  \\(ls'(out)\implies \lnot ls'(in))
  \end{array}\right)
}
We note that $(A \implies \lnot B) \land (B \implies \lnot A)$
is equivalent to $(\lnot A \lor \lnot B)$
\RLEQNS{
   \left(\begin{array}{c}
     ls(in) \land a \land ls'=(ls\setminus in\cup out)
  \\\land
  \\ (\lnot  ls(in) \lor  \lnot ls(out))
  \end{array}\right)
  &\implies&
  (\lnot ls'(in) \lor \lnot ls'(out))
}
We now point out that the antecedent is $\false$ if $ls(in)$
is not true, so we assume $ls(in)$ and simplify:
\RLEQNS{
   \left(\begin{array}{c}
     a \land ls'=(ls\setminus in\cup out)
  \\\land
  \\ \lnot ls(out)
  \end{array}\right)
  &\implies&
  (\lnot ls'(in) \lor \lnot ls'(out))
}
We now replace $ls'$ on the RHS with its value from the RHS:
\RLEQNS{
   \left(\begin{array}{c}
     a \land ls'=(ls\setminus in\cup out)
  \\\land
  \\ \lnot ls(out)
  \end{array}\right)
  &\implies&
  \left(\begin{array}{c}
     \lnot ((ls\setminus in\cup out))(in)
  \\ \lor
  \\\lnot((ls\setminus in\cup out))(out)
  \end{array}\right)
}
Set theory
\RLEQNS{
   \left(\begin{array}{c}
     a \land ls'=(ls\setminus in\cup out)
  \\\land
  \\ \lnot ls(out)
  \end{array}\right)
  &\implies&
  \left(\begin{array}{c}
     \lnot \false
  \\ \lor
  \\\lnot \true
  \end{array}\right)
}
QED

\textbf{WE REPEAT ABOVE WITH $\gamma$ TO DEMONSTRATE SUBSTITUTION INDEPENDENCE}


\subsection{Proof of split-inv-ok}

\RLEQNS{
   \{in|g|out\}\varsigma
   \land [in|(\ell_{g1}|\ell_{g1:}),(\ell_{g2}|\ell_{g2:})|out]\varsigma
   && \elabel{split-inv-ok}
\\ {} \land A(in|ii|\ell_{g1},\ell_{g2})\varsigma
   &\implies&
   [in|(\ell_{g1}|\ell_{g1:}),(\ell_{g2}|\ell_{g2:})|out]'\varsigma
}

\subsection{Proof of merge-inv-ok}

\RLEQNS{
   \{in|g|out\}\varsigma
   \land [in|(\ell_{g1}|\ell_{g1:}),(\ell_{g2}|\ell_{g2:})|out]\varsigma
   && \elabel{merge-inv-ok}
\\ {} \land A(\ell_{g1:},\ell_{g2:}|ii|out)\varsigma
   &\implies&
   [in|(\ell_{g1}|\ell_{g1:}),(\ell_{g2}|\ell_{g2:})|out]'\varsigma
}
