\section{Semantic Calculations}\label{sec:calc}

This semantic theory was validated by a series of test calculations
done to ensure that it was making the right predications about program behaviour.
This typically involved taking small programs with a few atomic actions
and trying to simplify their semantic predicates
down to a non-deterministic choice of atomic action sequences.

This required the development of a UTP calculator tool,
as some of the calculations proved to be very long,
but very mechanical and tedious \cite{conf/utp/Butterfield16}.

It also made it very clear that all program semantics predicates can be written
in a standard normal form.
We will discuss this aspect of the semantics first (\S\ref{ssec:normal-form}),
as it embodies a key simplifying principle.
We then address the issue of computing mumblings, i.e. the (UTP)
sequential composition of $A$ actions (\S\ref{ssec:comp-A}.
Finally, we present results of test calculations,
which turn out to produce our normal forms (\S\ref{ssec:nf-calc}).

\subsection{Normal Form}\label{ssec:normal-form}

All the semantics above have an invariant $I_P$,
zero or more atomic actions ($A_i$),
plus zero or more language sub-component predicates ($P_j$)
assembled into a predicate of the form:
\[
  I_P
  \land
  \DL(\W( A_1 \lor \dots \lor A_m
     \lor
     P_1\varsigma_1 \lor \dots \lor P_n\varsigma_n ))
\]
where each $A_i$ is a basic action that preserves $I_P$,
and $\varsigma_i = [G_i,I_i,O_i/g,in,out]$,
is a sound substitution, so $\setof{I_i|G_i|O_i}$ holds.
We can expand the definitions of $\DL$ and $\W$ to get:
\[
  I_P \land \setof{in|g|out}
  \land
  \bigvee_{i \in \Nat}
    \left( A_1 \lor \dots \lor A_m
          \lor
          P_1\varsigma_1 \lor \dots \lor P_n\varsigma_n
    \right)^i
\]
We can then force the invariants in:
\[
    \bigvee_{i \in \Nat}
      I_P \land \setof{in|g|out}
      \land
       \left( A_1 \lor \dots \lor A_m
             \lor
             P_1\varsigma_1 \lor \dots \lor P_n\varsigma_n
       \right)^i
\]
Each of the $P_1 \dots P_n$ have the same structure.
We now explore the nature of the sequential compositions,
starting with $i=2$.
\RLEQNS{
  && ( A_1 \lor \dots \lor A_m
       \lor
       P_1\varsigma_1 \lor \dots \lor P_n\varsigma_n )
     \seq
     ( A_1 \lor \dots \lor A_m
       \lor
       P_1\varsigma_1 \lor \dots \lor P_n\varsigma_n )
\EQ{$;$-$\lor$-distr}
\\&& A_1 \seq A_1 \lor \dots \lor A_1 \seq A_m
     \lor
     A_1 \seq P_1\varsigma_1 \lor \dots \lor A_1 \seq P_n\varsigma_n
\\&\lor& \vdots
\\&\lor& A_m \seq A_1 \lor \dots \lor A_m \seq A_m
         \lor
         A_m \seq P_1\varsigma_1 \lor \dots \lor A_m \seq P_n\varsigma_n
\\&\lor& P_1\varsigma_1 \seq A_1 \lor \dots \lor P_1\varsigma_1 \seq A_m
         \lor
         P_1\varsigma_1 \seq P_1\varsigma_1 \lor \dots \lor P_1\varsigma_1 \seq P_n\varsigma_n
\\&\lor& \vdots
\\&\lor& P_n\varsigma_n \seq A_1 \lor \dots \lor P_n\varsigma_n \seq A_m
         \lor
         P_n\varsigma_n \seq P_1\varsigma_1 \lor \dots \lor P_n\varsigma_n \seq P_n\varsigma_n
}
If our program is an atomic action, then $m=2$ and $n=0$, so
we have
\[
  A_1 \seq A_1
\]
which means its semantics reduces to
\[
 \bigvee_{i \in \Nat }I_P \land \setof{in|g|out} \land A_1^i
\]
If our program is a composite whose components are atomic,
then the $P_i$ have the above form.
Generalising, it should be easy to see that the semantics of any construct $P$
will ultimately have the form
\[
  \bigvee_p  I_p \land X_p)
\]
Here $p$ is an index from some appropriate set,
determined by the structure of $P$.

The $I_p$ are some conjunction of invariants
that includes $I_P$, the top-level invariant for $P$.
and $X_p$ is a sequence of $A$s sequentially composed together.
By convention, we assume a distinguished index value $p_0$,
and take $I_{p_0} = I_P$ and $X_{p_0} = \Skip$.
To summarise:
\RLEQNS{
   I_{p_0}  &=& I_P  & \elabel{I-p-0}
\\ X_{p_0} &=& \Skip   & \elabel{X-p-0}
\\ P        &=& \bigvee_p  (I_p \land X_p)
   & \elabel{UTCP-NF}
}

In effect the meaning of a program is a big non-deterministic choice
between stuttering, basic actions,
and a specific set of mumblings.
We will make those more precise after we have investigated mumbling in more detail.

The $X_{m_0} =\Skip$ occurs because we have the case $i=0$ to consider above,
and it propagates up.
In effect we have another healthiness condition:
\RLEQNS{
   P &=& \Skip \lor P & \elabel{stutter-anytime}
}


\subsection{Composing $A$s (a.k.a. Mumbling)}\label{ssec:comp-A}

We have a problem in that actions defined using $A(\dots)$
are not closed under standard UTP sequential composition (\eref{UTP-seq-def}).
For example consider the following,
where all three labels are different:
\[
   A(l_1|a|l_2) \seq A(l_2|b|l_3)
\]
The first action requires $l_1$ to be in $ls$ to run,
and if so, removes it and adds in $l_2$.
So the second action can then run immediately afterwards,
and removes $l_2$ and adds in $l_3$.
So the overall outcome is that $l_1$ needs to be in $ls$
for this mumbling to occur, $l_1$ and $l_2$ are removed,
and $l_3$ is added.
We could argue that the net effect is that $l_1$
is removed and $l_3$ is added,
but that assumes that $l_2$ was not present at the start.
If it was, then the net effect would still be,
requiring $l_1$ to be present to start,
removing $l_1$ and $l_2$, and adding in $l_3$.

We cannot describe this as an instance of $A$.
We need a more general form,
where the labels removed are not necessarily
exactly those that had to be present to enable the action.
We call this eXtended atomic action $X$ and it is defined as follows:
\RLEQNS{
   X(E|a|R|A)
   &\defs&
   E \subseteq ls \land a \land ls' = (ls \sminus R) \cup A
   & \elabel{X-def}
}
We can then re-define $A$ in terms of $X$:
\RLEQNS{
   A(E|a|N)
   &=&
   X(E|a|E|N)
   & \elabel{A-alt-def}
}
The key law is one regarding sequentially composing $X$s:
\RLEQNS{
   \lefteqn{X(E_1|a|R_1|A_1) ; X(E_2|b|R_2|A_2)} &&& \elabel{X-then-X}
\\ &=& E_2 \cap (R_1\sminus A_1) = \emptyset
\\ && {} \land
   X(E_1 \cup (E_2\sminus A_1)
       \mid a\seq_s b
       \mid R_1 \cup R_2
       \mid (A_1 \sminus R_2) \cup  A_2)
}
The condition $E_2 \cap (R_1\sminus A_1) = \emptyset$
characterises all those cases were the second $X$ is enabled
immediately after the first $X$ terminates
(i.e., without any outside interference).


From this we immediately get the following laws:
\RLEQNS{
   \lefteqn{A(E_1|a|N_1) \seq A(E_2|a|N_2)} &&& \elabel{A-then-A}
\\ &=& E_2 \cap (E_1\sminus N_1) = \emptyset
\\ && {} \land
       X(E_1 \cup (E_2\sminus N_1)
           \mid a\seq_s b
           \mid E_1 \cup E_2
           \mid (N_1 \sminus E_2) \cup  N_2)
\\
\\ \lefteqn{A(E_1|a|N_1) \seq X(E_2|a|R_2|A_2)} &&& \elabel{A-then-X}
\\ &=& E_2 \cap (E_1\sminus N_1) = \emptyset
\\ & & {} \land
       X(E_1 \cup (E_2\sminus N_1)
           \mid a\seq_s b
           \mid E_1 \cup R_2
           \mid (N_1 \sminus R_2) \cup  A_2)
\\
\\ \lefteqn{X(E_1|a|R_1|A_1) \seq A(E_2|a|N_2)} &&& \elabel{X-then-A}
\\ &=& E_2 \cap (R_1\sminus A_1) = \emptyset
\\ & & {} \land
       X(E_1 \cup (E_2\sminus A_1)
           \mid a\seq_s b
           \mid R_1 \cup E_2
           \mid (A_1 \sminus E_2) \cup  N_2)
}
There are a few cases where two $A$s lead to an $A$ outcome.
One obvious one is where the two actions have no labels in common:
\RLEQNS{
   \lefteqn{(E_1 \cup N_1) \cap (E_2 \cup N_2)=\emptyset} &&& \elabel{disjoint-As}
\\ &\implies&
  A(E_1|a|N_1) \seq A(E_2|b|N_2)
  =
  A(E_1\cup E_2|a\seq_s b|N_1\cup N_2)
}
The other is when the enabling labels of the second
are disjoint from those added by the first:
\RLEQNS{
   \lefteqn{N_1 \cap E_2 =\emptyset} &&& \elabel{A2-not-enabled-by-A1}
\\ &\implies&
  A(E_1|a|N_1) \seq A(E_2|b|N_2)
\\ && {} =
  E_2 \cap E_1 = \emptyset
  \land A(E_1\cup E_2|a\seq_s b|N_1\cup N_2)
}
Many of the latter two cases will be ruled out as impossible in
practise, due to the label invariants.

We can also illustrate what happens when we compose and $X$ or $A$ with
itself:
\RLEQNS{
   X(E|a|R|A) \seq X(E|a|R|A)
   &=&
   (E\sminus A) \cap R = \emptyset
   \land
   X(E|a^2|R|A)
   & \elabel{X-twice}
\\ A(E|a|N) \seq A(E|a|N)
   &=&
   E \subseteq N \land A(E|a^2|N)
   & \elabel{A-twice}
}

Finally, as for $A$,
we note that ground-substitutions can be driven inside $X$:
\RLEQNS{
   X(E|a|R|A)\gamma &=& X(E\gamma|a|R\gamma|A\gamma) & \elabel{X-gamma-subs}
}

\newpage
\subsubsection{Normal Form Revisited}

We can now be much more precise about the normal form,
in that we can now state that the $X$ in \eref{UTCP-NF} above
are really predicates of the form resulting from \eref{X-then-X}:
\RLEQNS{
   P
   &=&
   \bigvee_{p \in pindex}
      (I_p \land X(E_p|a_p|R_p|A_p) \land S_p)
   & \elabel{UTCP-NF-X-and-S}
}
for suitable choices of $pindex$, $I_p$, $E_p$, $a_p$, $R_p$, $A_p$ and $S_p$,
depending on $P$.
In particular, we find that the $E_p$ and $A_p$ all satisfy the relevant
label-set invariants.
The the $S_p$ term  is a ground-expression,
and is statically decidable as $\true$ or $\false$.
So, we only need the terms where they evaluate to true,
and so they need not appear explicitly:
\RLEQNS{
   P
   &=&
   \bigvee_{p \in pindex'}
      (I_p \land X(E_p|a_p|R_p|A_p))
   & \elabel{UTCP-NF-X-only}
}
where $pindex'$ is $pindex$ less those cases where $S_p = \false$.

A common case that emerges when we do calculations
is the need to compute sequential compositions
of terms that are an LE-invariant in conjunction with an $X$-term:
\RLEQNS{
   (I_1 \land X_1) \seq (I_2 \land X_2)
   &=&
   I_1 \land I_2[(ls\sminus R_1)\cup A_1/ls] \land (X_1 \seq  X_2)
   & \elabel{IandX-then-IandX}
\\ &=& I_{12} \land (X_1 \seq X_2)
\\ && \textbf{where } I_{ij} = I_i \land I_j[(ls\sminus R_i)\cup A_i/ls]
}
In essence the invariants come out with the second
one adjusted to accounts for the label changes made by $X_1$.

\subsubsection{Musings on the Normal Form}

For any complete program that does not use iteration constructs,
this expansion is finite.

For any complete program
we find often find that the $X$-terms
can be be simplified, using invariants,
to the $A$-form.
We do not prove this, but note that we have observed it as a result
of test calculations and these lead us to believe it is true in general.

An important law we shall require is that the UTP sequential
composition of two normal forms can itself be expressed
as such a normal form:
\RLEQNS{
   I \land (\Skip\lor\bigvee A_i) \seq J \land (\Skip\lor\bigvee B_j)
   &=&
   I \land J \land (\Skip\lor\bigvee (A_i \seq B_j))
   &\elabel{nf-seq}
}
This is an easy result of the fact that $\seq$ distributes through $\lor$,
and $\Skip$ is a unit for it as well.

\subsection{Normal Form Calculations}\label{ssec:nf-calc}

\subsubsection{Notation}

When we do calculations for small programs
we typically specialise the $\DL$ invariant to just refer
to the labels in use, rather than $g$ in general.
This often means that $DL$ mentions the same labels as occur
in the construct specific label-set $LE$ invariants.
We adopt the following shorthand for those cases:
\RLEQNS{
   \{[L_1|\dots|L_n]\}
   &\defs&
   \{L_1|\dots|L_n\}\land [L_1|\dots|L_n]
   & \elabel{DL-LE-overlap}
}

\subsubsection{Atomic Actions}
\RLEQNS{
   \catom{a}
   &=&
   \{[in|out]\} \land (\Skip \lor A(in|a|out))
   & \elabel{NF-atomic}
\\
\\ \cskip
   &=&
   \{[in|out]\} \land (\Skip \lor A(in|ii|out))
   & \elabel{NF-skip}
}


\subsubsection{Sequential Composition}


\RLEQNS{
   P \cseq Q
   &=& [in|\ell_g|out] \land \dots
   & \elabel{NF-seq}
\\
\\ \catom a \cseq \catom b
   &=& \{[in|\ell_g|out]\} \land {}
  & \elabel{NF-a-seq-b}
\\ && (~\Skip
      \lor
      A(in|a|\ell_g)
      \lor
      A(\ell_g|b|out)
      \lor
      A(in|a\seq b|out)~)
}


\subsubsection{Non-deterministic Choice}

\RLEQNS{
   P + Q
   &=& [in|\ell_{g1}|\ell_{g2}|out] \land \dots & \elabel{NF-NDC}
\\
\\ \catom a + \catom b
   &=& \{[in|\ell_{g1}|\ell_{g2}|out]\} \land {}
   & \elabel{NF-a-or-b}
\\ && \bigvee\left\{
       \begin{array}{l}
          \Skip,
       \\ A(in|ii|\ell_{g1}),
          A(in|ii|\ell_{g2}),
          A(\ell_{g1}|a|out),
          A(\ell_{g2}|b|out),
       \\ A(in|a|out),
          A(in|b|out)
       \end{array}
      \right\}
}


\subsubsection{Parallel Composition}
Here $\#n$ denotes mumblings of length $n$.
\RLEQNS{
   P \parallel Q
   &=& [in\mid([\ell_{g1}|\ell_{g1:}],[\ell_{g2}|\ell_{g2:}])\mid out]
       \land \dots
   & \elabel{NF-par}
\\
\\ \catom a \parallel \catom b
   &=& \{in\mid \ell_{g1},\ell_{g1:},\ell_{g2},\ell_{g2:}\mid out\}
        \land {}
\\ && [in\mid([\ell_{g1}|\ell_{g1:}],[\ell_{g2}|\ell_{g2:}])\mid out]
       \land {}
   & \elabel{NF-a-par-b}
\\ && \bigvee\left\{
       \begin{array}{l}
          \Skip,
       \\ A(in|ii|\ell_{g1},\ell_{g2}),
          A(\ell_{g1:},\ell_{g2:}|ii|out),
       \\ A(\ell_{g1}|a|\ell_{g1:}),
          A(\ell_{g2}|b|\ell_{g2:}),
       \\ A(in|a|\ell_{g1:},\ell_{g2}),
          A(in|b|\ell_{g2:},\ell_{g1}),
       \\ A(\ell_{g1},\ell_{g2}|b;a|\ell_{g1:},\ell_{g2:}),
          A(\ell_{g1},\ell_{g2}|a;b|\ell_{g1:},\ell_{g2:}),
       \\ A(\ell_{g2:},\ell_{g1}|a|out),
          A(\ell_{g1:},\ell_{g2}|b|out),
      \\ A(in|b;a|\ell_{g1:},\ell_{g2:}),
         A(in|a;b|\ell_{g1:},\ell_{g2:}),
      \\ A(\ell_{g1},\ell_{g2}|b;a|out),
         A(\ell_{g1},\ell_{g2}|a;b|out),
      \\ A(in|b;a|out),
         A(in|a;b|out)
       \end{array}
      \right\}
}

\subsubsection{Non-deterministic Iteration}
The actions that denote a terminating
run with zero or more complete iterations
are \textbf{emboldended} below.
\RLEQNS{
   P^* &=& [in|\ell_g|out] \land \dots
   & \elabel{NF-iter}
\\
\\ \catom a^*
   &=& \{[in|\ell_g|out]\} \land {}
   & \elabel{NF-a-star}
\\ && \bigvee\left\{
       \begin{array}{l}
          \Skip,
       \\ \mathbf{A(in|ii|out)},
          A(in|ii|\ell_g),
          A(\ell_g|a|in)
       \\ A(\ell_g|a|out),
          A(\ell_g|a|\ell_g)
          , A(in|a|in)
       \\ \mathbf{A(in|a|out)},
          A(in|a|\ell_g),
          A(\ell_g|a^2|in)
       \\ A(\ell_g|a^2|out,
          A(\ell_g|a^2|\ell_g),
          A(in|a^2|in)
       \\ \mathbf{A(in|a^2|out)},
          A(in|a^2|\ell_g),
          A(\ell_g|a^3|in)
       \\ A(\ell_g|a^3|out),
          A(\ell_g|a^3|\ell_g),
          A(in|a^3|in),
       \\ \vdots
       \end{array}
      \right\}
}
An obvious recurrent pattern (based on mumbling depth) has emerged:
\RLEQNS{
   &&  \mathbf{A(in|a^n|out)} \quad A(in|a^n|\ell_g) \quad A(\ell_g|a^{n+1}|in) &\#2n+1
\\ &&  A(\ell_g|a^{n+1}|out) \quad A(\ell_g|a^{n+1}|\ell_g) \quad A(in|a^{n+1}|in) &\#2n+2
}
We can also identify a pattern based on number of $a$s performed:
\RLEQNS{
   \alpha_0 && \mathbf{A(in|ii|out)} \quad A(in|ii|\ell_g)
\\ \alpha_1 && A(\ell_g|a|in) \quad A(\ell_g|a|out) \quad A(\ell_g|a|\ell_g) \quad A(in|a|in)
\\          && \mathbf{A(in|a|out)} \quad A(in|a|\ell_g)
\\ \alpha_2 && A(\ell_g|a^2|in) \quad A(\ell_g|a^2|out) \quad A(\ell_g|a^2|\ell_g) \quad A(in|a^2|in)
\\          && \mathbf{A(in|a^2|out)} \quad A(in|a^2|\ell_g)
\\
\\ \alpha_n && A(\ell_g|a^n|in) \quad A(\ell_g|a^n|out) \quad A(\ell_g|a^n|\ell_g) \quad A(in|a^n|in)
\\          && \mathbf{A(in|a^n|out)} \quad A(in|a^n|\ell_g)
}


\newpage
\subsection{Essential Properties?}


Looking forward we will want some key algebraic properties:
\RLEQNS{
   A(E|a|N) \textbf{ sat } I
   &=&
   A(E|a|N)\gamma \textbf{ sat } I\gamma
   , \quad \mbox{arbitrary }\gamma
\\ ((ls\sminus R)\cup A)(S)
   &=&
   ls(F(R,A,S)) \land P(R,A,S)
}


When $P\gamma$ executes:
\begin{description}
  \item[\elabel{P-removes-in}]
    it will remove $in\gamma$ from $ls$ at some point;
  \item[\elabel{P-removes-g}]
    the only other labels it removes are in $labs(g\gamma)$;
  \item[\elabel{P-adds-out}]
    it adds $out\gamma$ into $ls$ at some point;
  \item[\elabel{P-adds-g}]
    the only other labels it adds are in $labs(g\gamma)$;
  \item[\elabel{P-ignores-rest}]
    and its behaviour is not affected by any label not
    in
    \\$\setof{in\gamma,out\gamma} \cup labs(g\gamma)$.
\end{description}
We can summarise this as follows:
\begin{itemize}
  \item $P\gamma$ actions are enabled by labels in $in\gamma,g\gamma$
  \item $P\gamma$ actions enable actions enabled by $g\gamma,out\gamma$
  \item Remember that the following invariant holds: $[in\gamma|g\gamma|out\gamma]$
\end{itemize}





\RLEQNS{
   X(E|a|R|A)\gamma &=& X(E\gamma|a|R\gamma|A\gamma)
   & \elabel{X-gnd-subs}
\\ (\bigvee_n X_n)^2
   &=& \bigvee_{n_1,n_2} X_{n_1} \seq X_{n_2}
   & \elabel{NF-squared}
\\ (\bigvee_p Y_p \lor \bigvee_q Z_q)^2
     &=& \bigvee_{p_1,p_2}  (Y_{p_1} \seq Y_{p_2})
\\&\lor& \bigvee_{q_1,q_2}  (Z_{q_1} \seq Z_{q_2})
\\&\lor& \bigvee_{p,q}  (Y_p \seq Z_q)
\\&\lor& \bigvee_{q,p}  (Z_q \seq Y_p)
  & \elabel{NF+NF-squared}
\\ &=& \bigvee_r X_r,\textrm{ for appropriate } r
  & \elabel{NF-squared-alt}
\\ (\bigvee_p Y_p \lor \bigvee_q Z_q)^n
     &=& \bigvee_r X_r,\textrm{ for appropriate } r
  & \elabel{NF-powered}
\\ \W(\Skip \lor \bigvee_p Y_p \lor \bigvee_q Z_q)
     &=& \bigvee_r X_r,\textrm{ for appropriate } r
  & \elabel{W-of-NDC-is-NF}
}
Add in X-then-X and related properties.

Nice to have:
the combined invariants ensure that the normal form
has the form:
\[
  P = \bigvee_n A(E_n|a_n|N_n)  \qquad \elabel{NF-only-As}
\]
Is this the case?


\newpage
\subsection{Miscellaneous}

More stuff for proofs section.


We shall first demonstrate that the stuttering $\Skip$
introduced in the atomic action semantics
propagates up into the program composites,
so that, for example, we could have defined sequential composition
as:
\RLEQNS{
   P \cseq Q
   &\defs&
   [in|\ell_g|out]\land
   \W(\Skip \lor P[g_{:1},\ell_g/g,out] \lor Q[g_{:2},\ell_g/g,in])
   & \elabel{sem:seq-alt}
}
We argue by induction over program syntax.

What we shall demonstrate is that, for all commands $C$
with a definition $C = \W(P)$, for $P$ related to $C$ as appropriate,
that the following predicates are all equal:
\RLEQNS{
  \W(\Skip \lor P) \quad = & \W(P)& = \quad \Skip \lor \W(P)
 & \elabel{W-prog-stutters}
}
The base case is easy, here $c$ is $\catom a$,
and $P$ is $\Skip \lor A(in|a|out)$:
\RLEQNS{
  && \W(\Skip \lor (\Skip \lor A(in|a|out)))
\EQ{$\lor$-assoc, $\lor$-idem}
\\&& \W(\Skip \lor A(in|a|out))
\EQ{\eref{sem:atomic}}
\\&& \catom a
}
For the inductive step we assume the hypothesis
for command $C$ and its semantics predicate  $C = \W(P)$,
that:
\RLEQNS{
   \W(P) &=& \W(\Skip \lor P) &\elabel{hyp:P-stutters}
}
We now make use of the following deduction:
\RLEQNS{
  && \W(\Skip \lor Q)
\EQ{loop unrolling}
\\&& (\Skip\lor Q)\seq(\Skip\lor Q)\seq(\Skip\lor Q)\seq(\Skip\lor Q)\seq\dots
\EQ{$\lor$-$\seq$-distr, diagonalisation, removing identical terms}
\\&& \bigvee_{i \in \Nat} \Skip\seq Q^i & \elabel{W-stutter-is-NDC}
}
An immediate consequence of this is that
\RLEQNS{
  \W(\Skip \lor Q) &=& \Skip \lor \W(\Skip\lor Q) & \elabel{W-exposes-II}
}
We now show that the same holds for a composite $F(C,\ldots)$
that directly includes $P$, or rather its semantics $C$
, along with some other stuff.
Inspection of all the semantic definitions reveals
that the form of the semantics of $F$ is the following,
where $\sigma$ is a ground substitution:
\RLEQNS{
   F(C,\ldots) &=& \W(C\sigma \lor \ldots)
}
We now can easily show that we get stuttering here also:
\RLEQNS{
  && \W(C\sigma \lor \dots)
\EQ{$C$ is meaning of $P$}
\\&& \W(\W(P)\sigma \lor \dots)
\EQ{\eref{hyp:P-stutters}}
\\&& \W(\W(\Skip \lor P)\sigma \lor \dots)
\EQ{\eref{W-exposes-II}}
\\&& \W((\Skip\lor\W(\Skip \lor P))\sigma \lor \dots)
\EQ{substitution}
\\&& \W(\Skip\sigma\lor\W(\Skip \lor P)\sigma \lor \dots)
\EQ{Lemma $\Skip\sigma=\Skip$}
\\&& \W(\Skip\lor\W(\Skip \lor P)\sigma \lor \dots)
\EQ{reverse first two steps above}
\\&& \W(\Skip\lor C\sigma \lor \dots)
}


\subsection{Compact Semantics}

\RLEQNS{
   \lhsCA &\defs& \rhsCA & \eref{\lblCA}
\\ \lhsSEQ &\defs& \rhsSEQ & \eref{\lblSEQ}
\\ \lhsPAR
   &\defs&
   \leftW \rhsaPAR \lor {}
   & \eref\lblPAR
\\&& \phW \rhsbPAR \lor {}
\\&& \phW \rhscPAR \lor {}
\\&& \phW \rhsdPAR \rghtW
\\ \lhsNDC &\defs& \invNDC \land \rhsNDC & \eref{\lblNDC}
\\ \lhsSTAR
   &\defs&
   \leftW \rhsaSTAR \lor {}
   & \eref\lblSTAR
\\&& \phW \rhsbSTAR \lor {}
\\&& \phW \rhscSTAR \lor {}
\\&& \phW \rhsdSTAR \rghtW
\\ \lhsCND
   &\defs&
   \leftW \rhsaCND \lor {}
   & \eref\lblCND
\\&& \phW \rhsbCND \lor {}
\\&& \phW \rhscCND \lor {}
\\&& \phW \rhsdCND \rghtW
\\&& {} \land \invCND
\\ \lhsITER
   &\defs&
   \leftW \rhsaITER \lor {}
   & \eref\lblITER
\\&& \phW \rhsbITER \lor {}
\\&& \phW \rhscITER \lor {}
\\&& \phW \rhsdITER \rghtW
}
