\section{Semantic Calculations}\label{sec:calc}

\subsection{Composing $A$s}\label{ssec:comp-A}

We have a problem in that actions defined using $A(\dots)$
are not closed under standard UTP sequential composition (\eref{UTP-seq-def}).
We need a more general form,
where the labels removed are not necessarily
exactly those that had to be present to enable the action.
We call this eXtended atomic action $X$ and it is defined as follows:
\RLEQNS{
   X(E|a|R|A)
   &\defs&
   ls(E) \land s' \in \sem a s \land ls' = (ls \setminus R) \cup A
   & \elabel{X-def}
}
We can then re-define $A$ in terms of $X$:
\RLEQNS{
   A(E|a|N)
   &=&
   X(E|a|E|N)
   & \elabel{A-alt-def}
}
The key law is one regarding sequentially composing $X$s:
\RLEQNS{
   \lefteqn{X(E_1|a|R_1|A_1) ; X(E_2|b|R_2|A_2)} &&& \elabel{X-then-X}
\\ &=& (E_2\setminus A_1) \cap R_1 = \emptyset
\\ &\land&
   X(E_1 \cup (E_2\setminus A_1)
       |a\seq b
       |R_1 \cup R_2
       |(A_1 \setminus R_2) \cup  A_2)
}
The condition $(E_2\setminus A_1) \cap R_1 = \emptyset$
characterises all those cases were the second $X$ is enabled
immediately after the first $X$ terminates
(i.e., without any outside interference).

\subsection{Normal Form}\label{sec:normal-form}

All the semantics above have an invariant $I_P$,
zero or more atomic actions ($A_i$),
plus zero or more language sub-component predicates ($P_j$)
assembled into a predicate of the form:
\[
  I_P
  \land
  \W( A_1 \lor \dots \lor A_m
     \lor
     P_1\sigma_1 \lor \dots \lor P_n\sigma_n )
\]
where each $\sigma_i = [G_i,I_i,O_i/g,in,out]$,
with all $G_i$, $I_i$ and $O_i$ being ground
and satisfying the invariant $[I_i|labs(G_i)|O_i]$.

\subsubsection{Stuttering Propagates Up}

We show that for any predicate $P$ resulting from the
semantics applied to a program,
that the stuttering on such a predicate propagates up:
\RLEQNS{
   P &=& \Skip \lor P
}



\subsubsection{Nondeterministic Iteration}

Given that we have,
for predicates $P$ used in program semantic definitions,
that
\[
  \W(P) = \bigvee_{i \in 0\dots} \Skip \seq P^i
\]
we find that the following outcomes result:
\begin{enumerate}
  \item
    Every program $P$ can be transformed,
    in principle at least,
    to the form
    \[
      I_P \land \Skip
      \lor
      \bigvee_{n \in 0\dots}  (I_P \land I_n \land X(E_n|a_n|R_n|A_n) \land S_n)
    \]
    for suitable choices of $I_n$, $E_n$, $a_n$, $R_n$, $A_n$ and $S_n$,
    depending on $P$.
    In particular, we find that the $E_n$ and $A_n$ all satisfy the relevant
    label-set invariants.
  \item
    For any complete program that does not use iteration constructs,
    this expansion is finite.
  \item
    For any complete program the $S_n$
    term in $X(E_n|a_n|R_n|A_n) \land S_n$
    is a ground-expression, and is statically decidable as $\true$ or $\false$.
    If $\true$, we find that the above term
    can be be simplified to one of the form $A(E_n|a_n|A_n)$.
    We do not prove this, but note that we have observed it as a result
    of test calculations and these lead us to believe it is true in general.
\end{enumerate}
\RLEQNS{
   P
   &=&
   I_P \land \Skip
      \lor
      \bigvee_{n \in 0\dots}
      (I_P \land I_n \land X(E_n|a_n|R_n|A_n) \land S_n)
   & \elabel{UTCP-NF}
}
This can be shown by an inductive argument over the language syntax,
noting that the sequential composition of two such forms
can itself be transformed into such a form.
In effect this defines a \emph{normal form} for our semantics.


An important law we shall require is that the UTP sequential
composition of two normal forms can itself be expressed
as such a normal form:
\RLEQNS{
   I \land (\Skip\lor\bigvee A_i) \seq J \land (\Skip\lor\bigvee B_j)
   &=&
   I \land J \land (\Skip\lor\bigvee (A_i \seq B_j))
   &\elabel{nf-seq}
}
This is an easy result of the fact that $\seq$ distributes through $\lor$,
and $\Skip$ is a unit for it as well.


\subsection{Normal Form Results for Atomic components}

We now present the results of calculations of normal forms for every language
construct, given atomic sub-components.
For ease of reading,
we display the invariant $I$ on the first line,
followed by the list of $A(\dots)$ disjuncts, perhaps several on a line,
as follows
\RLEQNS{
   res &=& I
\\ && A(E_1|a_1|N_1) \quad A(E_1|a_2|N_2)) \quad \dots
}
We omit parentheses or logical operators.
In general we try to put all the single atomic steps on the first line,
with all the mumblings on lower lines.
Here we also just write $a$ instead of $\catom a$,
for brevity.

\subsubsection{Atomic Actions}
\RLEQNS{
   a &=& [in|out]
\\ && A(in|a|out) & \elabel{NF:atomic-a}
\\
\\ \cskip &=& [in|out]
\\ && A(in|ii|out) & \elabel{NF-skip}
}

\subsubsection{Sequential Composition}
\RLEQNS{
   a \cseq b
   &=& [in|\ell_g|out]
\\ && A(in|a|\ell_g)\quad  A(\ell_g|b|out)
\\ && A(in|a\seq b|out)
}

\subsubsection{Non-deterministic Choice}
\RLEQNS{
   a + b
   &=& [in|\ell_{g1}|\ell_{g2}|out]
\\ &&  A(in|ii|\ell_{g1}) \quad A(in|ii|\ell_{g2})
       \quad
       A(\ell_{g1}|a|out) \quad A(\ell_{g2}|b|out)
\\&& A(in|a|out) \quad A(in|b|out)
}

\subsubsection{Parallel Composition}
Here $\#n$ denotes mumblings of length $n$.
\RLEQNS{
   a \parallel b
   &=& [in\mid([\ell_{g1}|\ell_{g1:}],[\ell_{g2}|\ell_{g2:}])\mid out]
\\ &&  A(in|ii|\ell_{g1},\ell_{g2})
       \quad
       A(\ell_{g1:},\ell_{g2:}|ii|out)
       \quad
       A(\ell_{g1}|a|\ell_{g1:})
       \quad
       A(\ell_{g2}|b|\ell_{g2:}) &
\\ && A(in|a|\ell_{g1:},\ell_{g2})
      \quad
      A(\ell_{g1},\ell_{g2}|b;a|\ell_{g1:},\ell_{g2:})
      \quad
      A(in|b|\ell_{g2:},\ell_{g1}) & \#2
\\ && A(\ell_{g1},\ell_{g2}|a ; b|\ell_{g1:},\ell_{g2:})
      \quad
      A(\ell_{g2:},\ell_{g1}|a|out)
      \quad
      A(\ell_{g1:},\ell_{g2}|b|out) & \#2
\\ && A(in|b;a|\ell_{g1:},\ell_{g2:})
      \quad
      A(in|a;b|\ell_{g1:},\ell_{g2:}) & \#3
\\ && A(\ell_{g1},\ell_{g2}|b;a|out)
      \quad
      A(\ell_{g1},\ell_{g2}|a;b|out) & \#3
\\ && A(in|b;a|out)
      \quad
      A(in|a;b|out) & \#4
}

\subsubsection{Non-deterministic Iteration}
The actions that denote a terminating
run with zero or more complete iterations
are \textbf{emboldended} below.
\RLEQNS{
   a^*
   &=& [in|\ell_g|out]
\\ &&  \mathbf{A(in|ii|out)} \quad A(in|ii|\ell_g) \quad A(\ell_g|a|in)
\\ &&  A(\ell_g|a|out) \quad A(\ell_g|a|\ell_g) \quad A(in|a|in)         &\#2
\\ &&  \mathbf{A(in|a|out)} \quad A(in|a|\ell_g) \quad A(\ell_g|a^2|in)  &\#3
\\ &&  A(\ell_g|a^2|out) \quad A(\ell_g|a^2|\ell_g) \quad A(in|a^2|in)   &\#4
\\ &&  \mathbf{A(in|a^2|out)} \quad A(in|a^2|\ell_g) \quad A(\ell_g|a^3|in) &\#5
\\ &&  A(\ell_g|a^3|out) \quad A(\ell_g|a^3|\ell_g) \quad A(in|a^3|in) &\#6
\\ && \vdots & \#7\dots
}
An obvious recurrent pattern (based on mumbling depth) has emerged:
\RLEQNS{
   &&  \mathbf{A(in|a^n|out)} \quad A(in|a^n|\ell_g) \quad A(\ell_g|a^{n+1}|in) &\#2n+1
\\ &&  A(\ell_g|a^{n+1}|out) \quad A(\ell_g|a^{n+1}|\ell_g) \quad A(in|a^{n+1}|in) &\#2n+2
}
We can also identify a pattern based on number of $a$s performed:
\RLEQNS{
   \alpha_0 && \mathbf{A(in|ii|out)} \quad A(in|ii|\ell_g)
\\ \alpha_1 && A(\ell_g|a|in) \quad A(\ell_g|a|out) \quad A(\ell_g|a|\ell_g) \quad A(in|a|in)
\\          && \mathbf{A(in|a|out)} \quad A(in|a|\ell_g)
\\ \alpha_2 && A(\ell_g|a^2|in) \quad A(\ell_g|a^2|out) \quad A(\ell_g|a^2|\ell_g) \quad A(in|a^2|in)
\\          && \mathbf{A(in|a^2|out)} \quad A(in|a^2|\ell_g)
\\
\\ \alpha_n && A(\ell_g|a^n|in) \quad A(\ell_g|a^n|out) \quad A(\ell_g|a^n|\ell_g) \quad A(in|a^n|in)
\\          && \mathbf{A(in|a^n|out)} \quad A(in|a^n|\ell_g)
}


\newpage
\subsection{Essential Properties?}

\RLEQNS{
   X(E|a|R|A)\gamma &=& X(E\gamma|a|R\gamma|A\gamma)
   & \elabel{X-gnd-subs}
\\ (\bigvee_n X_n)^2
   &=& \bigvee_{n_1,n_2} X_{n_1} \seq X_{n_2}
   & \elabel{NF-squared}
\\ (\bigvee_p Y_p \lor \bigvee_q Z_q)^2
     &=& \bigvee_{p_1,p_2}  (Y_{p_1} \seq Y_{p_2})
\\&\lor& \bigvee_{q_1,q_2}  (Z_{q_1} \seq Z_{q_2})
\\&\lor& \bigvee_{p,q}  (Y_p \seq Z_q)
\\&\lor& \bigvee_{q,p}  (Z_q \seq Y_p)
  & \elabel{NF+NF-squared}
\\ &=& \bigvee_r X_r,\textrm{ for appropriate } r
  & \elabel{NF-squared-alt}
\\ (\bigvee_p Y_p \lor \bigvee_q Z_q)^n
     &=& \bigvee_r X_r,\textrm{ for appropriate } r
  & \elabel{NF-powered}
\\ \W(\Skip \lor \bigvee_p Y_p \lor \bigvee_q Z_q)
     &=& \bigvee_r X_r,\textrm{ for appropriate } r
  & \elabel{W-of-NDC-is-NF}
}
Add in X-then-X and related properties.

Nice to have:
the combined invariants ensure that the normal form
has the form:
\[
  P = \bigvee_n A(E_n|a_n|N+n)  \qquad \elabel{NF-only-As}
\]
Is this the case?

\subsubsection{STOP PRESS! $P^2=P$ ?}

The semantics of a program $P$
contains a stuttering step $\Skip$,
all the single atomic actions $A(\dots)$,
and all the feasible mumblings
$X(\dots) = A(\dots) \seq A(\dots) \seq \dots \seq A(\dots)$.
Calculations for law $\cskip\cseq P = P$ suggest the following conjecture:
\begin{center}
\fbox{$\large P \seq P = P \qquad \elabel{mumble-closure}$}
\end{center}
This is because combining elements of $P$ with each other results
in either infeasibility, or some other already present mumble.
\newpage
\subsection{Miscellaneous}

More stuff for proofs section.


We shall first demonstrate that the stuttering $\Skip$
introduced in the atomic action semantics
propagates up into the program composites,
so that, for example, we could have defined sequential composition
as:
\RLEQNS{
   P \cseq Q
   &\defs&
   [in|\ell_g|out]\land
   \W(\Skip \lor P[g_{:1},\ell_g/g,out] \lor Q[g_{:2},\ell_g/g,in])
   & \elabel{sem:seq-alt}
}
We argue by induction over program syntax.

What we shall demonstrate is that, for all commands $C$
with a definition $C = \W(P)$, for $P$ related to $C$ as appropriate,
that the following predicates are all equal:
\RLEQNS{
  \W(\Skip \lor P) \quad = & \W(P)& = \quad \Skip \lor \W(P)
 & \elabel{W-prog-stutters}
}
The base case is easy, here $c$ is $\catom a$,
and $P$ is $\Skip \lor A(in|a|out)$:
\RLEQNS{
  && \W(\Skip \lor (\Skip \lor A(in|a|out)))
\EQ{$\lor$-assoc, $\lor$-idem}
\\&& \W(\Skip \lor A(in|a|out))
\EQ{\eref{sem:atomic}}
\\&& \catom a
}
For the inductive step we assume the hypothesis
for command $C$ and its semantics predicate  $C = \W(P)$,
that:
\RLEQNS{
   \W(P) &=& \W(\Skip \lor P) &\elabel{hyp:P-stutters}
}
We now make use of the following deduction:
\RLEQNS{
  && \W(\Skip \lor Q)
\EQ{loop unrolling}
\\&& (\Skip\lor Q)\seq(\Skip\lor Q)\seq(\Skip\lor Q)\seq(\Skip\lor Q)\seq\dots
\EQ{$\lor$-$\seq$-distr, diagonalisation, removing identical terms}
\\&& \bigvee_{i \in \Nat} \Skip\seq Q^i & \elabel{W-stutter-is-NDC}
}
An immediate consequence of this is that
\RLEQNS{
  \W(\Skip \lor Q) &=& \Skip \lor \W(\Skip\lor Q) & \elabel{W-exposes-II}
}
We now show that the same holds for a composite $F(C,\ldots)$
that directly includes $P$, or rather its semantics $C$
, along with some other stuff.
Inspection of all the semantic definitions reveals
that the form of the semantics of $F$ is the following,
where $\sigma$ is a ground substitution:
\RLEQNS{
   F(C,\ldots) &=& \W(C\sigma \lor \ldots)
}
We now can easily show that we get stuttering here also:
\RLEQNS{
  && \W(C\sigma \lor \dots)
\EQ{$C$ is meaning of $P$}
\\&& \W(\W(P)\sigma \lor \dots)
\EQ{\eref{hyp:P-stutters}}
\\&& \W(\W(\Skip \lor P)\sigma \lor \dots)
\EQ{\eref{W-exposes-II}}
\\&& \W((\Skip\lor\W(\Skip \lor P))\sigma \lor \dots)
\EQ{substitution}
\\&& \W(\Skip\sigma\lor\W(\Skip \lor P)\sigma \lor \dots)
\EQ{Lemma $\Skip\sigma=\Skip$}
\\&& \W(\Skip\lor\W(\Skip \lor P)\sigma \lor \dots)
\EQ{reverse first two steps above}
\\&& \W(\Skip\lor C\sigma \lor \dots)
}
